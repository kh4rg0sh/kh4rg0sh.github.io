\begin{problem}
Let $\triangle S_1S_2S_3$ be an equilateral triangle, and $P$ be any point. Let $Q_1$ be the intersection of the perpendicular bisector of $PS_1$ with $S_2S_3$, and define $Q_2$, $Q_3$ similarly. Prove that $Q_1$, $Q_2$ and $Q_3$ are collinear.
\end{problem}

\begin{problem}
Let $\triangle S_1S_2S_3$ be an equilateral triangle, and $P$ be any point. Prove that the Euler lines of triangles $\triangle PS_2S_3$, $\triangle PS_3S_1$, and $\triangle PS_1S_2$ concur.
\end{problem}

\begin{problem}
Let $I$ be the incenter of $\triangle ABC$, $D$ be the foot of the perpendicular from $I$ to $BC$, and $M$ be the midpoint of $\overline{BC}$. Prove that $IM$ bisects $AD$.
\end{problem}

%     \begin{problem}
%     Let $\triangle S_1S_2S_3$ be an equilateral triangle, and $P$ be any point. Let $Q_1$ be the intersection of the perpendicular bisector of $PS_1$ with $S_2S_3$, and define $Q_2$, $Q_3$ similarly. Prove that $Q_1$, $Q_2$, and $Q_3$ are collinear.
%     \end{problem}

%     \begin{problem}
%     Let $\triangle S_1S_2S_3$ be an equilateral triangle, and $P$ be any point. Prove that the Euler lines of triangles $\triangle PS_2S_3$, $\triangle PS_3S_1$, and $\triangle PS_1S_2$ concur.
%     \end{problem}

%     \begin{problem}
%     Let $I$ be the incenter of $\triangle ABC$, $D$ be the foot of the perpendicular from $I$ to $BC$, and $M$ be the midpoint of $\overline{BC}$. Prove that $IM$ bisects $AD$.
%     \end{problem}

%     \begin{problem}
%     Let $O$ be the circumcenter of $\triangle ABC$, and let points $P \in CA$ and $Q \in AB$ satisfy that $P$, $O$, $Q$ are collinear. Let $M$ and $N$ be the midpoints of $\overline{BP}$ and $\overline{CQ}$, respectively. Prove that $\angle BAC = \angle MON$.
%     \end{problem}

%     \begin{problem}
%     Let $AB$ and $AC$ be two distinct rays, and let $\omega$ be a circle centered at $O$ tangent to $AC$ at $E$ and to $AB$ at $F$. Let $R$ be a point on $\overline{EF}$, and draw a line through $O$ parallel to $\overline{EF}$ intersecting $AB$ at $P$. Let $N$ be the intersection of $\overline{PR}$ with $AC$, and let $M$ be the intersection of $AB$ with the line through $R$ parallel to $AC$. Prove that $\overline{MN}$ is tangent to $\omega$.
%     \end{problem}

%     \begin{problem}
%     Let $ABCD$ be a cyclic quadrilateral, and let $P$ be the intersection of its diagonals $\overline{AC}$ and $\overline{BD}$. Let $M_{AB}, \dots, M_{DA}$ be the midpoints of $\overline{AB}, \dots, \overline{DA}$, and let $I_{AB}, \dots, I_{DA}$ be the incenters of triangles $\triangle PAB, \dots, \triangle PDA$. Prove that the four lines $M_{AB}I_{AB}, \dots, M_{DA}I_{DA}$ are concurrent.
%     \end{problem}

%     \begin{problem}
%     Let $H$ be the orthocenter of $\triangle ABC$, $M$ the midpoint of $\overline{AH}$, and $E$, $F$ the feet of perpendiculars from $B$, $C$ to $\overline{CA}$, $\overline{AB}$, respectively. On $\overline{EM}$ choose a point $R$ such that $\angle RBC = 90^\circ$, and on $\overline{FM}$ choose a point $S$ such that $\angle BCS = 90^\circ$. Prove that $A$, $R$, and $S$ are collinear.
%     \end{problem}

%     \begin{problem}
%     Let $\triangle DEF$ be the triangle formed by the arc midpoints of $\triangle ABC$. Let $X_1, X_2, Y_1, Y_2, Z_1, Z_2$ be the intersections of $\triangle ABC$ and $\triangle DEF$, ordered as $D, Z_1, Z_2, E, X_1, X_2, F, Y_1, Y_2, D$. Define
%     \[
%     P_{bc} = \overline{EY_1} \cap \overline{FZ_2}, \quad
%     P_{cb} = \overline{FZ_1} \cap \overline{EY_2},
%     \]
%     and similarly $P_{ca}, P_{ac}, P_{ab}, P_{ba}$. Prove that the three lines $P_{bc}P_{cb}, P_{ca}P_{ac}, P_{ab}P_{ba}$ are concurrent.
%     \end{problem}

%     \begin{problem}
%     Let $I$ be the incenter of $\triangle ABC$, and let $\ell$ be a tangent to its incircle $\omega$. Let $A', B', C'$ be three collinear points on $\overline{BC}, \overline{CA}, \overline{AB}$, respectively. Let $A^*$ be the intersection of $\ell$ with the tangent to $\omega$ from $A'$ (other than $\overline{BC}$), and define $B^*, C^*$ similarly. Prove that lines $AA^*, BB^*, CC^*$ are concurrent.
%     \end{problem}

%     \begin{problem}
%     On the sides of an equilateral triangle $\triangle ABC$, choose six points $A_1, A_2 \in \overline{BC}$, $B_1, B_2 \in \overline{CA}$, $C_1, C_2 \in \overline{AB}$ such that the hexagon $A_1A_2B_1B_2C_1C_2$ has all sides equal. Prove that lines $\overline{A_1B_2}, \overline{B_1C_2}, \overline{C_1A_2}$ are concurrent.
%     \end{problem}

%     \begin{problem}
%     Let $I$ be the incenter of $\triangle ABC$, and let $P$ be the inversion of $I$ with respect to $\odot(ABC)$. Let $X, Y$ be the intersections of one of the tangents from $P$ to the incircle of $\triangle ABC$ with $\odot(ABC)$. Prove that $\angle XIY = 120^\circ$.
%     \end{problem}

%     \begin{problem}
%     Let the incircle $\omega$ of $\triangle ABC$ touch $\overline{BC}, \overline{CA}, \overline{AB}$ at $D, E, F$, respectively. Let $\overline{AD}$ intersect $\omega$ at $K$, and let the tangent to $\omega$ at $K$ intersect $\overline{FD}$ and $\overline{DE}$ at $Y, Z$. Prove that lines $\overline{AD}, \overline{BZ}, \overline{CY}$ are concurrent.
%     \end{problem}

%     \begin{problem}
%     Let $P_1, P_2, \dots, P_8$ be eight points in the plane, no three collinear. Define
%     \[
%     Q_i := \overline{P_{i-2}P_{i-1}} \cap \overline{P_{i+1}P_{i+2}}.
%     \]
%     Prove that $P_1, \dots, P_8$ lie on a conic if and only if $Q_1, \dots, Q_8$ lie on a conic.
%     \end{problem}
