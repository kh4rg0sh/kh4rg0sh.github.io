We first introduce one way to define conic sections from high school geometry:

\begin{definition}
    Let $\odot(O)$ be a circle in space. Draw a line through $O$ such that $\ell$ is perpendicular to the plane of $\odot(O)$. Take any point $V$ on $\ell$ such that $V$ $\neq$ $O$. When a moving point $M$ $\in$ $\odot(O)$ moves along the circle, the surface formed by the lines $VM$ is called a \vocab{Circular Conic Surface}. The circle $\odot(O)$ is called its \vocab{Directrix}, and $V$ is called its \vocab{Vertex}.
\end{definition}

\begin{definition}
    A curve $\mathcal{C}$ on a plane $E$ is called a \vocab{Conic} if there exists a circular conical surface $S$ with vertex $V$ such that $\mathcal{C}$ $=$ $S \cap E$.
\end{definition}

Later, when the spatial context is not mentioned, we will omit ``on plane $E$''. Because the equation of a conic is quadratic, conics are also called \emph{quadratic curves}.

\begin{definition}
    Let $\mathcal{C}$ be a conic and $\mathcal{L}_{\infty}$ be the line at infinity. Then:
    \begin{enumerate}
        \item $\mathcal{C}$ is called an $\vocab{Ellipse}$, if $\mid \mathcal{C} \cap \mathcal{L}_{\infty} \mid = 0$.
        \item $\mathcal{C}$ is called an $\vocab{Parabola}$, if $\mid \mathcal{C} \cap \mathcal{L}_{\infty} \mid = 1$.
        \item $\mathcal{C}$ is called an $\vocab{Hyperbola}$, if $\mid \mathcal{C} \cap \mathcal{L}_{\infty} \mid = 2$.
    \end{enumerate}
\end{definition}
Hereafter, $\mathcal{L}_{\infty}$ will be used to denote the \vocab{Line at Infinity}. For a line $\ell$ $\neq$ $\mathcal{L}_{\infty}$, the intersection point 
\begin{align*}
    \infty_{\ell} \coloneqq \ell \cap \mathcal{L}_{\infty}
\end{align*}
is called the \vocab{Point at Infinity of $\ell$}. The following are the more commonly seen equivalent definitions of conic sections: