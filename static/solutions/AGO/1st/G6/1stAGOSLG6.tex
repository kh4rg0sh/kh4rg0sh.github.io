\documentclass[11pt]{scrartcl}
\let\captionof\undefined
\usepackage[sexy,von]{evan}
\usepackage{wrapfig}
% \renewcommand{\vonenvname}{example}
\lstset{basicstyle=\small\ttfamily,
  numbers=left,
  numbersep=5pt,
  numberstyle=\tiny,
  keywordstyle=\bfseries,
  showstringspaces=false,
  tabsize=4,
  frame=single,
  keywordstyle=\bfseries\color{blue},
  commentstyle=\color{green!70!black},
  identifierstyle=\color{green!20!black},
  stringstyle=\color{orange},
  breaklines=true,
  breakatwhitespace=true,
  frame=none
}

\usepackage{xcolor}
\setkomafont{captionlabel}{\bfseries\color{red}}
\renewcommand*{\figurename}{Fig}

\usepackage{answers}
\usepackage{cancel}
\usepackage{asymptote}

\begin{document}
\title{1st AGO Shortlist G6}
\date{\today}
\maketitle

\begin{abstract}
    \centering
\end{abstract}

\section{Problem}
\begin{problem*}[1st AGO Shortlist G6]
    Let $ABC$ be a triangle. Incircle of $ABC$ touch sides $AC$ and $AB$ at $E$ and $F$ respectively. Angle bisectors of $\angle ABC$ and $\angle BCA$ intersect line $EF$ at $P$ and $Q$, respectively. Lines $AP$ and $AQ$ intersect the circumcircle of $ABC$ again at $X$ and $Y$ , respectively. Prove that if $X$, midpoint of $BC$ and $Y$ are collinear, then $XY$ is perpendicular to angle bisector of $BAC$.
\end{problem*}

\section{Solution}
\begin{figure}[h]
    \centering
    % \begin{asy}
    % \end{asy}
\end{figure}

\begin{proof}
    
\end{proof}
\end{document}