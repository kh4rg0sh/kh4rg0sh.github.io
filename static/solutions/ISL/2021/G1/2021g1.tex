\documentclass[11pt]{scrartcl}
\let\captionof\undefined
\usepackage[sexy,von]{evan}
\usepackage{wrapfig}
% \renewcommand{\vonenvname}{example}
\lstset{basicstyle=\small\ttfamily,
  numbers=left,
  numbersep=5pt,
  numberstyle=\tiny,
  keywordstyle=\bfseries,
  showstringspaces=false,
  tabsize=4,
  frame=single,
  keywordstyle=\bfseries\color{blue},
  commentstyle=\color{green!70!black},
  identifierstyle=\color{green!20!black},
  stringstyle=\color{orange},
  breaklines=true,
  breakatwhitespace=true,
  frame=none
}

\usepackage{xcolor}
\setkomafont{captionlabel}{\bfseries\color{red}}
\renewcommand*{\figurename}{Fig}

\usepackage{answers}
\usepackage{cancel}
\usepackage{asymptote}

\begin{document}
\title{IMO Shortlist 2021 G1}
\date{\today}
\maketitle

\begin{abstract}
    \centering https://artofproblemsolving.com/community/c6h2882542p25627509
\end{abstract}

\tableofcontents

\section{Problem}
\begin{problem*}[IMO Shortlist 2021 G1]
    Let $ABCD$ be a parallelogram with $AC=BC.$ A point $P$ is chosen on the extension of ray $AB$ past $B.$ The circumcircle of $ACD$ meets the segment $PD$ again at $Q.$ The circumcircle of triangle $APQ$ meets the segment $PC$ at $R.$ Prove that lines $CD,AQ,BR$ are concurrent.
\end{problem*}

\section{Solution 1 (Using Angle Chasing)}
\begin{figure}[h]
    \centering
    \begin{asy}
        import geometry;
        size(8cm); defaultpen(fontsize(10pt));
        
        pair A, B, C, D, E, P, Q, R, S, T;
        A = (0, 0); B = (3, 0); P = (4.3, 0); C = (1.5, 3.8); D = C + (A - B);
        pair[] QQ = intersectionpoints(line(D, P), circumcircle(A, C, D));
        pair[] RR = intersectionpoints(line(C, P), circumcircle(A, QQ[0], P));
        
        T = extension(C, D, A, QQ[0]);

        draw(A--B--C--D--cycle);
        draw(B--P); draw(A--RR[1]); 
        draw(P--D); draw(A--C);
        draw(C--T); draw(A--T);
        draw(P--C); draw(B--T, gray+dashed); draw(C--QQ[0], gray);
        draw(circumcircle(A, P, QQ[0])); draw(circumcircle(A, C, D));

        dot("$A$", A, dir(210)); dot("$B$", B, dir(315));
        dot("$C$", C, dir(105)); dot("$D$", D, dir(135));
        dot("$P$", P, dir(330)); dot("$Q$", QQ[0], dir(295));
        dot("$T$", T, dir(45)); dot("$R$", RR[1], dir(0));

        draw(circumcircle(A, B, C), heavygray+dashed);
        draw(circumcircle(C, T, RR[1]), heavygray+dashed);
    \end{asy}
\end{figure}
\begin{proof}
    Define $CD$ $\cap$ $AQ$ as $T$. We want to show that $BR$ passes through $T$ too.
    \begin{claim}\label{claim:1}
        Quadrilateral $CABR$ is cyclic.
    \end{claim}
    \begin{proof}
        This is just angle chasing.
        \begin{align*}
            \angle CRA &= \angle CRQ + \angle QRA \\ 
                        &= \angle QAP + \angle QPA \\ 
                        &= \angle QAP + \angle QDC \\ 
                        &= \angle QAP + \angle QAC \\ 
                        &= \angle CAB = \angle ABC
        \end{align*}
        This implies that $CABR$ is a cyclic quadrilateral.
    \end{proof}
    
    \begin{claim}
        Quadrilateral $CQRT$ is cyclic.
    \end{claim}
    \begin{proof}
        Again we use angle chasing to prove this.
        \begin{align*}
            \angle CRQ &= \angle QAP \\
                        &= \angle QTC
        \end{align*}
        which implies that $CQRT$ is a cyclic quadrilateral.
    \end{proof}
    As a result, we have
    \begin{align*}
        \angle CRB + \angle CRT &= 180^{\circ} - \angle CAB + \angle CQT \\ 
                            &= 180^{\circ} - \angle CBA + \angle CDA \\
                            &= 180^{\circ}
    \end{align*}
    Therefore, points $B$, $R$ and $T$ are collinear.
\end{proof}

\section{Solution 2 (Using Radical Axis Theorem)}
\begin{figure}[h]
    \centering
    \begin{asy}
        import geometry;
        size(8cm); defaultpen(fontsize(10pt));
        
        pair A, B, C, D, E, P, Q, R, S, T;
        A = (0, 0); B = (3, 0); P = (4.3, 0); C = (1.5, 3.8); D = C + (A - B);
        pair[] QQ = intersectionpoints(line(D, P), circumcircle(A, C, D));
        pair[] RR = intersectionpoints(line(C, P), circumcircle(A, QQ[0], P));
        
        T = extension(C, D, A, QQ[0]);

        draw(A--B--C--D--cycle);
        draw(B--P); draw(A--RR[1]); 
        draw(P--D); draw(A--C);
        draw(C--T); draw(A--T);
        draw(P--C); draw(B--T, gray+dashed); draw(C--QQ[0], gray);
        draw(circumcircle(A, P, QQ[0])); draw(circumcircle(A, C, D));

        dot("$A$", A, dir(210)); dot("$B$", B, dir(315));
        dot("$C$", C, dir(105)); dot("$D$", D, dir(135));
        dot("$P$", P, dir(330)); dot("$Q$", QQ[0], dir(295));
        dot("$T$", T, dir(45)); dot("$R$", RR[1], dir(0));

        draw(circumcircle(A, B, C), heavygray+dashed);
        
        pair X = extension(A, D, B, RR[1]);
        draw(circumcircle(C, D, X), heavygray+dashed);
        dot("$X$", X, dir(245)); draw(A--X); draw(X--RR[1]);
    \end{asy}
\end{figure}
\begin{proof}
    Define $BR$ $\cap$ $\odot(AQP)$ $=$ $X$.
    \begin{claim}
        Quadrilateral $CABR$ is cyclic.
    \end{claim}
    \begin{proof}
        Same as \hyperref[claim:1]{Claim 1.1}.
    \end{proof}

    \begin{claim}
        $X$ lies on the line $AD$.
    \end{claim}
    \begin{proof}
        To prove the collinearity, we simply angle chase.
        \begin{align*}
            \angle DAB + \angle XAB &= 180^{\circ} - \angle ABC + \angle XRP \\
                                    &= 180^{\circ} - \angle CAB + \angle CAB \\
                                    &= 180^{\circ} 
        \end{align*}
        This implies that $X$ lies on $AD$.
    \end{proof}

    \begin{claim}
        Quadrilateral $CDXR$ is cyclic.
    \end{claim}
    \begin{proof}
        This is again angle chasing.
        \begin{align*}
            \angle CDX = \angle CBA = \angle CAB = 180^{\circ} - \angle CRB
        \end{align*}
        which implies that $CDXR$ is a cyclic quadrilateral.
    \end{proof}

    Applying the radical axis theorem on $\odot(AQCD)$, $\odot(AQRP)$ and $\odot(CDXR)$, we get that $CD$, $AQ$ and $BR$ are concurrent.
\end{proof}
\end{document}