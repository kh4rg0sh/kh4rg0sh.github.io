\documentclass[11pt]{scrartcl}
\let\captionof\undefined
\usepackage[sexy,von]{evan}
\usepackage{wrapfig}
% \renewcommand{\vonenvname}{example}
\lstset{basicstyle=\small\ttfamily,
  numbers=left,
  numbersep=5pt,
  numberstyle=\tiny,
  keywordstyle=\bfseries,
  showstringspaces=false,
  tabsize=4,
  frame=single,
  keywordstyle=\bfseries\color{blue},
  commentstyle=\color{green!70!black},
  identifierstyle=\color{green!20!black},
  stringstyle=\color{orange},
  breaklines=true,
  breakatwhitespace=true,
  frame=none
}

\usepackage{xcolor}
\setkomafont{captionlabel}{\bfseries\color{red}}
\renewcommand*{\figurename}{Fig}

\usepackage{answers}
\usepackage{cancel}
\usepackage{asymptote}

\begin{document}
\title{IMO Shortlist 2024 G2}
\date{\today}
\maketitle

\begin{abstract}
    \centering
    https://artofproblemsolving.com/community/c6h3359767p31218657
\end{abstract}

\tableofcontents

\section{Problem}
\begin{problem*}[IMO Shortlist 2024 G2]
    Let $ABC$ be a triangle with $AB < AC < BC$. Let the incenter and incircle of triangle $ABC$ be $I$ and $\omega$, respectively. Let $X$ be the point on line $BC$ different from $C$ such that the line through $X$ parallel to $AC$ is tangent to $\omega$. Similarly, let $Y$ be the point on line $BC$ different from $B$ such that the line through $Y$ parallel to $AB$ is tangent to $\omega$. Let $AI$ intersect the circumcircle of triangle $ABC$ at $P \ne A$. Let $K$ and $L$ be the midpoints of $AC$ and $AB$, respectively. Prove that $\angle KIL + \angle YPX = 180^{\circ}$.
\end{problem*}

\section{Solution 1 (Using Homothety)}
\begin{figure}[h]
    \centering
    \begin{asy}
        import geometry;

        defaultpen(fontsize(11pt));
        size(9cm);

        pair A, B, C, D, K, L, I, X, Y, J, P, Q, R, S;
        A = dir(110);
        B = dir(200);
        C = dir(340);
        I = incenter(A, B, C);
        K = (A+C)/2;
        L = (A+B)/2;
        J = 2*I-A;
        D = extension(A, I, B, C);
        X = extension(B, C, J, J+A-C);
        Y = extension(B, C, J, J+A-B);
        P = dir(270);
        Q = IP(A--(2*A-P), circumcircle(B, J, C));
        R = extension(J, X, A, B);
        S = extension(J, Y, A, C);
        draw(A--B--C--cycle);
        draw(circumcircle(A, B, C));
        draw(incircle(A, B, C));
        draw(R--J--S);
        draw(Y--P--X, gray);
        draw(B--J--C, heavygray);
        draw(K--I--L--cycle, heavygray);
        draw(A--P); draw(circumcircle(B, P, X), heavygray+dashed);
        draw(circumcircle(C, P, Y), heavygray+dashed);

        dot("$A$", A, dir(130));
        dot("$B$", B, dir(B));
        dot("$C$", C, dir(C));
        dot("$P$", P, dir(220));
        dot("$I$", I, dir(210));
        dot("$A'$", J, dir(200));
        dot("$D$", D, dir(240));
        dot("$X$", X, dir(240));
        dot("$Y$", Y, dir(320));
        dot("$L$", L, dir(150));
        dot("$K$", K, dir(40));
        
    \end{asy}
\end{figure}
\begin{proof}
    Let $A'$ be the reflection of point $A$ over point $I$.
    \begin{claim}
        $A'X$ $\parallel$ $AC$ and $A'Y$ $\parallel$ $AB$.
    \end{claim}
    \begin{proof}
        Construct point $E$ on line segment $AC$ such that $\omega$ touches $AC$ at $E$. Reflect the point $E$ over $I$ to $E'$. By SAS congruency criterion $\triangle$ $AIE$ $\cong$ $A'IE'$. Since, $\angle IE'A'$ $=$ $90^{\circ}$ $\implies$ $A'E'$ is tangent to $\omega$ and $A'E'$ $\parallel$ $AC$. However, $X$ lies on the line parallel to $AC$ and tangent to $\omega$ $\implies$ $A'X$ $\parallel$ $AC$, and similarly $A'Y$ $\parallel$ $AB$ which proves the claim.
    \end{proof}

    \begin{claim}
        $BXA'P$ and $CYA'P$ are cyclic quadrilaterals.
    \end{claim}
    \begin{proof}
        Just angle chasing.
        \begin{align*}
            \angle CXA' = \angle BCA = \angle BPA = \angle BPA'
        \end{align*}which proves that $BXA'P$ is cyclic. Similarly, we can show that $CYA'P$ is cyclic.
    \end{proof}

    \begin{claim}
        $\triangle$ $KIL$ is homothetic to $\triangle$ $CA'B$ from point $A$.
    \end{claim}
    \begin{proof}
        Since lines $BL$, $A'I$ and $CK$ are concurrent at point $A$ and $AB$ $=$ $2AL$, $A'A$ $=$ $2AI$ and $AC$ $=$ $2AK$ $\implies$ $\triangle$ $KIL$ $\mapsto$ $\triangle CA'B$ under a homothetic transformation with scaling factor $=$ $2$.
    \end{proof}
    Finally combining all the information from the proved claims,
    \begin{align*}
        \angle KIL + \angle YPX &= \angle BA'C + \angle XPY \\ 
                    &= \angle BA'C + \angle XPA' + \angle YPA' \\ 
                    &= \angle BA'C + \angle XBA' + \angle YCA' \\ 
                    &= \angle BA'C + \angle CBA' + \angle BCA' \\ 
                    &= 180^{\circ}
    \end{align*}as desired. 
\end{proof}


\end{document}