\documentclass[11pt]{scrartcl}
\let\captionof\undefined
\usepackage[sexy,von]{evan}
\usepackage{wrapfig}
% \renewcommand{\vonenvname}{example}
\lstset{basicstyle=\small\ttfamily,
  numbers=left,
  numbersep=5pt,
  numberstyle=\tiny,
  keywordstyle=\bfseries,
  showstringspaces=false,
  tabsize=4,
  frame=single,
  keywordstyle=\bfseries\color{blue},
  commentstyle=\color{green!70!black},
  identifierstyle=\color{green!20!black},
  stringstyle=\color{orange},
  breaklines=true,
  breakatwhitespace=true,
  frame=none
}

\usepackage{xcolor}
\setkomafont{captionlabel}{\bfseries\color{red}}
\renewcommand*{\figurename}{Fig}

\usepackage{answers}
\usepackage{cancel}
\usepackage{asymptote}

\begin{document}
\title{IMO Shortlist 2019 G1}
\date{\today}
\maketitle

\begin{abstract}
    \centering https://artofproblemsolving.com/community/c6h2278984p17828603
\end{abstract}

\tableofcontents

\section{Problem}
\begin{problem*}[IMO Shortlist 2019 G1]
    Let $ABC$ be a triangle. Circle $\Gamma$ passes through $A$, meets segments $AB$ and $AC$ again at points $D$ and $E$ respectively, and intersects segment $BC$ at $F$ and $G$ such that $F$ lies between $B$ and $G$. The tangent to circle $BDF$ at $F$ and the tangent to circle $CEG$ at $G$ meet at point $T$. Suppose that points $A$ and $T$ are distinct. Prove that line $AT$ is parallel to $BC$.
\end{problem*}

\section{Solution 1 (Using Angle Chasing)}

\begin{figure}[h]
    \centering
    \begin{asy}
        import geometry;
        size(9cm); defaultpen(fontsize(11pt));

        pair A, B, C, D, E, F, G, T;
        A = dir(120); D = dir(185); E = dir(340);
        F = dir(250); G = dir(290); T = dir(60);

        B = extension(A, D, F, G);
        C = extension(A, E, F, G);

        draw(A--B--C--cycle);
        draw(circumcircle(B, D, F));
        draw(circumcircle(G, E, C));
        draw(unitcircle, heavygray+dashed);
        draw(F--T); draw(T--G); draw(A--T);
        draw(D--F); draw(G--E); draw(D--G);

        dot("$A$", A, dir(110));
        dot("$T$", T, dir(60));
        dot("$B$", B, dir(225));
        dot("$C$", C, dir(315));
        dot("$F$", F, dir(315));
        dot("$G$", G, dir(315));
        dot("$D$", D, dir(140));
        dot("$E$", E, dir(120));
    \end{asy}
\end{figure}
\begin{proof}
\begin{claim}
    $T$ lies on $\Gamma$.
\end{claim}
\begin{proof}
    We want to show that $ADFGET$ is cyclic. It suffices to show that $DFGT$ is cyclic. Notice that $\angle DFT$ $=$ $\angle ABC$ and $\angle TGE$ $=$ $\angle ACB$. Since, 
    \begin{align*}
        \angle DGT &= \angle DGE - \angle TGE \\ 
                &= 180 - \angle BAC - \angle ACB \\ 
                &= \angle ABC = \angle DFT 
    \end{align*}
    Hence $\angle DGT$ $=$ $\angle DFT$ $\implies$ $DFGT$ is cyclic.
\end{proof}
From here, it's easy to show that $\overline{AT}$ $\parallel$ $\overline{BC}$, which holds because
\begin{align*}
    \angle TAC = \angle TAE = \angle TGE = \angle ACB
\end{align*}
\end{proof}

\section{Solution 2 (Using Reim's Theorem)}
\begin{figure}[h]
    \centering
    \begin{asy}
        import geometry;
        size(9cm); defaultpen(fontsize(11pt));

        pair A, B, C, D, E, F, G, T;
        A = dir(120); D = dir(185); E = dir(340);
        F = dir(250); G = dir(290); T = dir(60);

        B = extension(A, D, F, G);
        C = extension(A, E, F, G);

        draw(A--B--C--cycle);
        draw(circumcircle(B, D, F));
        draw(circumcircle(G, E, C));
        draw(unitcircle, heavygray);
        draw(F--T); draw(T--G); draw(A--T);
        draw(D--F); draw(G--E);

        dot("$A$", A, dir(110));
        dot("$T$", T, dir(60));
        dot("$B$", B, dir(225));
        dot("$C$", C, dir(315));
        dot("$F$", F, dir(315));
        dot("$G$", G, dir(315));
        dot("$D$", D, dir(140));
        dot("$E$", E, dir(120));
    \end{asy}
\end{figure}
\begin{proof}
    Suppose the tangents $FT$ and $GT$ meet $\Gamma$ at $T_1$ and $T_2$. Applying reim's theorem on pairs of circles $\odot(BDF)$, $\Gamma$ and $\odot(CGE)$, $\Gamma$ $\implies$ $\overline{AT_1}$ $\parallel$ $\overline{BF}$ and $\overline{AT_2}$ $\parallel$ $\overline{GC}$. Hence $T_1$ $=$ $T_2$, which implies that $\overline{AT}$ $\parallel$ $\overline{BC}$.
\end{proof}

\end{document}