\section{Jan 14, 2026}
\subsection{Dirac Notation}
For the simplicity of discussion, we assume that the dimension of the space is $2$. We first define two linearly independent vectors in space.
\begin{equation*}
    \ket{0} = \twovec{1}{0} \qquad
    \ket{1} = \twovec{0}{1}
\end{equation*}
We can now define the \highlight{ket notation} and the \highlight{bra notation}.
\begin{definition}
    For a state $\psi$, the \vocab{ket notation} is defined as 
    \begin{equation*}
        \ket{\psi} = \psi_0 \ket{0} + \psi_1 \ket{1} = \twovec{\psi_0}{\psi_1}
    \end{equation*}
\end{definition}
\begin{definition}
For any given \emph{ket notation}, there exists a corresponding \emph{bra notation}. The \vocab{bra notation} is defined as the transpose of the complex conjugate of the ket notation.
\begin{equation*}
    \bra{\psi} = \tworow{\psi_0^{*}}{\psi_1^{*}}
\end{equation*}
\end{definition}
\subsection{Inner product and Outer product}
We introduce a few more operations with no real significance yet.
\begin{definition}
Suppose we have two states $\phi$ and $\psi$. The \vocab{inner product} of these two states is denoted and defined as
\begin{align*}
\braket{\phi}{\psi} &= \tworow{\phi_0^{*}}{\phi_1^{*}} \twovec{\psi_0}{\psi_1} \\ 
        &= \phi_0^{*} \psi_0 + \phi_1^{*}\psi_1
\end{align*}
\end{definition}
\begin{definition}
Suppose we have two states $\phi$ and $\psi$. The \vocab{outer product} of these two states is denoted and defined as
\begin{align*}
\ket{\phi} \bra{\psi} &= \twovec{\phi_0}{\phi_1} \tworow{\psi_0^{*}}{\psi_1^{*}} \\ 
                &= \twomat{\phi_0 \psi_0^{*}}{\phi_0 \psi_1^{*}}{\phi_1 \psi_0^{*}}{\phi_1 \psi_1^{*}}
\end{align*}
\end{definition}

\subsection{Unitary Operator}
A unitary operator $\hat{U}$ is said to be linear if its inverse $\hat{U}^{-1}$ is equal to its \vocab{Hermitian Adjoint}.
\begin{equation*}
    \hat{U} \hat{U}^{\dagger} = \hat{U}^{\dagger} \hat{U} = \hat{I} 
\end{equation*}
