\section{Jan 12, 2026}
The recommended book for this course is \highlight{Quantum Computation and Quantum Information}, \emph{Neilsen and Chuang}. It is not meant to be followed linearly. Instead we will jump over topics and read only what is relevant.

\subsection{The Six Postulates of Quantum Mechanics}
\begin{enumerate}
    \item The first postulate describes the state of a quantum mechanical system. It can be completely specified by a function $\psi$ known as the \vocab{wave function}. The wave function is a function of the coodinates in space and time, usually represented as
    \begin{equation*}
        \lvert \psi \left( \textbf{r}, t \right) \rangle
    \end{equation*}
    This function has an important property that $\left( \psi^{*} (\textbf{r}, t) \psi (\textbf{r}, t) dV \right)$ is the probability that the particle lies in the volume element $dV$ located at \textbf{r} at time $t$. The wave function must also satisfy the following normalization condition due to the previous probabilistic interpretation,
    \begin{equation*}
        \int_{-\infty}^{\infty} \psi^{*} (\textbf{r}, t) \psi (\textbf{r}, t) dV = 1
    \end{equation*}
    \item The second postulate describes a correspondence between every observable quantity $A$ in classical mechanics to its quantum counter-part using the linear, Hermitian operator $\hat{A}$ in quantum mechanics.
    \item The third postulate describes the measurements of an observable. The only values that will ever be observed for an observable associated with $\hat{A}$ are its eigenvalues
    \begin{equation*}
        \hat{A} \psi = a \psi
    \end{equation*}
    \item The fourth postulate describes the expectation of an observable corresponding to $\hat{A}$, which is given by
    \begin{equation*}
        \langle A \rangle = \int_{-\infty}^{\infty} \psi^{*} \hat{A} \psi dV
    \end{equation*}
    An immediate consequence of the fourth postulate is that, after the measurement of $\psi$, the wave functions immediately \emph{collapses} into the corresponding eigenstate. In other words, the measurement affects the state of the system.
    \item The fifth postulate describes the \highlight{time evolution} of a wave function. The time evolution of the wave function is governed by the \vocab{Schrödinger Equation}.
    \begin{equation*}
        i\hbar \frac{\partial}{\partial t}\, \lvert \psi(t) \rangle = \hat{H}\, \lvert \psi(t) \rangle
    \end{equation*}
    \item The sixth, and the last postulate, states that the wave function is \emph{symmetric} for particles with integer spin, called the \highlight{bosons}, and it is \emph{antisymmetric} for particles with half-integer spin, called the \highlight{fermions}. The mathematical treatment of this postulate yields the \vocab{Pauli Exclusion Principle} which states that no two identical fermions can occupy the same quantum state.
\end{enumerate}
