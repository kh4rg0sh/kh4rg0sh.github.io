\documentclass[11pt]{scrartcl}
\let\captionof\undefined
\usepackage[sexy,von]{evan}
\usepackage{wrapfig}
% \renewcommand{\vonenvname}{example}
\lstset{basicstyle=\small\ttfamily,
  numbers=left,
  numbersep=5pt,
  numberstyle=\tiny,
  keywordstyle=\bfseries,
  showstringspaces=false,
  tabsize=4,
  frame=single,
  keywordstyle=\bfseries\color{blue},
  commentstyle=\color{green!70!black},
  identifierstyle=\color{green!20!black},
  stringstyle=\color{orange},
  breaklines=true,
  breakatwhitespace=true,
  frame=none
}

\usepackage{xcolor}
\setkomafont{captionlabel}{\bfseries\color{red}}
\renewcommand*{\figurename}{Fig}

\usepackage{answers}
\usepackage{cancel}
\usepackage{asymptote}
\usepackage{hyperref}

\begin{document}
\title{Casey's Theorem}
\date{\today}
\maketitle

\begin{abstract}
    \centering In this article we study Casey's Theorem and its applications.
\end{abstract}
\section{Casey's Theorem}
Casey's Theorem is a result on lengths in the configuration of tangent circles. However, this theorem is also applicable to point circles making it a much more powerful result.
\begin{theorem}[Casey's Theorem]
  Given a circle $\Omega$ and four circles $\omega_1$, $\omega_2$, $\omega_3$, $\omega_4$ internally tangent to $\Omega$ at $A$, $B$, $C$ and $D$. Let $\delta_{ij}$ denote the length of the external common tangent between circles $\omega_i$ and $\omega_j$, then
  \begin{align*}
    \delta_{12} \cdot \delta_{34} + \delta_{23} \cdot \delta_{14} = \delta_{13} \cdot \delta_{24}
  \end{align*}
\end{theorem}

Infact the converse of this theorem holds too!
\begin{theorem}[Converse of Casey's Theorem]
    Given four circles $\omega_1$, $\omega_2$, $\omega_3$ and $\omega_4$ that satisfy
    \begin{align*}
      \pm \delta_{12} \cdot \delta_{34} \pm \delta_{23} \cdot \delta_{14} \pm \delta_{13} \cdot \delta_{24} = 0
    \end{align*}
    where $\delta_{ij}$ is the length of external common tangent of circles $\omega_i$ and $\omega_j$ then there exists a circle that is tangent to all the four circles.
\end{theorem}
Since the proof of this result is challenging, we shall omit it here. Instead, let's learn how to use this result on problems.

\section{Point Circles!}
\vocab{Point circles} refer to circles with zero radius. They are circles that precisely collapse to a singular point. Such circles are also known as \highlight{degenerate circles}. Interestingly, Casey's Theorem is also applicable to point circles! And when we consider all the four circles in Casey's Theorem to be degenerate, the condition boils down to a very famous result in the literature of geometry known as \vocab{Ptolemy's Theorem}. In other words, Casey's Theorem can also be understood as a generalisation of Ptolemy's Theorem.
\subsection{Ptolemy's Theorem}
\begin{theorem}[Ptolemy's Theorem]
  For four points $A$, $B$, $C$ and $D$ in a plane, $ABCD$ is a cyclic quadrilateral if and only if, 
  \begin{align*}
    \overline{AB} \cdot \overline{CD} + \overline{AD} \cdot \overline{BC} = \overline{AC} \cdot \overline{BD}
  \end{align*}
\end{theorem}
Notice how this is occurs exactly when the four circles in Casey's Theorem are chosen to be point circles at $A$, $B$, $C$ and $D$. Since we have mentioned Ptolemy's Theorem, it's also worth mentioning the following result.
\begin{theorem}[Ptolemy's Inequality]
  For four points $A$, $B$, $C$ and $D$ in a plane, we always have 
  \begin{align*}
    \overline{AB} \cdot \overline{CD} + \overline{AD} \cdot \overline{BC} \geq \overline{AC} \cdot \overline{BD}
  \end{align*}
  Equality holds if and only if $ABCD$ is a cyclic quadrilateral.
\end{theorem}

\subsection{Examples}
\begin{problem}[IMO Shortlist 1997]
  The lengths of the sides of a convex hexagon $ ABCDEF$ satisfy $ AB = BC$, $ CD = DE$, $ EF = FA$. Prove that
  \[ \frac {BC}{BE} + \frac {DE}{DA} + \frac {FA}{FC} \geq \frac {3}{2}.
  \]
\end{problem}
\begin{proof}
  Applying ptolemy's inequality on the quadrilateral $ABCE$, we have that 
  \begin{align*}
    \overline{CE} \cdot \overline{AB} + \overline{AE} \cdot \overline{BC} &\geq \overline{AC} \cdot \overline{BE} \\ 
    \implies \frac{\overline{BC}}{\overline{BE}} &\geq \frac{\overline{AC}}{\overline{CE} + \overline{AE}}
  \end{align*}
  Similarly,
  \begin{align*}
    \frac{\overline{DE}}{\overline{DA}} \geq \frac{\overline{CE}}{\overline{AE} + \overline{AC}} \text{  and  } \frac{\overline{FA}}{\overline{FC}} \geq \frac{\overline{AE}}{\overline{AC} + \overline{CE}}
  \end{align*}
  However, it is well known that
  \begin{align*}
  \frac{x}{y + z} + \frac{y}{z + x} + \frac{z}{x + y} \geq \frac{3}{2}
  \end{align*}
  Hence,
  \begin{align*}
    \frac {BC}{BE} + \frac {DE}{DA} + \frac {FA}{FC} \geq \frac {3}{2}
  \end{align*}
\end{proof}

\subsection{Exercises}
\begin{exercise}[RMO 2024]
  Let $ABCD$ be a cyclic quadrilateral such that $AB \parallel CD$. Let $O$ be the circumcenter of $ABCD$ and $L$ be the point on $AD$ such that $OL$ is perpendicular to $AD$. Prove that
\[ OB\cdot(AB+CD) = OL\cdot(AC + BD).\]
\end{exercise}

\begin{exercise}[IMO 1995]
  Let $ ABCDEF$ be a convex hexagon with $ AB = BC = CD$ and $ DE = EF = FA$, such that $ \angle BCD = \angle EFA = \frac {\pi}{3}$. Suppose $ G$ and $ H$ are points in the interior of the hexagon such that $ \angle AGB = \angle DHE = \frac {2\pi}{3}$. Prove that $ AG + GB + GH + DH + HE \geq CF$.
\end{exercise}

\section{Feuerbach's Theorem}
\vocab{Feuerbach's Theorem} is an extremely beautiful and celebrated result in Euclidean Geometry. Here is the statement for the theorem.
\begin{theorem}[Feuerbach's Theorem]
    The \vocab{Nine-Point Circle} of a triangle is \emph{internally tangent} to the \vocab{Incircle} and \emph{externally tangent} to the three \vocab{Excircles}. The points of tangencies are the interior and three exterior \vocab{Feuerbach Points}. 
\end{theorem}
It's difficult to prove this theorem with synthetic geometry, however we can easily prove this using Casey's Theorem.
\begin{figure}[h]
  \centering
  \begin{asy}
      import geometry; 
      size(10cm); defaultpen(fontsize(10pt));

      pair A, B, C, I, O, I_A, I_B, I_C, M_A, M_B, M_C;
      A = dir(110); B = dir(185); C = dir(355);

      O = origin; I = incenter(A, B, C);
      I_A = excenter(B, C, A); I_B = excenter(C, A, B); I_C = excenter(A, B, C);
      M_A = (B + C) / 2; M_B = (C + A) / 2; M_C = (A + B) / 2;
      
      dot("$A$", A, dir(A)); dot("$B$", B, dir(210)); dot("$C$", C, dir(335));
      
      draw(A--B--C--cycle);
      draw(incircle(A, B, C), red);
      draw(circumcircle(M_A, M_B, M_C));

      pair X_1, X_2, X_3, X_4, X_5, X_6;
      X_1 = foot(I_A, A, B); X_2 = foot(I_A, A, C);
      X_3 = foot(I_B, B, C); X_4 = foot(I_B, A, B);
      X_5 = foot(I_C, A, C); X_6 = foot(I_C, B, C);
      
      pair D = foot(I, B, C); pair Dp = reflect(A, I) * D;
      pair FF[] = intersectionpoints(line(M_A, Dp), circumcircle(M_A, M_B, M_C));
      
      dot("$Fe$", FF[1], dir(45));
      pair E = foot(I, A, C); pair F = foot(I, A, B);
      dot("$D$", D, 2*dir(225));
      dot("$E$", E, dir(45));
      dot("$F$", F, 2*dir(135));
      dot("$I$", I, dir(45));
  \end{asy}
\end{figure}
\begin{proof}
  Apply converse of Casey's Theorem on the incircle $\odot(I)$ and the point circles $\odot(M_A)$, $\odot(M_B)$ and $\odot(M_C)$,
  \begin{align*}
    \begin{cases} 
      \delta_{DE} \cdot \delta_{FI} = \tfrac{c}{2} \left| \tfrac{b - a}{2} \right| \\ 
      \delta_{EF} \cdot \delta_{DI} = \tfrac{a}{2} \left| \tfrac{b - c}{2} \right| \\ 
      \delta_{DF} \cdot \delta_{EI} = \tfrac{b}{2} \left| \tfrac{c - a}{2} \right| \\ 
    \end{cases}
    \end{align*}
    which is easy to show that satisfies
    \begin{align*}
      \pm \delta_{DE} \cdot \delta_{FI} \pm \delta_{EF} \cdot \delta_{DI} \pm \delta_{DF} \cdot \delta_{EI} = 0
    \end{align*}
    Therefore, $\odot(DEF)$ which is the nine point circle is tangent to $\odot(I)$ which is the incircle. Similarly, we can show that the incircle is tangent to all the three excircles.
\end{proof}

\section{Sawayama's Theorem}
We can even produce a short proof of \vocab{Sawayama's Theorem} using Casey's Theorem.

\begin{figure}[h]
  \centering
  \begin{asy}
    import geometry;
    size(6cm); defaultpen(fontsize(10pt));

    pair O, O1;
    real R, r;
    R = 4; r = 2.2;

    O = origin; O1 = O + (R - r) * dir(45);
    draw(circle(O, R)); draw(circle(O1, r));

    pair K = O1 + r * dir(270); dot("$K$", K, dir(305));
    pair P1 = K - rotate(90) * (O1 - K);
    pair P2 = K + rotate(90) * (O1 - K);

    pair[] AA = intersectionpoints(line(P1, P2), circle(O, R));
    dot("$B$", AA[0], dir(225)); dot("$C$", AA[1], dir(315));
    
    pair A = R * dir(98); dot("$A$", A, dir(100));
    pair T = O1 + r * dir(45); dot("$T$", T, dir(45));
    pair M = O + R * dir(270); 

    pair I = incenter(A, AA[0], AA[1]);

    pair[] LL = intersectionpoints(line(K, I), circle(O1, r));
    dot("$L$", LL[0], dir(150));

    pair D = extension(A, LL[0], AA[0], AA[1]);
    dot("$D$", D, dir(225)); 
    
    draw(A--AA[0]--AA[1]--cycle); draw(A--D);

    dot("$I$", I, dir(45));

    pair E = extension(A, I, AA[0], AA[1]);
    dot("$E$", E, dir(225));

    draw(K--LL[0], heavygray+dashed);
    draw(A--E);
  \end{asy}
\end{figure}
\begin{proof}
  We want to show that the points $L$, $I$ and $K$ are collinear so applying menelaus' theorem on $\triangle ADE$, we get
  \begin{align*}
    \frac{AL}{LD} \cdot \frac{DK}{KE} \cdot \frac{EI}{IA} = 1
  \end{align*}
  However $\overline{DL}$ $=$ $\overline{DK}$, so we just want to show that 
  \begin{align*}
    \frac{AL}{KE} \cdot \frac{EI}{IA} = 1
  \end{align*}
  Applying Casey's Theorem on $\odot(TLK)$, and point circles $\odot(A)$, $\odot(B)$ and $\odot(C)$ we get 
  \begin{align*}
    \overline{AL} \cdot \overline{BC} + \overline{CK} \cdot \overline{AB} = \overline{BK} \cdot \overline{AC}
  \end{align*}
  this implies that
  \begin{align*}
    \frac{AL}{KE} = \frac{b + c}{a}
  \end{align*}
  but using angle bisector theorem, we already know that $\tfrac{AI}{IE} = \tfrac{b + c}{a}$ $\implies$ $\tfrac{AL}{KE} \cdot \tfrac{EI}{IA} = 1$, thus proving the result.
\end{proof}

\section{Examples}
\begin{problem}[IMO Shortlist 2017]
  In triangle $ABC$, let $\omega$ be the excircle opposite to $A$. Let $D, E$ and $F$ be the points where $\omega$ is tangent to $BC, CA$, and $AB$, respectively. The circle $AEF$ intersects line $BC$ at $P$ and $Q$. Let $M$ be the midpoint of $AD$. Prove that the circle $MPQ$ is tangent to $\omega$.
\end{problem}
\begin{figure}[h]
  \centering
  \begin{asy}
    import geometry;
    size(9cm); defaultpen(fontsize(11pt));

    pair A, B, C, I, Ia, E, F, D, M, O;

    A = dir(120);
    B = dir(220);
    C = dir(320);

    I = incenter(A, B, C);
    Ia = excenter(B, C, A);
    D = foot(Ia, B, C);
    E = foot(Ia, A, B);
    F = foot(Ia, A, C);
    M = (A + D) / 2;
    O = (A + Ia) / 2;

    pair[] PP = intersectionpoints(line(B, C), circumcircle(A, E, F));
    pair P = PP[0]; pair Q = PP[1];

    dot("$A$", A, dir(130));
    dot("$B$", B, dir(225));
    dot("$C$", C, dir(315));
    dot("$D$", D, dir(225));
    dot("$E$", E, dir(215));
    dot("$F$", F, dir(315));
    dot("$I_A$", Ia, dir(315));
    dot("$M$", M, dir(45));
    dot("$O$", O, dir(225));
    dot("$P$", P, dir(225));
    dot("$Q$", Q, dir(315));

    draw(A--B--C--cycle);
    draw(B--E); draw(C--F);
    draw(A--D); draw(A--Ia);
    draw(Ia--E); draw(Ia--F);
    draw(Ia--D); draw(O--M);
    draw(M--P); draw(M--Q);
    draw(P--B); draw(C--Q);
    draw(A--P); draw(P--Ia);
    draw(A--Q); draw(Q--Ia);

    draw(circumcircle(A, E, F));
    draw(arc(circumcircle(M, P, Q), 0, 180), heavygray+dashed);
    markrightangle(Ia, E, A, 7);
    markrightangle(A, F, Ia, 7);
    markrightangle(Ia, D, C, 7);
    markrightangle(Ia, P, A, 7);
    markrightangle(A, Q, Ia, 7);
  \end{asy}
\end{figure}
\begin{proof}
  Let $O$ be the center of $\odot(AEF)$. Since $\odot(AEF)$ passes through the $A$-excenter and $\angle AEI_A$ $=$ $90^{\circ}$ $\implies$ $O$ is the midpoint of $\overline{AI_A}$. Since, $\overline{I_AD}$ $\perp$ $\overline{BC}$ and $\overline{OM}$ $\parallel$ $\overline{I_AD}$ by midpoint theorem $\implies$ $\overline{OM}$ $\perp$ $\overline{BC}$. Since $O$ is the center of $\odot(AEF)$ $\implies$ $\overline{MP}$ $=$ $\overline{MQ}$. Since
  \begin{align*}
    \overline{PQ}^2 \cdot (\overline{MI_A}^2 - \overline{DI_A}^2) &= \left(\overline{MP} \cdot \overline{QD} + \overline{MQ} \cdot \overline{PD} \right)^2 \\ 
    \Longleftrightarrow \overline{PQ}^2 \cdot (\overline{MI_A}^2 - \overline{DI_A}^2) &= \overline{MP}^2 \cdot \overline{PQ}^2 \\ 
    \Longleftrightarrow \overline{MI_A}^2 - \overline{DI_A}^2 &= \overline{MP}^2 \\ 
    \Longleftrightarrow \overline{MI_A}^2 - \overline{MP}^2 &= \overline{DI_A}^2 \\ 
    \Longleftrightarrow \frac{1}{4} \left( 2\overline{AI}^2 + 2\overline{DI_A}^2 - \overline{AD}^2 \right) - \frac{1}{4} \left( 2\overline{PD}^2 + 2\overline{AP}^2 - \overline{AD}^2 \right) &= \overline{DI_A}^2 \\ 
    \Longleftrightarrow \frac{1}{2} \left( \overline{AI_A}^2 - \overline{PD}^2 - \overline{AP}^2 + \overline{DI_A}^2 \right) &= \overline{DI_A}^2
  \end{align*}
  which is true due to pythagoras' theorem on $\triangle API_A$ and $\triangle PDI_A$. Therefore using the converse of Casey's Theorem on $\odot(DEF)$ and point circles $\odot(M)$, $\odot(P)$ and $\odot(Q)$, we have shown that $\odot(MPQ)$ is tangent to $\odot(DEF)$. 
\end{proof}


\section{Practice Problems}
\begin{exercise}
  Let $ABCD$ be a cyclic quadrilateral. Prove that,
$$\frac{AC}{BD}= \frac{AB\cdot AD + CB\cdot CD}{BA\cdot BC + DA\cdot DC}$$
\end{exercise}

\begin{exercise}[USA 1997]
  Let $Q$ be a quadrilateral whose side lengths are $a, b, c, d$ in that order. Show that the area of $Q$ does not exceed $\frac{ac + bd}{2}.$
\end{exercise}

\begin{exercise}[APMO 2014]
  Circles $\omega$ and $\Omega$ meet at points $A$ and $B$. Let $M$ be the midpoint of the arc $AB$ of circle $\omega$ ($M$ lies inside $\Omega$). A chord $MP$ of circle $\omega$ intersects $\Omega$ at $Q$ ($Q$ lies inside $\omega$). Let $\ell_P$ be the tangent line to $\omega$ at $P$, and let $\ell_Q$ be the tangent line to $\Omega$ at $Q$. Prove that the circumcircle of the triangle formed by the lines $\ell_P$, $\ell_Q$ and $AB$ is tangent to $\Omega$.
\end{exercise}

\begin{exercise}[IMO 1997]
  It is known that $ \angle BAC$ is the smallest angle in the triangle $ ABC$. The points $ B$ and $ C$ divide the circumcircle of the triangle into two arcs. Let $ U$ be an interior point of the arc between $ B$ and $ C$ which does not contain $ A$. The perpendicular bisectors of $ AB$ and $ AC$ meet the line $ AU$ at $ V$ and $ W$, respectively. The lines $ BV$ and $ CW$ meet at $ T$. Show that $ AU = TB + TC$.
\end{exercise}

\begin{exercise}
  Let $ABC$ be a triangle with centroid $G$, incenter $I$, incircle $\omega$, and nine-point circle $\Gamma$. Let the line $IG$ meet $BC$ at $P$ and let the common tangent $\omega$ and $\Gamma$ meet $BC$ at $Q$. Prove that the midpoint of $BC$ is also the midpoint of $PQ$ .
\end{exercise}

\begin{exercise}
  Let $D, E, F$ be points on sides $BC, CA, AB$ of triangle $ABC$ respectively such that lines $AD, BE, CF$ concur. Let $\Omega$ be the circumcircle of triangle $ABC$ and let $\omega_{A}$ be the circle internally tangent to $\Omega$ and tangent to $BC$ at $D$. Define circle $\omega_{B}$ and $\omega_{C}$ similarly. Show that there exists a circle tangent to circles $\omega_{A}, \omega_{B}, \omega_{C}$ that is also tangent to the incircle of triangle of $ABC$.
\end{exercise}

\begin{exercise}[IMO 2001]
  Let $a > b > c > d$ be positive integers and suppose that\[ ac + bd = (b+d+a-c)(b+d-a+c).  \]Prove that $ab + cd$ is not prime.
\end{exercise}



\end{document}