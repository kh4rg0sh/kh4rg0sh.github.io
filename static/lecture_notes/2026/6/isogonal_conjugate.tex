\documentclass[11pt]{scrartcl}
\let\captionof\undefined
\usepackage[sexy,von]{evan}
\usepackage{wrapfig}
% \renewcommand{\vonenvname}{example}
\lstset{basicstyle=\small\ttfamily,
  numbers=left,
  numbersep=5pt,
  numberstyle=\tiny,
  keywordstyle=\bfseries,
  showstringspaces=false,
  tabsize=4,
  frame=single,
  keywordstyle=\bfseries\color{blue},
  commentstyle=\color{green!70!black},
  identifierstyle=\color{green!20!black},
  stringstyle=\color{orange},
  breaklines=true,
  breakatwhitespace=true,
  frame=none
}

\usepackage{xcolor}
\setkomafont{captionlabel}{\bfseries\color{red}}
\renewcommand*{\figurename}{Fig}

\usepackage{answers}
\usepackage{cancel}
\usepackage{asymptote}
\usepackage{hyperref}

\begin{document}
\title{The Isogonal Conjugate of Feuerbach Point}
\date{\today}
\maketitle

\begin{abstract}
    \centering
    In this article, we shall explore some properties of the \vocab{Isogonal Conjugate} of the \vocab{Feuerbach Point}. Check out \href{https://kh4rg0sh.github.io/static/lecture_notes/all_final_notes/steiner_line_feuerbach_point.pdf}{this} article for some pre-requisites.
\end{abstract}

\section{Characterization 1}
\begin{problem}[Well Known]
    Given a $\triangle ABC$, its incenter $I$ and circumcenter $O$. Suppose $\triangle DEF$ is the contact triangle of $\triangle ABC$. Reflect $D$ over $\overline{OI}$ to $D'$, and similarly define $E'$ and $F'$. Then $\overline{AD'}$, $\overline{BE'}$ and $\overline{CF'}$ concur at the isogonal conjugate of the feuerbach point. 
\end{problem}
\begin{figure}[h]
    \centering
    \begin{asy}
        import geometry;
        size(11cm); defaultpen(fontsize(11pt));

        pair A, B, C, O;
        A = dir(150); B = dir(215); C = dir(325);
        dot("$A$", A, dir(A)); dot("$B$", B, dir(210)); dot("$C$", C, dir(335));
    
        draw(A--B--C--cycle);
        draw(circumcircle(A, B, C));

        pair D, E, F, I;
        I = incenter(A, B, C); D = foot(I, B, C); E = foot(I, A, C); F = foot(I, A, B);
        pair Dp, Ep, Fp;
        Dp = reflect(A, I) * D; 
        Ep = reflect(B, I) * E;
        Fp = reflect(C, I) * F;

        pair Ma, Mb, Mc;
        Ma = (B + C) / 2; Mb= (C + A) / 2; Mc = (A + B) / 2;
        pair Mar = reflect(A, I) * Ma;

        draw(incircle(A, B, C));
        dot("$D$", D, dir(240)); dot("$E$", E, dir(45)); dot("$F$", F, dir(165));

        dot("$I$", I, dir(80));
        dot("$O$", O, dir(30));
        draw(D--E--F--cycle);

        pair[] TT = intersectionpoints(line(Ma, Dp), circumcircle(D, E, F));
        pair[] TTr = intersectionpoints(line(Mar, D), circumcircle(D, E, F));

        dot("$Fe$", TT[1], dir(80));
        dot("$D'$", TTr[1], dir(130));

        draw(D--TTr[1], gray+dashed);
        draw(I--O); draw(D--TT[1], lightgray);

        pair Ft = foot(O, D, TTr[1]);
        draw(I--Ft, gray+dotted);
        markrightangle(D, Ft, I, 7);

        pair Er, Fr;
        Er = reflect(O, I) * E;
        Fr = reflect(O, I) * F;

        dot("$E'$", Er, dir(320));
        dot("$F'$", Fr, dir(185));

        pair Fer = extension(B, Er, C, Fr);
        dot("$Fe'$", Fer, dir(270));

        draw(A--Fer, red); draw(B--Er, red); draw(C--Fr, red);
    \end{asy}
\end{figure}
\begin{proof}
It's well known that the isogonal line of $\overline{DFe}$ with respect to $\angle FDE$ is perpendicular to $\overline{OI}$. Therefore, $D'FEFe$ is an isosceles trapezium $\implies$ $\overline{AD'}$ and $\overline{AFe}$ are symmetric about $\overline{AI}$. Hence, $\overline{AD'}$, $\overline{BE'}$ and $\overline{CF'}$ concur at the isogonal conjugate of the feuerbach point.
\end{proof}

\section{Characterization 2}
\begin{problem}[INMO 2019 Mock P5 by Shantanu Nene]
Given $\triangle ABC$ with circumcenter $O$. Suppose $\triangle M_AM_BM_C$ is the medial triangle of $\triangle ABC$. Let $\omega_A$, $\omega_B$ and $\omega_C$ be the $A$-, $B$- and $C$-mixtilinear excircles of $\triangle AM_BM_C$, $\triangle BM_CM_A$ and $\triangle CM_AM_B$. Show that
\begin{enumerate}[itemsep=0.01em]
    \item $\omega_A$, $\omega_B$ and $\omega_C$ are tangent to $\odot(BOC)$, $\odot(COA)$ and $\odot(AOB)$.
    \item Suppose $X_A$, $X_B$ and $X_C$ are the tangency points, then $\overline{AX_A}$, $\overline{BX_B}$ and $\overline{CX_C}$ are concurrent at the isogonal conjugate of the feuerbach point of $\triangle ABC$.
\end{enumerate}
\end{problem}
\begin{proof}
Performing a $\sqrt{\tfrac{bc}{2}}$ inversion around point $A$ followed by a reflection along the angle bisector of $\angle BAC$, maps $\omega_A$ to the incircle and $\odot(BOC)$ to the nine-point circle. Therefore, the point of tangency $X_A$ is the image of the feuerbach point under inversion $\implies$ $\overline{AX_A}$ is isogonal to $\overline{AFe}$ in $\angle BAC$. Therefore, $\overline{AX_A}$, $\overline{BX_B}$ and $\overline{CX_C}$ concur at the isogonal conjugate of the feuerbach point.
\end{proof}
\end{document}