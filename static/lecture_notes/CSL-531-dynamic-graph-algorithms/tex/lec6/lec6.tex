\section{[Lecture] Jan 28, 2026}

In this lecture we discuss about the incremental versions of the connectivity problem. By definition, the connectivity problem is defined for undirected graphs.
\begin{definition}
    In a graph $G = (V, E)$, we say that $u$ is \vocab{connected} to $v$ if there exists a path of edges from $u$ to $v$ for $u, v \in V$.
\end{definition}
\subsection{Incremental Single Source Connectivity}
\begin{problem}[Incremental Single Source Connectivity]
    Given an undirected graph $G = (V, E)$ and a source vertex $s$, maintain the set of vertices connected to $s$ under edge additions.
\end{problem}

Since an undirected graph can be modelled as a directed graph, therefore algorithms that can compute the \highlight{Incremental Single Source Reachability} can also be used here. This leads to 
\begin{quote}
    \centering
    Worst case time = $\mathcal{O}(m)$, and Amortized time = $\mathcal{O}(1)$
\end{quote}

\subsection{Incremental All Pairs Connectivity}
\begin{problem}[Incremental All Pairs Connectivity]
    Given an undirected graph $G = (V, E)$, maintain the sets of vertices connected to each $x$ $\in$ $V$ under edge additions.
\end{problem}
We maintain a two-dimensional array $R[n][n]$, where
\begin{align*}
    R[u][v] = \begin{cases}
        1, \qquad \text{if v is connected to u} \\ 
        0, \qquad \text{otherwise}
    \end{cases}
\end{align*}
Following a similar algorithm to \highlight{Incremental Single Source Reachability} and updating the data structure for all vertices $u$ $\in$ $V$, we end up with 
\begin{quote}
    \centering
    Worst case time = $\mathcal{O}(mn)$, and Amortized time = $\mathcal{O}(n)$
\end{quote}

\subsection{Disjoint Set Union}
Another data structure that could be used to solve the \highlight{Incremental All Pairs Connectivity} problem is \vocab{Disjoint Set Union}. It is a data structure that supports 
\begin{enumerate}
    \item \textsc{Find}: identify the component an element belongs to.
    \item \textsc{Union}: merge two components.
\end{enumerate}

With this data structure, we can propose an algorithm to solve the \highlight{Incremental All Pairs Connectivity} problem.
\begin{algorithm}
    \begin{algorithmic}[1]
        \Function{Update}{$x, y$}
            \State Add $e(x, y)$ to $E$
            \State \Call{Merge}{$x, y$}
        \EndFunction

        \Function{Query}{$x, y$}
            \State \Return \Call{Find}{$x$} $==$ \Call{Find}{$y$}
        \EndFunction
    \end{algorithmic}
\end{algorithm}

So now we focus on improving the time complexity of this data structure.

\subsubsection{DSU using Lists and Arrays}
We use two data structures here.
\begin{enumerate}
    \item List $L[i]$ that stores all elements in a component $c_i$.
    \item Array $A[i]$ that stores the component index for each $i$.
\end{enumerate}


