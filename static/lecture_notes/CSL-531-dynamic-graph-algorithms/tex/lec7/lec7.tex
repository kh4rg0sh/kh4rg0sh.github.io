\section{[Lecture] Jan 28, 2026}
We discuss \highlight{\emph{Tarjan's}} analysis of Disjoint Set Union in this lecture. First we shall define the \highlight{Ackermann Function} and then the \highlight{Inverse Ackermann Function}.
\subsection{Ackermann Function}
\begin{definition}
    The \vocab{Ackermann Function} is defined for $k \geq 0$ and $j \geq 1$ as 
    \begin{align*}
        A_k(j) = \begin{cases}
            j + 1, \qquad &\text{if } k = 0 \\ 
            A_{k - 1}^{j + 1} (j), \qquad &\text{if } k \geq 1
        \end{cases}
    \end{align*}
    where, $A_k^{0}(j) = j$ and
    \begin{align*}
        A_k^{i} (j) = A_k \left( A_k^{i - 1} (j)\right) \qquad \forall \, i \geq 1
    \end{align*}
\end{definition}

\subsubsection{Some Properties}
\begin{proposition}
    Show that, for $j \geq 1$
    \begin{align*}
        A_1(j) = 2j + 1
    \end{align*}
\end{proposition}
\begin{proof}
    \begin{align*}
        A_1(j) &= A_0^{j + 1}(j) = A_0^{j} \left( A_0 (j)\right) \\ 
        &= A_0^{j}(j + 1) = A_0^{j - 1} \left( A_0(j + 1)\right) \\ 
        &= A_0^{j - 1}(j + 2) = A_0^{j - 2} \left( A_0(j + 2)\right) \\
        \vdots \\  
        &= A_0^{0} \left( 2j + 1 \right) = 2j + 1
    \end{align*}
\end{proof}
\begin{proposition}
    Show that, for $j \geq 1$
    \begin{align*}
        A_2(j) = 2^{j + 1} \left( j + 1 \right) - 1
    \end{align*}
\end{proposition}
\begin{proof}
    \begin{align*}
        A_2(j) &= A_1^{j + 1} (j) = A_1^{j} \left( A_1(j)  \right) \\ 
            &= A_1^{j} \left( 2j + 1 \right) = A_1^{j - 1} \left( A_1(2j + 1)\right) \\ 
            &= A_1^{j - 1} \left( 4j + 3 \right) = A_1^{j - 2} \left( A_1(4j + 3)\right) \\ 
            \vdots \\ 
            &= A_1^{0} \left( 2^{j + 1} \left(j + 1\right) - 1\right) = 2^{j + 1} \left(j + 1\right) - 1
    \end{align*}
\end{proof}

\subsection{Inverse Ackermann Functiom}
\begin{definition}
    The \vocab{Inverse Ackermann Function} is defined as
    \begin{align*}
        \alpha (n) = \min \left\{ k \,:\, A_k(1) \geq n \right\}
    \end{align*}
\end{definition}
\begin{remark}
    The \highlight{Inverse Ackermann Function} grows extremely slowly. 
    \begin{enumerate}[itemsep=0.01em]
        \item $\alpha(n) = 0$ for $n \in [0, 2]$
        \item $\alpha(n) = 1$ for $n = 3$
        \item $\alpha(n) = 2$ for $n \in [4, 7]$
        \item $\alpha(n) = 3$ for $n \in [8, 2047]$
        \item $\alpha(n) = 4$ for $n \in [2048, 2^{2048}]$ \\
        $\vdots$
    \end{enumerate}
\end{remark}

\subsection{Potential Function for Amortized Analysis}
For a graph $G = (V, E)$, we define the potential function $\phi$ as 
\begin{align*}
    \phi = \sum_{v \in V} \psi(v)
\end{align*}
where,
\begin{align*}
    \psi(v) = \begin{cases}
        \alpha(n) \times v.\text{rank}, &\qquad \text{if } v \text{ is root or } v.\text{rank} = 0 \\ 
        \left( \alpha(n) - \operatorname{level}(v)\right) \times v.\text{rank} - \operatorname{iter}(v), &\qquad \text{otherwise} 
    \end{cases}
\end{align*}
and
\begin{align*}
\operatorname{level}(x) &= \max\left\{ k \,:\, \operatorname{par}(x).\text{rank} \geq A_k\left(x.\text{rank}\right) \right\} \\ 
\operatorname{iter}(x) &= \max\left\{ i \,:\, \operatorname{par}(x).\text{rank} \geq A_{\operatorname{level}(x)}^{i} \left( x.\text{rank}\right)\right\}
\end{align*}

\begin{proposition}
    The following hold true for these functions.    
    \begin{enumerate}
        \item $\operatorname{level}(x) \in [\,0, \alpha(n)\,]$
        \item $\operatorname{iter}(x) \in [\,1, x.\text{rank}\,]$
        \item $\psi(x) \in [\,0, \alpha(n)\times x.\text{rank}\,]$
    \end{enumerate}
\end{proposition}
\begin{proof} Consider a fixed node $x$ in the tree.
\begin{enumerate}
    \item Since, $$\operatorname{par}(x).\text{rank} \le n \implies \operatorname{level}(x) \le \alpha(n)$$
    \item Since, 
    \begin{align*}
        A_{\operatorname{level}(x)}^{i} \left( x.\text{rank}\right) &\le \operatorname{par}(x).\text{rank} \\ 
            &< A_{\operatorname{level}(x) + 1}\left( x.\text{rank} \right) = A_{\operatorname{level}(x)}^{x.\text{rank} + 1} \left( x.\text{rank} \right) \\ 
            &\implies i \le x.\text{rank}
    \end{align*}
\end{enumerate}
\end{proof}

