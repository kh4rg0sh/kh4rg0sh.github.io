\documentclass[11pt]{scrartcl}
\let\captionof\undefined
\usepackage[sexy,von]{evan}
\usepackage{wrapfig}
% \renewcommand{\vonenvname}{example}
\lstset{basicstyle=\small\ttfamily,
  numbers=left,
  numbersep=5pt,
  numberstyle=\tiny,
  keywordstyle=\bfseries,
  showstringspaces=false,
  tabsize=4,
  frame=single,
  keywordstyle=\bfseries\color{blue},
  commentstyle=\color{green!70!black},
  identifierstyle=\color{green!20!black},
  stringstyle=\color{orange},
  breaklines=true,
  breakatwhitespace=true,
  frame=none
}

\usepackage{xcolor}
\setkomafont{captionlabel}{\bfseries\color{red}}
\renewcommand*{\figurename}{Fig}

\usepackage{answers}
\usepackage{cancel}
\usepackage{asymptote}
\usepackage{hyperref}

\begin{document}
\title{Problem Set 1}
\date{\today}
\maketitle

\begin{abstract}
    \centering
    Some practice problems based on everything I've taught so far.
\end{abstract}

\section{Instructions}
\begin{enumerate}[itemsep=0.01em]
    \item Try out each problem on paper for a sufficient amount of time before using GeoGebra.
    \item Ideally you should continue working on the problem until you have exhausted all ideas and are unsure how to proceed further.
    \item You may then use GeoGebra to identify key observations that may help you in solving the problem.
    \item Feel free to ask queries or hints if you have made substantial progress and feel stuck. I will release the hints for each problem soon.
\end{enumerate}

\section{Practice Problems}
\begin{exercise}
    Suppose $H$ and $O$ are the orthocenter and circumcenter of $\triangle ABC$. Let $O_A$, $O_B$ and $O_C$ be the reflections of $O$ over the sides $\overline{BC}$, $\overline{CA}$ and $\overline{AB}$. Show that
    \begin{enumerate}[itemsep=0.01em]
        \item $O$ is the orthocenter and $H$ is the circumcenter of $\triangle O_AO_BO_C$.
        \item $O_A$, $O_B$ and $O_C$ are the circumcenters of $\triangle BHC$, $\triangle CHA$ and $\triangle AHB$.
    \end{enumerate}
\end{exercise}

\begin{exercise}[IMO Shortlist 1989]
The vertex $ A$ of the acute triangle $ ABC$ is equidistant from the circumcenter $ O$ and the orthocenter $ H.$ Determine all possible values for the measure of angle $A.$
\end{exercise}

\begin{exercise}[RMO 2025]
Let $ABC$ be an acute-angled triangle with $AB<AC$ , orthocenter $H$ and circumcircle $\Omega$ . Let $M$ be the midpoint of the minor arc $BC$ of $\Omega$ . Suppose the $MH$ is equal to the radius of $\Omega$. Prove that $\angle BAC=60^\circ$
\end{exercise}

\begin{exercise}
    Let $\triangle ABC$ be an acute triangle such that $\angle A = 60^{\circ}$. Prove that $IH = IO$, where $I$, $H$ and $O$ are the incenter, orthocenter and circumcenter.
\end{exercise}

\begin{exercise}[APMO 2007]
    Let $ABC$ be an acute angled triangle with $\angle{BAC}=60^\circ$ and $AB > AC$. Let $I$ be the incenter, and $H$ the orthocenter of the triangle $ABC$ . Prove that $2\angle{AHI}= 3\angle{ABC}$.
\end{exercise}

\begin{exercise}
    Let triangle $ABC$ satisfy $2BC = AB+AC$ and have incenter $I$ and circumcircle $\omega$. Let $D$ be the intersection of $AI$ and $\omega$ (with $A, D$ distinct). Prove that $I$ is the midpoint of $AD$.
\end{exercise}

\begin{exercise}[IMO 2020]
    Consider the convex quadrilateral $ABCD$. The point $P$ is in the interior of $ABCD$. The following ratio equalities hold:
\[\angle PAD:\angle PBA:\angle DPA=1:2:3=\angle CBP:\angle BAP:\angle BPC\]Prove that the following three lines meet in a point: the internal bisectors of angles $\angle ADP$ and $\angle PCB$ and the perpendicular bisector of segment $AB$.
\end{exercise}

\begin{exercise}
    $\triangle ABC$ is inscribed in circle $\omega$. A circle with chord $BC$ intersects segments $AB$ and $AC$ again at $S$ and $R$, respectively. Segments $BR$ and $CS$ meet at $L$, and rays $LR$ and $LS$ intersect $\omega$ at $D$ and $E$, respectively. The internal angle bisector of $\angle BDE$ meets line $ER$ at $K$. Prove that if $BE = BR$, then
    \begin{enumerate}[itemsep=0.01em]
        \item $R$ is the incenter of $\triangle CDE$.
        \item $\angle ELK = \tfrac{1}{2} \angle BCD$.
    \end{enumerate}
\end{exercise}

\begin{exercise}[Japan 2017]
    Let $ABC$ be an acute-angled triangle with the circumcenter $O$. Let $D,E$ and $F$ be the feet of the altitudes from $A,B$ and $C$, respectively, and let $M$ be the midpoint of $BC$. $AD$ and $EF$ meet at $X$, $AO$ and $BC$ meet at $Y$, and let $Z$ be the midpoint of $XY$. Prove that $A,Z,M$ are collinear.
\end{exercise}

\begin{exercise}[IGO 2021 Advanced]
    $H$ is the orthocenter of the acute-angled triangle $ABC$. ($AB<AC$). Call the point where the perpendicular bisector of $BC$ meets the sides $AB$ and $AC$ as $P$ and $Q$, respectively. Consider $M$ and $N$ as the midpoints of segments $BC$ and $PQ$. Prove that lines $HM$ and $AN$ intersect on the circumcircle of $ABC$.
\end{exercise}

\begin{exercise}[IMO Shortlist 2024]
    Let $ABC$ be a triangle with $AB < AC < BC$. Let the incenter and incircle of triangle $ABC$ be $I$ and $\omega$, respectively. Let $X$ be the point on line $BC$ different from $C$ such that the line through $X$ parallel to $AC$ is tangent to $\omega$. Similarly, let $Y$ be the point on line $BC$ different from $B$ such that the line through $Y$ parallel to $AB$ is tangent to $\omega$. Let $AI$ intersect the circumcircle of triangle $ABC$ at $P \ne A$. Let $K$ and $L$ be the midpoints of $AC$ and $AB$, respectively. Prove that $\angle KIL + \angle YPX = 180^{\circ}$.
\end{exercise}

\end{document}