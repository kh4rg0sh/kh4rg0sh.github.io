\begin{document}


\section{Bisectors}

what does bisecting mean?

\subsection{Segment Bisector}
what does a segment bisector mean.

definition of a midpoint.
how to construct the midpoint:
proof for it.

\subsection{Perpendicular Bisector}
perpendicular bisector.
show the construction.
leave the proof as an exercise to the reader.
also ask the reader to prove the construction for perpendicular line.

any point on perpendicular bisector of a line segment forms an isoceles triangle.

\subsection{Angle Bisector}
angle bisector.
show the construction.
leave the proof as an exercise to the reader. 

perpendicular bisector and the angle bisector of an isosceles triangle are the same. 


\subsection{Tangents}
Tangents to a circle give rise to a lot of beautiful properties. This happens because they meet the circle in a very delicate way and thus, encode deep geometric properties. 

\begin{figure}[h]
    \centering
    \begin{asy}
        import geometry; unitsize(1cm);

        pair O = (0, 0);
        real r = 1.5;

        pair P = r * dir(35);
        pair v = rotate(90) * (P - O);
        pair u = v/length(v);

        draw(circle(O, r));
        draw(P-2*u -- P+2*u, dotted+linewidth(1));
        label("$\ell$", P+2*u, NE);

        draw(O--P, gray+1bp);
        dot(O); label("$O$", O, S);
        dot(P); label("$P$", P, NE);
    \end{asy}
    \caption{line $\ell$ tangent to circle $\mathcal{C}$ at point $P$}
\end{figure}

One of the interesting results about tangents is the following.
\begin{theorem}
Radius is perpendicular to the tangent at the point of contact.
\end{theorem}
\begin{figure}[h]
    \centering
    \begin{asy}
        import geometry; unitsize(1cm);

        pair O = (0, 0);
        real r = 1.5;

        pair P = r * dir(90);
        pair v = rotate(90) * (P - O);
        pair u = v/length(v);

        draw(circle(O, r));
        draw(P-2*u -- P+2*u, linewidth(1));
        label("$\ell$", P+2*u, NE);

        draw(O--P, gray+1bp);
        dot(O); label("$O$", O, S);
        dot(P); label("$P$", P, NE);
        markrightangle(O, P, P+u, 8);
    \end{asy}
    \caption{line $\ell$ is $\perp$ to $\overline{OP}$}
\end{figure}
\begin{proof}
    
\end{proof}

\begin{exercise}
Use the above theorem to show that, for any point not lying on a given line, the perpendicular from the point to the line is the path of shortest distance to the line.
\end{exercise}

\begin{exercise}
Use the above theorem to show that, for a point $P$ on circle $\mathcal{C}$ there exists a unique tangent $\ell$ to circle $\mathcal{C}$ that passes through point $P$.
\end{exercise}


\begin{proposition}[Number of common tangents]

\end{proposition}

\section{Circles: Revisited}
\subsection{Tangents from a Point}
how many tangents can you draw from a point to a circle:
0 if point interior.
1 if point lies on the circle. 
2 if the point lies outside the circle 

tangent drawn are equal in length.
pitots theorem.


\section{Quadrilaterals}
sum of all angles of quadrilateral
general quadrilateral inequality.

convex quadrilateral 

trapezium: one side parallel.
isoscseles trapezium: non parallel sides equal

parallelogram:
show diagonals of a parallelogram are segment bisectors.

rhombus:
show diagonals of a rhombus are perpendicular bisectors
diagonals of a rhombus are angle bisectors

kite:
show diagonals of kite are perpendicular 
and diagonal between equal sides is angle bisector 

rectangle:
diagonals are segment bisectors. 

square:
diagonals are perpendicular bisectors
diagonals are angle bisectors 

angle is 45


\section{Polygons}
sum of angles of polygon
exterior angle of equal sided polygon 
interior angle of equal sided polygon

general polygon inequality


\end{document}