\documentclass[11pt]{scrartcl}
\let\captionof\undefined
\usepackage[sexy,von]{evan}
\usepackage{wrapfig}
% \renewcommand{\vonenvname}{example}
\lstset{basicstyle=\small\ttfamily,
  numbers=left,
  numbersep=5pt,
  numberstyle=\tiny,
  keywordstyle=\bfseries,
  showstringspaces=false,
  tabsize=4,
  frame=single,
  keywordstyle=\bfseries\color{blue},
  commentstyle=\color{green!70!black},
  identifierstyle=\color{green!20!black},
  stringstyle=\color{orange},
  breaklines=true,
  breakatwhitespace=true,
  frame=none
}

\usepackage{xcolor}
\setkomafont{captionlabel}{\bfseries\color{red}}
\renewcommand*{\figurename}{Fig}

\usepackage{answers}
\usepackage{cancel}
\usepackage{asymptote}

\begin{document}
\title{IZhO 2026 P2}
\date{\today}
\maketitle

\begin{abstract}
    \centering
    https://artofproblemsolving.com/community/c6h3749391
\end{abstract}

\tableofcontents

\section{Problem}
\begin{problem*}[IZhO 2026 P2]
    We consider a positive integer $n$ for which there exist positive integers $a$ and $b$ for which $\lfloor  a\sqrt{10} \rfloor = n = \lfloor b\sqrt{11} \rfloor$. Prove there exists a positive integer $c$ for which $n=\lfloor c(11\sqrt{10}-10\sqrt{11} ) \rfloor$
\end{problem*}

\section{Solution (Using Rationalization)}
\begin{proof}
    We claim that $\boxed{c = a + b}$ works. Consider the equation $n = \lfloor a \sqrt{10} \rfloor$. This could be written as
    \begin{align*}
    n &\le a \sqrt{10} < n + 1 \\
    n &\le a \sqrt{10} \left( 11 - 10 \right) < n + 1 \\ 
    \frac{n}{\sqrt{11} + \sqrt{10}} &\le a \sqrt{10} \left( \sqrt{11} - \sqrt{10} \right) < \frac{n + 1}{\sqrt{11} + \sqrt{10}} \\ 
    \frac{n\sqrt{11}}{\sqrt{11} + \sqrt{10}} &\le 11 a \sqrt{10} - 10 a \sqrt{11} < \frac{(n + 1) \sqrt{11}}{\sqrt{11} + \sqrt{10}}
    \end{align*}Similarly, we can write $n = \lfloor b \sqrt{11} \rfloor$ as
    \begin{align*}
    n &\le b \sqrt{11} < n + 1 \\
    n &\le b \sqrt{11} \left( 11 - 10 \right) < n + 1 \\ 
    \frac{n}{\sqrt{11} + \sqrt{10}} &\le b \sqrt{11} \left( \sqrt{11} - \sqrt{10} \right) < \frac{n + 1}{\sqrt{11} + \sqrt{10}} \\ 
    \frac{n\sqrt{10}}{\sqrt{11} + \sqrt{10}} &\le 11 b \sqrt{10} - 10 b \sqrt{11} < \frac{(n + 1) \sqrt{10}}{\sqrt{11} + \sqrt{10}}
    \end{align*}Adding these two inequalities we get,
    \begin{align*}
    n &\le 11 a \sqrt{10} - 10 a \sqrt{11} + 11 b \sqrt{10} - 10 b \sqrt{11} < n + 1 \\ 
    n &\le (a + b) \left( 11 \sqrt{10} - 10 \sqrt{11} \right) < n + 1
    \end{align*}this implies that $\left\lfloor (a + b) \left( 11 \sqrt{10} - 10 \sqrt{11} \right) \right\rfloor = n$ as desired. 
\end{proof}

\section{Generalization 1}
\begin{theorem*}
    For any positive integers $n$, $a$ and $b$ and positive real numbers $x$ and $y$, if $n$ $=$ $\lfloor ax \rfloor$ $=$ $\lfloor by \rfloor$, then there exists a positive integer $c$ such that
    \begin{align*}
        n = \left\lfloor c \cdot \frac{xy}{x + y} \right\rfloor
    \end{align*}
\end{theorem*}
\begin{proof}
    For the equations
        \[
            n = \lfloor ax \rfloor \quad \text{and} \quad n = \lfloor by \rfloor,
        \]
    we have
        \[
            ax - 1 < n \le ax
            \quad \text{and} \quad
            by - 1 < n \le by.
        \]
    Dividing on both the sides by $x$ and $y$ gives us
        \[
            a - \frac{1}{x} < \frac{n}{x} \le a,
            \qquad
            b - \frac{1}{y} < \frac{n}{y} \le b.
        \]
    Adding these two inequalities, we obtain
        \[
            (a+b) - \left( \frac{1}{x} + \frac{1}{y} \right)
            < n\left( \frac{1}{x} + \frac{1}{y} \right)
            \le a+b.
        \]
    Multiplying throughout by $\dfrac{xy}{x+y}$ gives
        \[
        (a+b)\frac{xy}{x+y} - 1
        < n
        \le (a+b)\frac{xy}{x+y}.
        \]
    Therefore,
        \[
        n = \left\lfloor (a+b)\frac{xy}{x+y} \right\rfloor.
        \]
    For which, $c = a + b$ proves the result.
\end{proof}
\end{document}