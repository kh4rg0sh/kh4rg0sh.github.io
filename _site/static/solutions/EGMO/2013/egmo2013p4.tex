\documentclass[11pt]{scrartcl}
\let\captionof\undefined
\usepackage[sexy,von]{evan}
\usepackage{wrapfig}
% \renewcommand{\vonenvname}{example}
\lstset{basicstyle=\small\ttfamily,
  numbers=left,
  numbersep=5pt,
  numberstyle=\tiny,
  keywordstyle=\bfseries,
  showstringspaces=false,
  tabsize=4,
  frame=single,
  keywordstyle=\bfseries\color{blue},
  commentstyle=\color{green!70!black},
  identifierstyle=\color{green!20!black},
  stringstyle=\color{orange},
  breaklines=true,
  breakatwhitespace=true,
  frame=none
}

\usepackage{xcolor}
\setkomafont{captionlabel}{\bfseries\color{red}}
\renewcommand*{\figurename}{Fig}

\usepackage{answers}
\usepackage{cancel}
\usepackage{asymptote}

\begin{document}
\title{EGMO 2013 P4}
\date{\today}
\maketitle

\begin{abstract}
    \centering
    https://artofproblemsolving.com/community/c6h529188p3014762
\end{abstract}

\tableofcontents

\section{Problem}
\begin{problem*}
    Find all positive integers $a$ and $b$ for which there are three consecutive integers at which the polynomial\[ P(n) = \frac{n^5+a}{b} \]takes integer values.
\end{problem*}

\section{Solution 1 (Using Modular Arithmetics)}
\begin{proof}
Suppose the three numbers at which $P(n)$ takes integer values are $x - 1$, $x$ and $x + 1$. Then we must have
\begin{align*}
(x - 1)^5 \equiv x^5 \equiv (x + 1)^5 \equiv -a \pmod{b}
\end{align*}


\end{proof}
\end{document}