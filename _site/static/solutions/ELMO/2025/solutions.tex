\documentclass[11pt]{scrartcl}
\let\captionof\undefined
\usepackage[sexy,von]{evan}
\usepackage{wrapfig}
% \renewcommand{\vonenvname}{example}
\lstset{basicstyle=\small\ttfamily,
  numbers=left,
  numbersep=5pt,
  numberstyle=\tiny,
  keywordstyle=\bfseries,
  showstringspaces=false,
  tabsize=4,
  frame=single,
  keywordstyle=\bfseries\color{blue},
  commentstyle=\color{green!70!black},
  identifierstyle=\color{green!20!black},
  stringstyle=\color{orange},
  breaklines=true,
  breakatwhitespace=true,
  frame=none
}

\usepackage{xcolor}
\setkomafont{captionlabel}{\bfseries\color{red}}
\renewcommand*{\figurename}{Fig}

\usepackage{answers}
\usepackage{cancel}
\usepackage{asymptote}

\begin{document}
\title{ELMO 2025 Shortlist Solution Notes}
\date{\today}
\maketitle

\begin{abstract}
    \centering
    A compilation of solutions for the ELMO 2025 Shortlist.
    % https://artofproblemsolving.com/downloads/printable_post_collections/4369812
\end{abstract}

\tableofcontents

\newpage 
\section{Problems}
\subsection{Algebra}
\begin{problem*}[\texorpdfstring{\hyperref[sec:A1]{A1}}{A1}]
    Let $\mathbb{Z}_{>0}$ denote the set of positive integers. Find all functions $f \colon \mathbb{Z}_{>0} \to \mathbb{Z}_{>0}$ such that for all positive integers $m$ and $n$,
    \[f^m(n)+f(mn)=f(m)f(n).\]
    Note: $f^m(n)=\underbrace{f(f(\cdots f}_{m\text{ times}}(n)\cdots))$, that is, $f$ applied $m$ times to $n$.
\end{problem*}

\newpage
\subsection{Combinatorics}


\newpage
\subsection{Geometry}
\begin{problem*}[\texorpdfstring{\hyperref[sec:G1]{G1}}{G1}]
    Let $ABCD$ be a convex quadrilateral with $DA=AB=BC$. Let $M$ be the midpoint of $\overline{AB}$, and let $P$ be a point in the plane with $\angle PCA=\angle PDB=90^\circ$. A circle centered at $O$ is tangent to segments $DA$, $AB$, and $BC$. Prove that $M$, $O$, and $P$ are collinear.
\end{problem*}

\newpage
\subsection{Number Theory}

\newpage
\section{Solutions to Algebra}
\subsection{ELMO 2025 Shortlist A1}\label{sec:A1}
\begin{problem*}[A1]
    Let $\mathbb{Z}_{>0}$ denote the set of positive integers. Find all functions $f \colon \mathbb{Z}_{>0} \to \mathbb{Z}_{>0}$ such that for all positive integers $m$ and $n$,
    \[f^m(n)+f(mn)=f(m)f(n).\]
    Note: $f^m(n)=\underbrace{f(f(\cdots f}_{m\text{ times}}(n)\cdots))$, that is, $f$ applied $m$ times to $n$.
\end{problem*}

\subsubsection{Solution 1 (Using Injectivity \& Periodicity)}
Let $P(m, n)$ denote the assertion,
\[
    f^m(n) + f(mn) = f(m) f(n)
\]

\begin{claim}
    $f \equiv 2$ and $f \equiv m + 1$ are the only solutions.
\end{claim}
\begin{proof}
    It's easy to see that these satisfy the assertion $P(m, n)$. We will show that these are the only solutions that satisfy.

    From $P(1, 1)$, we have 
    \[
        f(1) + f(1) = f(1)f(1) \implies f(1) = 2
    \]

    From $P(m, 1)$, we have
    \[
        f^m(1) + f(m) = f(1) f(m) = 2f(m)
    \]

    Therefore, $f^m(1) = f(m)$. We will now deal with the following two cases independently.

    \begin{enumerate}[itemsep=0.01em]
        \item Suppose $f$ is injective. Then,
        \[
            f^{m+1}(1) = f(m + 1) \implies f(m + 1) = f(f(m)) \implies \boxed{f(m) = m + 1}
        \]
        
        \item Suppose $f$ is not injective. Then there exists $a$, $b$ such that $a \neq b$ and $f(a) = f(b)$. Comparing $P(a, 1)$ and $P(b, 1)$,
        \[ 
            f^a(1) = f(a) = f(b) = f^b (1)
        \]

        Since, $f(m) = f^m(1)$. It follows that $f$ only takes finite values and $f$ is periodic. Suppose $f$ achieves the largest value $L$ at $u$. From $P(2, u)$,
        \[
            f(f(u)) + f(2u) = f(2)f(u) \implies 2L \geq f(L) + f(2u) = f(2)L
        \]
        Therefore $f(2) \le 2$. If $f(2) = 1$, then we get that
        \[
            f(m) =
            \begin{cases}
            1, & \text{if $m$ is even},\\
            2, & \text{if $m$ is odd}.
            \end{cases}
        \]

        which is wrong since $P(2, 2)$ $\implies$ $f(f(2)) + f(4) = f(2)^2$, for which we get that its impossible. If $f(2) = 2$, then $f(m) = f^{m - 1}(2) = 2$ $\implies$ $\boxed{f(m) = 2}$
    \end{enumerate}
\end{proof}

\newpage
\section{Solutions to Combinatorics}

\newpage
\section{Solutions to Geometry}
\subsection{ELMO 2025 Shortlist G1}\label{sec:G1}
\begin{problem*}[G1]
    Let $ABCD$ be a convex quadrilateral with $DA=AB=BC$. Let $M$ be the midpoint of $\overline{AB}$, and let $P$ be a point in the plane with $\angle PCA=\angle PDB=90^\circ$. A circle centered at $O$ is tangent to segments $DA$, $AB$, and $BC$. Prove that $M$, $O$, and $P$ are collinear.
\end{problem*}

\subsubsection{Solution 1 (Using Centroid)}

\begin{figure}[h]
    \centering
    \begin{asy}
        import geometry;
        size(8cm); defaultpen(fontsize(10pt));

        pair A, B, C, O, H, BB, CC, M, N, P, D, E, F, X;
        A = dir(110); B = dir(220); C = dir(320);
        H = orthocenter(A, B, C);
        M = (B+C)/2; N = (C+A)/2; P=(A+B)/2;
        D = extension(A, H, B, C); E = extension(B, H, C, A); F = extension(C, H, A, B);
        CC = 2 * F - C; BB = 2 * E - B;

        pair v1 = rotate(90)*(BB - B);
        pair v2 = rotate(90)*(CC - C);
        X = extension(BB, BB + v1, CC, CC + v2);
        

        dot("$O$", A, dir(185)); dot("$A$", B, dir(B)); dot("$B$", C, dir(C));
        dot("$C$", BB,  dir(35)); dot("$D$", CC, dir(155)); dot("$P$", X, dir(95));
    
        draw(B--C--BB--X--CC--cycle); draw(BB--CC); draw(B--BB); draw(C--CC);
        markrightangle(X, BB, B, 7); markrightangle(C, CC, X, 7); dot("$M$", M, dir(315));

        draw(M--X, heavygray+dashed); draw(A--B, gray); draw(A--C, gray);
        draw(B--X, gray+dotted); draw(C--X, gray+dotted);

        dot("$K$", (B+X)/2, dir(165)); dot("$L$", (C+X)/2, dir(35));
        draw((B+X)/2--A, gray+dashed); draw((C+X)/2--A, gray+dashed);

        draw(circle(A, abs(A-D)), gray+dotted);
    \end{asy}
\end{figure}

\begin{proof}
    Let $K$ and $L$ be the midpoints of $\overline{AP}$ and $\overline{BP}$. Suppose $\omega$ is the circle tangent to sides $\overline{DA}$, $\overline{AB}$ and $\overline{BC}$.

    \begin{claim}\label{sec:Claim4.1}
        $L$ lies on line $OA$ and $K$ lies on line $OB$.
    \end{claim}
    \begin{proof}
        Since, $\overline{AB}$ and $\overline{BC}$ are equal in length and are tangents drawn from $B$ to $\omega$, we have that $\overline{OB}$ is the perpendicular bisector of $\overline{AC}$. Similarly, $\overline{OA}$ is the perpendicular bisector of $\overline{BD}$. Since $\angle ACP = 90^{\circ}$ $\implies$ $\overline{OB}$ $\parallel$ $\overline{CP}$ and similarly, $\overline{OA}$ $\parallel$ $\overline{DP}$. Therefore by midpoint theorem in $\triangle ACP$ and $\triangle BDP$, we get that $OA$ passes through $L$ and $OB$ passes through $K$.
    \end{proof}
    
    From \hyperref[sec:Claim4.1]{Claim 1}, we get that $O$ is the centroid of $\triangle PAB$ and therefore, $PO$ bisects $\overline{AB}$.    
\end{proof}

\newpage
\subsubsection{Solution 2 (Using Areas)}

\begin{figure}[h]
    \centering
    \begin{asy}
        import geometry;
        size(8cm); defaultpen(fontsize(10pt));

        pair A, B, C, O, H, BB, CC, M, N, P, D, E, F, X;
        A = dir(110); B = dir(220); C = dir(320);
        H = orthocenter(A, B, C);
        M = (B+C)/2; N = (C+A)/2; P=(A+B)/2;
        D = extension(A, H, B, C); E = extension(B, H, C, A); F = extension(C, H, A, B);
        CC = 2 * F - C; BB = 2 * E - B;

        pair v1 = rotate(90)*(BB - B);
        pair v2 = rotate(90)*(CC - C);
        X = extension(BB, BB + v1, CC, CC + v2);
        

        dot("$O$", A, dir(60)); dot("$A$", B, dir(B)); dot("$B$", C, dir(C));
        dot("$C$", BB,  dir(35)); dot("$D$", CC, dir(155)); dot("$P$", X, dir(95));
    
        draw(B--C--BB--X--CC--cycle); draw(BB--CC); draw(B--BB); draw(C--CC);
        markrightangle(X, BB, B, 7); markrightangle(C, CC, X, 7); dot("$M$", M, dir(315));

        draw(A--X);
        draw(M--A, heavygray+dashed); draw(A--B, gray); draw(A--C, gray);
        draw(B--X, gray); draw(C--X, gray);
        draw(A--CC, gray); draw(A--BB, gray);

        draw(circle(A, abs(A-D)), gray+dotted);
    \end{asy}
\end{figure}

\begin{proof}
    \begin{claim}
        Line $\overline{OA} \parallel \overline{DP}$ and $\overline{OB} \parallel \overline{CP}$.
    \end{claim}
    \begin{proof}
        Same as \hyperref[sec:Claim4.1]{Claim 1}
    \end{proof}
    
    Therefore,
    \begin{align*}
    \operatorname{Area} \left( \triangle AOP \right) &= \operatorname{Area} \left( \triangle AOD \right) = \operatorname{Area}\left( \triangle AOB \right) = \operatorname{Area} \left( \triangle BOC \right) \\ 
        &= \operatorname{Area} \left( \triangle BOP \right)
    \end{align*}
    
    Since, $\operatorname{Area} \left( \triangle AOP \right) = \operatorname{Area} \left( \triangle BOP \right)$ $\implies$ $PO$ bisects $\overline{AB}$    
\end{proof}

\newpage


\section{Solutions to Number Theory}

\end{document}