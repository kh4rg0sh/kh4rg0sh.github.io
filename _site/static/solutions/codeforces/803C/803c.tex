\documentclass[11pt]{scrartcl}
\let\captionof\undefined
\usepackage[sexy,von]{evan}
\usepackage{wrapfig}

\usepackage[utf8]{inputenc}
\usepackage[T1]{fontenc}
\usepackage{geometry}
\geometry{margin=1in}

\usepackage{xcolor}
\usepackage{hyperref}
\hypersetup{
    colorlinks=true,
    linkcolor=blue,
    urlcolor=cyan
}

\usepackage{listings}
\usepackage{amsmath, amssymb}
\usepackage{graphicx}
\usepackage{float}

% Code listing setup for C++ (or Python)
\lstset{
    basicstyle=\ttfamily\small,
    numbers=left,
    numberstyle=\tiny,
    stepnumber=1,
    numbersep=5pt,
    tabsize=4,
    breaklines=true,
    breakatwhitespace=true,
    showstringspaces=false,
    frame=single,
    keywordstyle=\bfseries\color{blue},
    commentstyle=\color{green!70!black},
    stringstyle=\color{orange}
}

% Optional: caption color
\usepackage[labelfont=bf,labelsep=colon]{caption}
\usepackage{tikz} % for diagrams if needed

\title{Codeforces 803C (1900)}
\author{Mmukul Khedekar}
\date{\today}

\begin{document}

\maketitle
\begin{abstract}
    \centering
    \url{https://codeforces.com/problemset/problem/803/C} \\ 
    \highlight{Accepted}: \url{https://codeforces.com/contest/803/submission/360178477}
\end{abstract}

\section{Solution}

\subsection{Explanation}

\begin{lemma}
    For an increasing sequence of length $k$ of positive integers that sum up to $n$, we must have 
    \begin{align*}
        \frac{k (k + 1)}{2} \le n
    \end{align*}
\end{lemma}
The first observation is that if $d$ is the gcd of the $k$ integers, then $k$ divides $n$. Hence we can iterate on the divisors of $n$ and check if there exists an increasing sequence of lenth $k$ that sums upto $n$ and has a gcd equal to that divisor. Using the lemma, we will have a solution if 
\begin{align*}
    \frac{k(k + 1)}{2} \le \frac{n}{d}
\end{align*}
and the maximum such $d$ will be the answer. Construction for a valid $d$ is simple. 
\begin{align*}
\left( d, 2d, 3d, \cdots , (k - 1)d, \left(n -  \tfrac{k(k - 1)}{2}\right) d\right)
\end{align*}

\subsection{Code}

\begin{lstlisting}[language=C++]
void solve() {
    ll n, k;
    std::cin >> n >> k;

    ll ans = -1;
    if (k > 1e6) {
        std::cout << -1 << '\n';
        return;
    }
    
    if ((k * (k + 1)) / 2 > n) {
        std::cout << -1 << '\n';
        return;
    }
    
    for (ll i = 1; i * i <= n; i++) {
        if (n % i == 0) {
            if ((k * (k + 1)) / 2 <= (n / i)) {
                ans = std::max(ans, i);
            }

            if (i * i != n) {
                if ((k * (k + 1)) / 2 <= i) {
                    ans = std::max(ans, n / i);
                }
            }
        }
    }

    if (ans == -1) {
        std::cout << -1 << '\n';
    } else {
        ll sum = 0;
        for (ll i = 1; i < k; i++) {
            sum += i * ans;
            std::cout << i * ans << ' ';
        }
        std::cout << n - sum << '\n';
    }
}    
\end{lstlisting}


\end{document}
