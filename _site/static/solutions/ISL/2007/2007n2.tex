\documentclass[11pt]{scrartcl}
\let\captionof\undefined
\usepackage[sexy,von]{evan}
\usepackage{wrapfig}
% \renewcommand{\vonenvname}{example}
\lstset{basicstyle=\small\ttfamily,
  numbers=left,
  numbersep=5pt,
  numberstyle=\tiny,
  keywordstyle=\bfseries,
  showstringspaces=false,
  tabsize=4,
  frame=single,
  keywordstyle=\bfseries\color{blue},
  commentstyle=\color{green!70!black},
  identifierstyle=\color{green!20!black},
  stringstyle=\color{orange},
  breaklines=true,
  breakatwhitespace=true,
  frame=none
}

\usepackage{xcolor}
\setkomafont{captionlabel}{\bfseries\color{red}}
\renewcommand*{\figurename}{Fig}

\usepackage{answers}
\usepackage{cancel}
\usepackage{asymptote}
\usepackage{hyperref}

\begin{document}
\title{IMO Shortlist 2007 N2}
\date{\today}
\maketitle

\begin{abstract}
    \centering https://artofproblemsolving.com/community/c6h214712
\end{abstract}

\tableofcontents

\section{Problem}
\begin{problem*}[IMO Shortlist 2007 N2]
  Let $b,n > 1$ be integers. Suppose that for each $k > 1$ there exists an integer $a_k$ such that $b - a^n_k$ is divisible by $k$. Prove that $b = A^n$ for some integer $A$.
\end{problem*}

\section{Solution 1 (Using $\nu_{p} (b)$)}
\begin{proof}
Since $b$ $>$ $1$, therefore there must exist a prime $p$ such that $p$ $\mid$ $b$. Choose any such prime $p$.
\begin{claim}
  For any prime $p$ that divides $b$, we will always have $n$ $\mid$ $\nu_p \left( b \right)$
\end{claim}
\begin{proof}
  Using the division algorithm, we can write 
  \[
    \nu_p \left( b \right) = qn + r, \quad \text{where, } 0 \leq r < n
  \]
  If $r = 0$, we are done. Hence, assume $r > 0$ here onwards. Choose an integer $\ell$ such that $\ell n > qn + r$. If we set $k = p^{\ell n}$, then 
  \begin{align*}
      p^{\ell n} \mid b - a_k^n \implies \nu_p \left( b - a_k^n \right) \geq \ell n
  \end{align*}
  If $\nu_p (b) \neq \nu_p (a_k^n)$, then we must have $\min \left( \nu_p(b), \nu_p (a_k^n) \right)$ $\geq$ $\ell n$. However,
  \begin{align*}
    \nu_p (b) = qn + r < \ell n \implies \min \left( \nu_p (b), \nu_p (a_k^n) \right) < \ell n
  \end{align*}
  this implies that, we must have $\nu_p (b) = \nu_p (a_k^n)$. However, this would mean that
  \begin{align*}
    \nu_p(a_k^n) = n \nu_p (a_k) = \nu_p (b) = qn + r \implies n \mid qn + r
  \end{align*}
  which forces $n \mid r$. Given the bounds on $r$, this is impossible contradicting our assumption $r > 0$. Hence for any prime $p$ that divides $b$, we will always have $n$ $\mid$ $\nu_p (b)$.
\end{proof}
From the claim, we have that for any prime $p$, $\nu_p(b)$ is a multiple of $n$. Therefore, $b$ must be of the form $A^n$.
\end{proof}

\section{Solution 2 (Using Construction for $k$)}
\begin{proof}
Choose $k$ $=$ $b^2$. This implies,
\begin{align*}
  b^2 \mid b - a_k^n &\Longleftrightarrow b - a_k^n = qb^2 \\ 
      &\Longleftrightarrow a_k^n = b \left(1 - qb \right)
\end{align*}
Since $\gcd(b, 1 - qb) = 1$, therefore they do not share any common prime factors. Any prime $p$ that divides $b$ must divide $a_k^n$. If
\begin{align*}
  \nu_p(a_k) = \ell \implies \nu_p(a_k^n) = \ell n
\end{align*}
Consequently, $\nu_p \left( b(1 - qb) \right) = \nu_p \left( b \right) = \ell n$ $\implies$ $b$ is of the form $A^n$.  
\end{proof}

\end{document}