\documentclass[11pt]{scrartcl}
\let\captionof\undefined
\usepackage[sexy,von]{evan}
\usepackage{wrapfig}
% \renewcommand{\vonenvname}{example}
\lstset{basicstyle=\small\ttfamily,
  numbers=left,
  numbersep=5pt,
  numberstyle=\tiny,
  keywordstyle=\bfseries,
  showstringspaces=false,
  tabsize=4,
  frame=single,
  keywordstyle=\bfseries\color{blue},
  commentstyle=\color{green!70!black},
  identifierstyle=\color{green!20!black},
  stringstyle=\color{orange},
  breaklines=true,
  breakatwhitespace=true,
  frame=none
}

\usepackage{xcolor}
\setkomafont{captionlabel}{\bfseries\color{red}}
\renewcommand*{\figurename}{Fig}

\usepackage{answers}
\usepackage{cancel}
\usepackage{asymptote}
\usepackage{hyperref}

\begin{document}
\title{IMO Shortlist 2010 N3}
\date{\today}
\maketitle

\begin{abstract}
    \centering
    https://artofproblemsolving.com/community/c6h418641p2362006
\end{abstract}

\tableofcontents

\section{Problem}
\begin{problem*}[IMO Shortlist 2010 N3]
Find the smallest number $n$ such that there exist polynomials $f_1, f_2, \ldots , f_n$ with rational coefficients satisfying\[x^2+7 = f_1\left(x\right)^2 + f_2\left(x\right)^2 + \ldots + f_n\left(x\right)^2.\]
\end{problem*}

\section{Solution 1 (Using Legendre's Three-Square Theorem)}
\begin{proof}
    We begin with the first observation that bounds the degree of the polynomials $f_i$. 
    \begin{claim}
        For any polynomial $f_i$ $\in$ $\mathbb{Q} [x]$, we will always have
        \begin{align*}
            \operatorname{deg} \left( f_i \right) \le 1
        \end{align*}
    \end{claim}
    \begin{proof}
        Comparing the degrees of polynomials on both sides of the equation, we get
        \begin{align*}
            \max \left( \operatorname{deg} (f_1(x)^2), \ldots, \operatorname{deg} (f_n(x)^2) \right) 
                            &= 2 \max \left( \operatorname{deg} (f_1), \ldots, \operatorname{deg} (f_n) \right) \\ 
                            &= \operatorname{deg} \left( x^2 + 7 \right) = 2
        \end{align*}
        which implies that $\max \left( \operatorname{deg} (f_1), \ldots, \operatorname{deg} (f_n) \right) = 1$. Hence, $\operatorname{deg} \left( f_i \right) \le 1$.
    \end{proof}
    \begin{lemma}
        For $r$ $\in$ $\mathbb{Q}$, with its simplest form $r = \tfrac{p}{q}$ for $p$, $q$ $\in$ $\mathbb{Z}$ and $q \neq 0$. If $pq$ is of the form $(8x + 7)$, then the equation
        \begin{align*}
            x^2 + y^2 + z^2 = r
        \end{align*}
        has no solutions in $\mathbb{Q}$.
    \end{lemma}
    \begin{proof}
        Suppose for the sake of contradiction, there exists rational solutions to the equation
        \begin{align*}
            x^2 + y^2 + z^2 = r
        \end{align*}
        This could be re-written as, 
        \begin{align*}
            \left( \frac{a}{b} \right)^2 + \left( \frac{c}{d} \right)^2 + \left( \frac{e}{f} \right)^2 &= \frac{p}{q} \\ 
        \Longleftrightarrow (qadf)^2 + (qcbf)^2 + (qbde)^2 &= pq (bdf)^2
        \end{align*}
        Since the quadratic residues $\pmod{8}$ are $\{ 0, 1, 4 \}$, we get that $pq(bdf)^2$ is of the form $4^k \left( 8\ell + 7\right)$ which has no integer solutions due to Legendre's three-square theorem, proving the claim.
    \end{proof}

    We claim that $\boxed{n = 5}$ is the smallest number for which there exists such polynomials. Constructing the answer is easy. Consider,
    \begin{align*}
        \left( f_1, f_2, f_3, f_4, f_5 \right) = \left( x, 1, 1, 1, 2 \right)
    \end{align*}
    squares of which add up to $x^2 + 7$. Now we shall show that this is the smallest $n$ possible.
    \begin{claim}
        For $n \le 4$, there exists no such polynomials $f_1, f_2, \ldots, f_n$ that satisfy
        \begin{align*}
            x^2 + 7 = f_1(x)^2 + f_2(x)^2 + \ldots + f_n(x)^2
        \end{align*}
    \end{claim}
    \begin{proof}
        Suppose there exist such polynomials $f_i$. Then we can write,
        \begin{align*}
            x^2 + 7 = \left( ax + e \right)^2 + \left( bx + f \right)^2 + \left( cx + g \right)^2 + \left( dx + h \right)^2
        \end{align*}
        So, we want to prove the existence of integers that satisfy,
        \begin{align*}
            a^2 + b^2 + c^2 + d^2 &= 1 \\
            ae + bf + cg + dh &= 0 \\ 
            e^2 + f^2 + g^2 + h^2 &= 7 
        \end{align*} 
        Using Euler's four-square identity,
        \begin{align*}
            &\Bigg( a^2 + b^2 + c^2 + d^2 \Bigg) \Bigg( e^2 + f^2 + g^2 + h^2 \Bigg) \\ 
            &= (ae + bf + cg + dh)^2 + (-af + be + ch - dg)^2 \\[2pt]
            &\quad + (-ag - bh + ce + df)^2 + (-ah + bg - cf + de)^2
        \end{align*}
        we get that 
        \begin{align*}
            7 = (-af + be + ch - dg)^2 + (-ag - bh + ce + df)^2 + (-ah + bg - cf + de)^2
        \end{align*}
        which has no rational solutions due to the previously proven lemma. This proves that for $n \le 4$, we have no solutions.
    \end{proof}
\end{proof}


\end{document}