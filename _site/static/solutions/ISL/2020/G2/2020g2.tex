\documentclass[11pt]{scrartcl}
\let\captionof\undefined
\usepackage[sexy,von]{evan}
\usepackage{wrapfig}
% \renewcommand{\vonenvname}{example}
\lstset{basicstyle=\small\ttfamily,
  numbers=left,
  numbersep=5pt,
  numberstyle=\tiny,
  keywordstyle=\bfseries,
  showstringspaces=false,
  tabsize=4,
  frame=single,
  keywordstyle=\bfseries\color{blue},
  commentstyle=\color{green!70!black},
  identifierstyle=\color{green!20!black},
  stringstyle=\color{orange},
  breaklines=true,
  breakatwhitespace=true,
  frame=none
}

\usepackage{xcolor}
\setkomafont{captionlabel}{\bfseries\color{red}}
\renewcommand*{\figurename}{Fig}

\usepackage{answers}
\usepackage{cancel}
\usepackage{asymptote}

\begin{document}
\title{IMO Shortlist 2020 G2}
\date{\today}
\maketitle

\begin{abstract}
    \centering
    https://artofproblemsolving.com/community/c6h2278651p17821635
\end{abstract}

\tableofcontents

\section{Problem}
\begin{problem*}[IMO Shortlist 2020 G2]
    Consider the convex quadrilateral $ABCD$. The point $P$ is in the interior of $ABCD$. The following ratio equalities hold:
\[\angle PAD:\angle PBA:\angle DPA=1:2:3=\angle CBP:\angle BAP:\angle BPC\]
Prove that the following three lines meet in a point: the internal bisectors of angles $\angle ADP$ and $\angle PCB$ and the perpendicular bisector of segment $AB$.
\end{problem*}

\begin{figure}[h]
    \centering
    \begin{asy}
        import geometry;

        size(9cm); defaultpen(fontsize(11pt));
        pen sec=heavygreen; pen pri=heavycyan; pen tri=olive;
        pen sfil=invisible; pen fil=invisible; pen tfil=invisible;

        pair O,P,A,B,I,C,D; 
        O=(0,0); P=dir(100); A=dir(200); B=dir(340); 
        I=incenter(P,A,B); 
        C=extension(B,B+(P-B)*(I-A)/(P-A),P,P+(B-P)*(P-A)/(B-A)*(P-A)/(I-A)); 
        D=extension(A,A+(P-A)*(I-B)/(P-B),P,P+(A-P)*(P-B)/(A-B)*(P-B)/(I-B));

        draw(A--B--C--D--cycle);
        draw(A--P); draw(B--P); draw(C--P); draw(D--P);
        draw(circumcircle(A, P, B), gray(0.2));
        draw(circumcircle(D, A, P), heavygray+dashed);
        draw(circumcircle(B, C, P), heavygray+dashed);
        draw(A--O, gray); draw(B--O, gray); draw(P--O, gray);
        draw(O--D, lightgray); draw(O--C, lightgray);

        dot("\(P\)",P,P); dot("\(A\)",A,SW); dot("\(B\)",B,SE); dot("\(C\)",C,N); 
        dot("\(D\)",D,N); dot("\(O\)",O,dir(15));
    \end{asy}
\end{figure}

\section{Solution 1 (Construction of Circumcenter)}
\begin{proof}
    Let $O$ be the circumcenter of $\odot(PAB)$. Due to the following chase
    \begin{align*}
        \angle AOP &= 2 \angle PBA = 4 \angle PAD \\ 
                                &= \angle PAD + \angle DPA \\ 
                                &= 180^{\circ} - \angle ADP
    \end{align*}
    we have that $\odot(APD)$ passes through $O$. Similarly $\odot(BPC)$ passes through point $O$. Since $\overline{OA}$ $=$ $\overline{OP}$ $\implies$ $\overline{OD}$ is the internal angle bisector of $\angle ADP$. Similarly $\overline{OC}$ is the internal angle bisector of $\angle BPC$. But $O$ lies on the perpendicular bisector of segment $\overline{AB}$ which proves that the desired three lines are concurrent at point $O$.
\end{proof}
\end{document}