\documentclass[11pt]{scrartcl}
\let\captionof\undefined
\usepackage[sexy,von]{evan}
\usepackage{wrapfig}
% \renewcommand{\vonenvname}{example}
\lstset{basicstyle=\small\ttfamily,
  numbers=left,
  numbersep=5pt,
  numberstyle=\tiny,
  keywordstyle=\bfseries,
  showstringspaces=false,
  tabsize=4,
  frame=single,
  keywordstyle=\bfseries\color{blue},
  commentstyle=\color{green!70!black},
  identifierstyle=\color{green!20!black},
  stringstyle=\color{orange},
  breaklines=true,
  breakatwhitespace=true,
  frame=none
}

\usepackage{xcolor}
\setkomafont{captionlabel}{\bfseries\color{red}}
\renewcommand*{\figurename}{Fig}

\usepackage{answers}
\usepackage{cancel}
\usepackage{asymptote}
\usepackage{hyperref}

\begin{document}
\title{IMO Shortlist 2020 G3}
\date{\today}
\maketitle

\begin{abstract}
    \centering
    https://artofproblemsolving.com/community/c6h2625881
\end{abstract}

\tableofcontents

\section{Problem}
\begin{problem*}[IMO Shortlist 2020 G3]
    Let $ABCD$ be a convex quadrilateral with $\angle ABC>90$, $CDA>90$ and $\angle DAB=\angle BCD$. Denote by $E$ and $F$ the reflections of $A$ in lines $BC$ and $CD$, respectively. Suppose that the segments $AE$ and $AF$ meet the line $BD$ at $K$ and $L$, respectively. Prove that the circumcircles of triangles $BEK$ and $DFL$ are tangent to each other.
\end{problem*}

\section{Solution 1 (Using Steiner Line)}
\begin{figure}[h]
    \centering
    \begin{asy}
        import geometry;
        size(12cm); defaultpen(fontsize(12pt));

        pair A, B, C, D, P;
        B = (3, 0); D = (-3, 0); A = (-0.8, -4);
        
        P = reflect(B, D) * A;
        
        pair O = circumcenter(B, P, D);
        C = O + abs(B - O) * dir(80);

        dot("$A$", A, dir(225));
        dot("$B$", B, dir(315));
        dot("$C$", C, dir(85));
        dot("$D$", D, dir(225));
        dot("$P$", P, dir(155));

        draw(circumcircle(B, P, D));
        draw(A--B--C--D--cycle);

        pair E, F, K, L, X;
        E = reflect(B, C) * A;
        F = reflect(C, D) * A;
        K = extension(B, D, A, E);
        L = extension(A, F, B, D);

        dot("$K$", K, dir(325));
        dot("$L$", L, dir(225));
        dot("$E$", E, dir(315));
        dot("$F$", F, dir(135));

        draw(A--F); draw(A--E); draw(K--L); draw(A--C);
        draw(F--D); draw(B--E); draw(C--F); draw(C--E);

        pair DD = foot(D, A, F);
        draw(D--DD, gray+dotted); markrightangle(A, DD, D, 7);

        pair BB = foot(B, A, E);
        draw(B--BB, gray+dotted); markrightangle(B, BB, A, 7);

        draw(B--P, gray); draw(D--P, gray); draw(A--P, gray+dotted);
        pair AA = foot(A, B, D); markrightangle(B, AA, P, 7);

        pair P1 = reflect(C, D) * P; pair P2 = reflect(B, C) * P;
        draw(A--P2, red);

        draw(arc(circumcircle(F, D, P), -190, 40), heavygray+dashed);
        draw(arc(circumcircle(P, B, E), -180, -30), heavygray+dashed);
        
        X = extension(A, P2, B, D);
        dot("$X$", X, dir(315)); draw(X--P, red);
    \end{asy}
\end{figure}
\begin{proof}
    Define $P$ as the reflection of $A$ over $\overline{BD}$. We will show that $P$ is the point of tangency of the circumcircles $\odot(BEK)$ and $\odot(DFL)$.
    \begin{claim}
        $P$ lies on the circle $\odot(BCD)$.
    \end{claim}
    \begin{proof}
        Since $\angle BPD$ $=$ $\angle BAD$ $=$ $\angle BCD$ $\implies$ $P$ lies on $\odot(BCD)$.
    \end{proof}

    \begin{claim}
        Quadrilaterals $DLFP$ and $BKEP$ are cyclic.
    \end{claim}
    \begin{proof}
        Since $\angle DFL$ $=$ $\angle DAL$ $=$ $\angle DPL$ $\implies$ $DLFP$ is a cyclic quadrilateral. Similarly, we can show that $BKEP$ is a cyclic quadrilateral too.
    \end{proof}
    Draw the steiner line of $P$ wrt $\triangle BCD$ and let it intersect $BD$ at $X$. Then,
    \begin{align*}
    \angle XPD = \angle XAD = \angle DFP
    \end{align*}
    which follows due to $XA$ being the reflection of $FP$ over $\overline{CD}$. Hence, $\overline{XP}$ is tangent to $\odot(DLFP)$ at $P$. Similarly, we can show that $\overline{XP}$ is tangent to $\odot(BKEP)$ at $P$. Thus, $\odot(BEK)$ and $\odot(DFL)$ are tangent to each other at $P$.
\end{proof}
\end{document}