\documentclass[11pt]{scrartcl}
\let\captionof\undefined
\usepackage[sexy,von]{evan}
\usepackage{wrapfig}
% \renewcommand{\vonenvname}{example}
\lstset{basicstyle=\small\ttfamily,
  numbers=left,
  numbersep=5pt,
  numberstyle=\tiny,
  keywordstyle=\bfseries,
  showstringspaces=false,
  tabsize=4,
  frame=single,
  keywordstyle=\bfseries\color{blue},
  commentstyle=\color{green!70!black},
  identifierstyle=\color{green!20!black},
  stringstyle=\color{orange},
  breaklines=true,
  breakatwhitespace=true,
  frame=none
}

\usepackage{xcolor}
\setkomafont{captionlabel}{\bfseries\color{red}}
\renewcommand*{\figurename}{Fig}

\usepackage{answers}
\usepackage{cancel}
\usepackage{asymptote}

\begin{document}
\title{1st AGO Shortlist G3}
\date{\today}
\maketitle

\begin{abstract}
    \centering https://artofproblemsolving.com/community/c6h3336441p30900015
\end{abstract}

\section{Problem}
\begin{problem*}[1st AGO Shortlist G3]
    Let $ABC$ be an acute triangle with circumcircle $\Omega$ and $K$ be a point such that $ABKC$ is a parallelogram. Lines $BK$ and $CK$ intersect $\Omega$ again at $P$ and $Q$, respectively. The line that passes through $K$ and perpendicular to $BC$ intersects $\Omega$ at $X$ such that $X$ is closer to $K$. Prove that $XP=XQ$.
\end{problem*}

\section{Solution}
\begin{figure}[h]
    \centering
    \begin{asy}
        import geometry;
        size(8cm); defaultpen(fontsize(10pt));

        pair A = dir(120);
        pair B = dir(195);
        pair C = dir(345);
        pair K = B + C - A;

        pair X = orthocenter(B, K, C);
        pair[] PP = intersectionpoints(line(B, K), circumcircle(A, B, C));
        pair P = PP[0];

        pair[] QQ = intersectionpoints(line(K, C), circumcircle(A, B, C));
        pair Q = QQ[0];

        pair H = orthocenter(A, B, C);

        draw(A--B--C--cycle); draw(A--H);
        draw(B--K--C); draw(A--P); draw(A--Q); draw(A--X); draw(K--X); 
        draw(circumcircle(A, B, C)); draw(circumcircle(B, K, C));

        dot("$A$", A, dir(130));
        dot("$B$", B, dir(225));
        dot("$C$", C, dir(315));
        dot("$K$", K, dir(315));
        dot("$H$", H, dir(170));
        dot("$X$", X, dir(80));
        dot("$P$", P, dir(225));
        dot("$Q$", Q, dir(0)); draw(K--H);
        draw(circumcircle(K, P, Q), gray+dashed);
    \end{asy}
\end{figure}

\begin{proof}
    Let $H$ be the orthocenter of $\triangle ABC$. Since $\triangle ABC$ $\cong$ $\triangle BKC$ $\implies$ $\odot(ABC)$ is the reflection of $\odot(BKC)$ over $\overline{BC}$. Hence $H$ lies on $\odot(BKC)$. Infact $\triangle ABC \cup H$ is mapped to $\triangle KCB \cup X$ under reflection over midpoint of $\overline{BC}$. Thus
    \begin{align*}
        \angle XPK = \angle BAX = \angle HAC = \angle BKX = \angle PKX
    \end{align*}
    which implies that $\overline{XP}$ $=$ $\overline{XK}$. Similarly, we can show that $\overline{XK}$ $=$ $\overline{XQ}$ $\implies$ $\overline{XP}$ $=$ $\overline{XQ}$.
\end{proof}
\end{document}