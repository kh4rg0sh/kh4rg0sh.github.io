\documentclass[11pt]{scrartcl}
\let\captionof\undefined
\usepackage[sexy,von]{evan}
\usepackage{wrapfig}
% \renewcommand{\vonenvname}{example}
\lstset{basicstyle=\small\ttfamily,
  numbers=left,
  numbersep=5pt,
  numberstyle=\tiny,
  keywordstyle=\bfseries,
  showstringspaces=false,
  tabsize=4,
  frame=single,
  keywordstyle=\bfseries\color{blue},
  commentstyle=\color{green!70!black},
  identifierstyle=\color{green!20!black},
  stringstyle=\color{orange},
  breaklines=true,
  breakatwhitespace=true,
  frame=none
}

\usepackage{xcolor}
\setkomafont{captionlabel}{\bfseries\color{red}}
\renewcommand*{\figurename}{Fig}

\usepackage{answers}
\usepackage{cancel}
\usepackage{asymptote}

\begin{document}
\title{1st AGO Shortlist G5}
\date{\today}
\maketitle

\begin{abstract}
    \centering https://artofproblemsolving.com/community/c6h3352516p31104702
\end{abstract}

\section{Problem}
\begin{problem*}[1st AGO Shortlist G5]
    Let $ABC$ be a triangle and $X$ be a point distinct from $A,B,C$. A line $l$ intersects lines $AX, BX, CX$ at $D, E, F$ respectively. The perpendicular bisectors of segments $DX, EX, FX$ define a triangle with circumcircle $\Theta$. Prove that $X$ lies on $\Theta$.
\end{problem*}

\section{Solution}
\begin{figure}[h]
    \centering
    \begin{asy}
        import geometry;
        size(12cm); defaultpen(fontsize(12pt));

        pair A = dir(120);
        pair B = dir(210);
        pair C = dir(330);

        pair X = (-0.3, 0.4);
        pair Y = (0.4, -1);
        pair Z = (0.4, 1);

        pair D = extension(A, X, Y, Z);
        pair E = extension(B, X, Y, Z);
        pair F = extension(C, X, Y, Z);

        pair K = circumcenter(X, E, F);
        pair L = circumcenter(X, F, D);
        pair M = circumcenter(X, D, E);

        dot("$A$", A, dir(130));
        dot("$B$", B, dir(225));
        dot("$C$", C, dir(315));
        dot("$X$", X, dir(180));
        dot("$D$", D, dir(290));
        dot("$E$", E, dir(45));
        dot("$F$", F, dir(45));
        dot("$K$", K, dir(45));
        dot("$L$", L, dir(225));
        dot("$M$", M, dir(45));

        draw(A--B--C--cycle);
        draw(circumcircle(K, L, M), gray+dashed);
        draw(D--E); draw(A--D); 
        draw(B--E); draw(C--X);
        draw(X--L--D); draw(X--M--D); draw(X--K--E);
        draw(K--F); draw(L--F); draw(M--E);

        draw(circumcircle(X, E, F), gray);
        draw(circumcircle(X, F, D), gray);
        draw(circumcircle(X, D, E), gray);
        draw(M--L--K--cycle, heavygray);
    \end{asy}
\end{figure}
\begin{proof}
    Let the center of the circles $\odot(XEF)$, $\odot(XDF)$ and $\odot(XFE)$ be $K$, $L$ and $M$. So,
    \begin{align*}
        \angle XKL = \tfrac{1}{2} \angle XKF = \angle XEF = \angle XED = \tfrac{1}{2} \angle XMD = \angle XML
    \end{align*}
    which implies that $XKML$ is a cyclic quadrilateral, as desired.
\end{proof}
\end{document}