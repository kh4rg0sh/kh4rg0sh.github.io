\documentclass[11pt]{scrartcl}
\let\captionof\undefined
\usepackage[sexy,von]{evan}
\usepackage{wrapfig}
% \renewcommand{\vonenvname}{example}
\lstset{basicstyle=\small\ttfamily,
  numbers=left,
  numbersep=5pt,
  numberstyle=\tiny,
  keywordstyle=\bfseries,
  showstringspaces=false,
  tabsize=4,
  frame=single,
  keywordstyle=\bfseries\color{blue},
  commentstyle=\color{green!70!black},
  identifierstyle=\color{green!20!black},
  stringstyle=\color{orange},
  breaklines=true,
  breakatwhitespace=true,
  frame=none
}

\usepackage{xcolor}
\setkomafont{captionlabel}{\bfseries\color{red}}
\renewcommand*{\figurename}{Fig}

\usepackage{answers}
\usepackage{cancel}
\usepackage{asymptote}

\begin{document}
\title{1st AGO Shortlist G4}
\date{\today}
\maketitle

\begin{abstract}
    \centering https://artofproblemsolving.com/community/c6h3336442p30900021
\end{abstract}

\section{Problem}
\begin{problem*}[1st AGO Shortlist G4]
    Given a triangle $ABC$, points $A_0, B_0, C_0$ lie on sides $BC, CA, AB$ respectively. Triangle $A_0B_0C_0$ is called $\textit{reflective}$ with respect to triangle $ABC$ if all of the following hold.$$\angle BA_0C_0 = \angle CA_0B_0, \qquad \angle BC_0A_0 = \angle AC_0B_0, \qquad \angle AB_0C_0 = \angle CB_0A_0.$$
Prove that every acute triangle $ABC$ has a unique reflective triangle with respect to triangle $ABC$.
\end{problem*}

\section{Solution}
\begin{figure}[h]
    \centering
    \begin{asy}
        import geometry;
        size(8cm); defaultpen(fontsize(10pt));

        pair A, B, C, A0, B0, C0, H;
        A = dir(120);
        B = dir(210);
        C = dir(330);
        A0 = foot(A, B, C);
        B0 = foot(B, A, C);
        C0 = foot(C, A, B);
        H = orthocenter(A, B, C);

        draw(A--B--C--cycle);
        draw(A0--B0--C0--cycle);
        dot("$A$", A, dir(130));
        dot("$B$", B, dir(225));
        dot("$C$", C, dir(315));
        dot("$A_0$", A0, dir(225));
        dot("$B_0$", B0, dir(45));
        dot("$C_0$", C0, dir(135));
        dot("$I$", H, dir(0));

        draw(A0--H); draw(B0--H); draw(C0--H);
        draw(circumcircle(A, B0, C0), gray+dashed);
        draw(circumcircle(A0, B, C0), gray+dashed);
        draw(circumcircle(A0, B0, C), gray+dashed);
    \end{asy}
\end{figure}

\begin{proof}
    It's easy to verify that the orthic triangle satifies the condition. Now we will show that it is the only triangle that does. Suppose $\triangle A_0B_0C_0$ is not the orthic triangle of $\triangle ABC$. Suppose $I$ is the incenter of $\triangle A_0B_0C_0$. Then $\overline{IA_0}$ $\perp$ $\overline{BC}$, $\overline{IB_0}$ $\perp$ $\overline{AC}$ and $\overline{IC_0}$ $\perp$ $\overline{AB}$ $\implies$ $AC_0IB_0$, $BC_0IA_0$ and $CA_0IB_0$ are cyclic. Moreover, $A$ is the excenter opposite to vertex to $A_0$ which implies that $A_0I$ passes through $A$. Similarly, $B_0I$ passess through $B$ and $C_0I$ passes through $C$. Thus $A_0$, $B_0$ and $C_0$ are the feet of perpendiculars from $A$, $B$ and $C$ $\implies$ $\triangle A_0B_0C_0$ is the orthic triangle, arriving at a contradiction to our initial assumption.
\end{proof}
\end{document}
