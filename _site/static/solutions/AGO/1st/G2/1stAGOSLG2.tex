\documentclass[11pt]{scrartcl}
\let\captionof\undefined
\usepackage[sexy,von]{evan}
\usepackage{wrapfig}
% \renewcommand{\vonenvname}{example}
\lstset{basicstyle=\small\ttfamily,
  numbers=left,
  numbersep=5pt,
  numberstyle=\tiny,
  keywordstyle=\bfseries,
  showstringspaces=false,
  tabsize=4,
  frame=single,
  keywordstyle=\bfseries\color{blue},
  commentstyle=\color{green!70!black},
  identifierstyle=\color{green!20!black},
  stringstyle=\color{orange},
  breaklines=true,
  breakatwhitespace=true,
  frame=none
}

\usepackage{xcolor}
\setkomafont{captionlabel}{\bfseries\color{red}}
\renewcommand*{\figurename}{Fig}

\usepackage{answers}
\usepackage{cancel}
\usepackage{asymptote}

\begin{document}
\title{1st AGO Shortlist G2}
\date{\today}
\maketitle

\begin{abstract}
    \centering https://artofproblemsolving.com/community/c6h3336439p30900010
\end{abstract}

\section{Problem}
\begin{problem*}[1st AGO Shortlist G2]
    Let $P$ and $Q$ be points on the sides $AB$ and $AD$, respectively, of a convex quadrilateral $ABCD$ such that $PQ \parallel BD$. Let the segments $CP$ and $CQ$ intersect $BD$ at points $G$ and $H$, respectively. Prove that if the quadrilateral $AGCH$ is a parallelogram, then the quadrilateral $ABCD$ is a parallelogram.
\end{problem*}

\section{Solution}
\begin{figure}[h]
    \centering
    \begin{asy}
        import geometry;
        size(8cm); defaultpen(fontsize(10pt));

        pair O = origin;
        pair A = (-2, 3); pair B = (-4, -1);
        pair C = 2 * O - A; pair D = 2 * O - B;

        real r = 0.35;
        pair P = A + r * (B - A);
        pair Q = A + r * (D - A);

        pair G = extension(C, P, B, D);
        pair H = extension(C, Q, B, D);

        draw(A--B--C--D--cycle);
        draw(A--C); draw(B--D);
        draw(C--P); draw(C--Q);
        draw(A--G); draw(A--H);
        draw(P--Q);

        dot("$A$", A, dir(120));
        dot("$B$", B, dir(225));
        dot("$C$", C, dir(225));
        dot("$D$", D, dir(60));
        dot("$P$", P, dir(135));
        dot("$Q$", Q, dir(45));
        dot("$G$", G, dir(245));
        dot("$H$", H, dir(65));
    \end{asy}
\end{figure}
\begin{proof}
    If $AGCH$ is a parallelogram, then we can show that $\overline{BG}$ $=$ $\overline{HD}$ because
    \begin{align*}
        \frac{\overline{BG}}{\overline{GH}} = \frac{\overline{BP}}{\overline{PA}} = \frac{\overline{DQ}}{\overline{QA}} = \frac{\overline{HD}}{\overline{GH}}
    \end{align*}
    Using this we can use SAS congruence criterion to show that $\triangle AGB$ $\cong$ $\triangle CHD$ and $\triangle ABH$ $\cong$ $\triangle CDG$. Therefore $\overline{AB}$ $\parallel$ $\overline{CD}$ and $\overline{AD}$ $\parallel$ $\overline{BC}$ which implies that $ABCD$ is a parallelogram. If $ABCD$ is a parallelogram, then $\overline{AC}$ bisects $\overline{BD}$ $\implies$ $\overline{AC}$ bisects $\overline{PQ}$ $\implies$ $\overline{AC}$ bisects $\overline{GH}$. Since $\overline{BD}$ bisects $\overline{AC}$ $\implies$ $\overline{GH}$ bisects $\overline{AC}$ $\implies$ $AGCH$ is a parallelogram.
\end{proof}
\end{document}