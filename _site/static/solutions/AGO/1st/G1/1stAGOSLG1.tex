\documentclass[11pt]{scrartcl}
\let\captionof\undefined
\usepackage[sexy,von]{evan}
\usepackage{wrapfig}
% \renewcommand{\vonenvname}{example}
\lstset{basicstyle=\small\ttfamily,
  numbers=left,
  numbersep=5pt,
  numberstyle=\tiny,
  keywordstyle=\bfseries,
  showstringspaces=false,
  tabsize=4,
  frame=single,
  keywordstyle=\bfseries\color{blue},
  commentstyle=\color{green!70!black},
  identifierstyle=\color{green!20!black},
  stringstyle=\color{orange},
  breaklines=true,
  breakatwhitespace=true,
  frame=none
}

\usepackage{xcolor}
\setkomafont{captionlabel}{\bfseries\color{red}}
\renewcommand*{\figurename}{Fig}

\usepackage{answers}
\usepackage{cancel}
\usepackage{asymptote}

\begin{document}
\title{1st AGO Shortlist G1}
\date{\today}
\maketitle

\begin{abstract}
    \centering https://artofproblemsolving.com/community/c6h3352519p31104706
\end{abstract}

\section{Problem}
\begin{problem*}[1st AGO Shortlist G1]
    Let $ABC$ be a triangle and $D$ a point on side $BC$. Points $M$ and $N$ are midpoints of sides $AB$ and $AC$, respectively.$P$ and $Q$ are points such that $ABPD$ and $ACQD$ are parallelograms. Prove that lines $BC, PN$ and $QM$ concur.
\end{problem*}
\section{Solution}
\begin{figure}[h]
    \centering
    \begin{asy}
        import geometry;
        size(10cm); defaultpen(fontsize(12pt));

        pair A = dir(120);
        pair B = dir(200);
        pair C = dir(340);

        real r = 0.32;
        pair D = B + (C - B) * r;

        pair M = (A + B) / 2; pair N = (A + C) / 2;
        pair G = (A + D) / 2; pair F = 2 * G - M;
        pair E = 2 * G - N; pair P = B + D - A; 
        pair Q = D + C - A; pair R = extension(M, Q, B, C);

        dot("$A$", A, dir(130));
        dot("$B$", B, dir(225));
        dot("$C$", C, dir(315));
        dot("$D$", D, dir(225));
        dot("$E$", E, dir(135));
        dot("$N$", N, dir(45));
        dot("$P$", P, dir(225));
        dot("$Q$", Q, dir(315));
        dot("$R$", R, dir(315));
        dot("$M$", M, dir(135));
        dot("$G$", G, dir(45));
        dot("$F$", F, dir(45));

        draw(A--B--C--cycle);
        draw(B--P); draw(P--F);
        draw(Q--E); draw(Q--M);
        draw(M--D); draw(A--D); draw(A--F);
        draw(C--Q); draw(P--N, gray+dashed);
        draw(E--N); draw(P--Q);

    \end{asy}
\end{figure}
\begin{proof}
    Suppose $PD$ intersects $MN$ at $F$. Then $\overline{DF}$ $=$ $\overline{BM}$ $=$ $\overline{AM}$. Thus $AMDF$ is a parallelogram. Let $G$ be the midpoint of $\overline{MF}$ and $E$ be the reflection of $N$ over $G$. Since $\overline{AG}$ $=$ $\overline{GD}$ and $\overline{GN}$ $=$ $\overline{GE}$ $\implies$ $ANDE$ is a parallelogram. But $\overline{DQ}$ $\parallel$ $\overline{AC}$, thus $E$ lies on $DQ$. Let $QM$ intersect $BC$ at $R$, then
    \begin{align*}
    \frac{\overline{DR}}{\overline{FN}} = \frac{\overline{DR}}{\overline{EM}} = \frac{\overline{DQ}}{\overline{EQ}} = \frac{1}{\tfrac{\overline{EQ}}{\overline{DQ}}} = \frac{1}{1 + \tfrac{\overline{ED}}{\overline{DQ}}} = \frac{1}{1 + \tfrac{\overline{FD}}{\overline{DP}}} = \frac{1}{\tfrac{\overline{FP}}{\overline{DP}}} = \frac{\overline{DP}}{\overline{FP}}
    \end{align*}
    which implies that $\triangle PDR$ $\sim$ $\triangle PFN$ $\implies$ $R$ lies on $\overline{PN}$. In other words, $\overline{PN}$, $\overline{QM}$ and $\overline{BC}$ are concurrent at $R$.
\end{proof}

\end{document}