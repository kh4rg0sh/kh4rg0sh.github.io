We first introduce one way to define conic sections from high school geometry:

\begin{definition}
    Let $\odot(O)$ be a circle in space. Draw a line through $O$ such that $\ell$ is perpendicular to the plane of $\odot(O)$. Take any point $V$ on $\ell$ such that $V$ $\neq$ $O$. When a moving point $M$ $\in$ $\odot(O)$ moves along the circle, the surface formed by the lines $VM$ is called a \vocab{Circular Conic Surface}. The circle $\odot(O)$ is called its \vocab{Directrix}, and $V$ is called its \vocab{Vertex}.
\end{definition}

\begin{definition}
    A curve $\mathcal{C}$ on a plane $E$ is called a \vocab{Conic}, if there exists a circular conical surface $S$ with vertex $V$ such that $\mathcal{C}$ $=$ $S \cap E$.
\end{definition}

Later, when the spatial context is not mentioned, we will omit ``on plane $E$''. Because the equation of a conic is quadratic, conics are also called \emph{quadratic curves}.

\begin{definition}
    Let $\mathcal{C}$ be a conic and $\mathcal{L}_{\infty}$ be the line at infinity. Then:
    \begin{enumerate}
        \item $\mathcal{C}$ is called an $\vocab{Ellipse}$, if $\mid \mathcal{C} \cap \mathcal{L}_{\infty} \mid = 0$.
        \item $\mathcal{C}$ is called an $\vocab{Parabola}$, if $\mid \mathcal{C} \cap \mathcal{L}_{\infty} \mid = 1$.
        \item $\mathcal{C}$ is called an $\vocab{Hyperbola}$, if $\mid \mathcal{C} \cap \mathcal{L}_{\infty} \mid = 2$.
    \end{enumerate}
\end{definition}
Hereafter, $\mathcal{L}_{\infty}$ will be used to denote the \vocab{Line at Infinity}. For a line $\ell$ $\neq$ $\mathcal{L}_{\infty}$, the intersection point 
\begin{align*}
    \infty_{\ell} \coloneqq \ell \cap \mathcal{L}_{\infty}
\end{align*}
is called the \vocab{Point at Infinity on $\ell$}. The following are the more commonly seen equivalent definitions of conic sections:

\begin{proposition}
    Let $\mathcal{C}$ be a conic. Then
    \begin{enumerate}
        \item $\mathcal{C}$ is an ellipse \emph{if and only if} there exists two points $F_1$, $F_2$ and a positive real number $\alpha$, where $\alpha > \tfrac{\overline{F_1F_2}}{2}$, such that 
            \begin{align*}
                \mathcal{C} = \{ \, P \, \vert \, \overline{F_1P} + \overline{F_2P} = 2 \alpha \}.
            \end{align*}
        \item $\mathcal{C}$ is a parabola \emph{if and only if} there exists a point $F$ and a line $\ell$ such that
            \begin{align*}
                \mathcal{C} = \{ \, P \, \vert \, \overline{FP} = d \left( \ell, P \right) \}.
            \end{align*}
        \item $\mathcal{C}$ is a hyperbola \emph{if and only if} there exists two points $F_1$, $F_2$ and a positive real number $\alpha$, where $\alpha < \tfrac{\overline{F_1F_2}}{2}$, such that
            \begin{align*}
                \mathcal{C} = \{ \, P \, \big\vert \, \vert \overline{F_1P} - \overline{F_2P} \vert = 2 \alpha \}.
            \end{align*}
    \end{enumerate}
    \noindent Moreover, the points $F_1$, $F_2$ are called the focii of $\mathcal{C}$, and the line $\ell$ is called the directrix of $\mathcal{C}$.
\end{proposition}

\begin{proof}
// to be added
\end{proof}

Before discussing the next property, we first state the following: \emph{all conic sections are smooth}. We also introduce some more notations.
\begin{enumerate}
    \item $T \left( \mathcal{C} \right)$ denotes the set of all tangents to $\mathcal{C}$.
    \item $T_{P} \left( \mathcal{C} \right)$ denotes the tangent to $\mathcal{C}$ at the point $P$, where $P \in \mathcal{C}$.
    \item $T_{\ell} \left( \mathcal{C} \right)$ denotes the point of tangency of the line $\ell$ with $\mathcal{C}$, where $\ell \in T \left( \mathcal{C} \right)$.
\end{enumerate}
Thus, it is clear that
\begin{align*}
T \left( \mathcal{C} \right) = \{ T_{P} \left( \mathcal{C} \right) \, \big\vert \, P \in \mathcal{C} \}, \twotab \mathcal{C} = \{ \, T_{\ell} \left( \mathcal{C} \right) \, \big\vert \, \ell \in T \left( \mathcal{C} \right) \}.
\end{align*}
