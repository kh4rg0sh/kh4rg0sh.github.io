\documentclass[11pt]{scrartcl}
\let\captionof\undefined
\usepackage[sexy,von]{evan}
\usepackage{wrapfig}
% \renewcommand{\vonenvname}{example}
\lstset{basicstyle=\small\ttfamily,
  numbers=left,
  numbersep=5pt,
  numberstyle=\tiny,
  keywordstyle=\bfseries,
  showstringspaces=false,
  tabsize=4,
  frame=single,
  keywordstyle=\bfseries\color{blue},
  commentstyle=\color{green!70!black},
  identifierstyle=\color{green!20!black},
  stringstyle=\color{orange},
  breaklines=true,
  breakatwhitespace=true,
  frame=none
}

\usepackage{xcolor}
\setkomafont{captionlabel}{\bfseries\color{red}}
\renewcommand*{\figurename}{Fig}

\usepackage{answers}
\usepackage{cancel}
\usepackage{asymptote}
\usepackage{hyperref}

\begin{document}
\title{Coaxial Circles}
\date{\today}
\maketitle

\begin{abstract}
    \centering In this article, we will discuss about the properties of coaxial circles.
\end{abstract}

So far, we have studied tools for dealing with pairs of circles that have distinct radical axes. Let us now consider the situation where multiple circles share the same radical axis.
\begin{definition}
    A family of circles is called \vocab{Coaxial} if they all share the same radical axis.
\end{definition}

\section{Reim's Theorem}
This particular configuration appears very frequently in geometry problems. 
\begin{theorem}[Reim's Theorem]
    Suppose circles $\omega_1$ and $\omega_2$ intersect in points $A$ and $B$. Let $\ell_1$ and $\ell_2$ be two lines through passing through $A$ and $B$ such that they intersect $\omega_1$ in $U$ and $V$, and $\omega_2$ in $X$ and $Y$. Then $\overline{UV}$ $\parallel$ $\overline{XY}$.
\end{theorem}

\begin{figure}[h]
    \centering
    \begin{asy}
        import geometry;
        size(7cm); defaultpen(fontsize(10pt));

        pair O1, O2;
        O1 = (0, 0); O2 = (0, 2);

        path C1 = circle(O1, 2.5);
        path C2 = circle(O2, 1);

        draw(C1); draw(C2);

        pair[] AA = intersectionpoints(C1, C2);
        dot("$A$", AA[1], dir(135)); dot("$B$", AA[0], dir(45));

        pair X, Y;
        X = dir(10) * 2.5; Y = dir(230) * 2.5;
        dot("$X$", X, dir(325)); dot("$Y$", Y, dir(225));

        pair[] UU = intersectionpoints(line(AA[1], X), C2);
        pair[] VV = intersectionpoints(line(AA[0], Y), C2);

        dot("$U$", UU[1], dir(15));
        dot("$V$", VV[1], dir(120));

        draw(AA[1]--X); draw(AA[0]--Y);
        draw(X--Y, red); draw(UU[1]--VV[1], red);
    \end{asy}
\end{figure}
Furthermore, the converse of this theorem is true too.

\begin{theorem}[Converse of Reim's Theorem (Collinearity)]
    Suppose two circles $\omega_1$ and $\omega_2$ intersect in points $A$ and $B$. Let $\ell$ be a line that passes through $A$ and intersects $\omega_1$ and $\omega_2$ at $X$ and $U$. Suppose $Y$ and $V$ lie on circles $\omega_1$ and $\omega_2$ such that $\overline{UV}$ $\parallel$ $\overline{XY}$. Then, points $B$, $V$ and $Y$ are collinear.
\end{theorem}

\begin{figure}[h]
    \centering
    \begin{asy}
        import geometry;
        size(7cm); defaultpen(fontsize(10pt));

        pair O1, O2;
        O1 = (0, 0); O2 = (0, 2);

        path C1 = circle(O1, 2.5);
        path C2 = circle(O2, 1);

        draw(C1); draw(C2);

        pair[] AA = intersectionpoints(C1, C2);
        dot("$A$", AA[1], dir(135)); dot("$B$", AA[0], dir(45));

        pair X, Y;
        X = dir(10) * 2.5; Y = dir(230) * 2.5;
        dot("$X$", X, dir(325)); dot("$Y$", Y, dir(225));

        pair[] UU = intersectionpoints(line(AA[1], X), C2);
        pair[] VV = intersectionpoints(line(AA[0], Y), C2);

        dot("$U$", UU[1], dir(15));
        dot("$V$", VV[1], dir(120));

        draw(AA[1]--X); draw(AA[0]--Y, dashed);
        draw(X--Y, red); draw(UU[1]--VV[1], red);
    \end{asy}
\end{figure}

\begin{theorem}[Converse of Reim's Theorem (Concyclicity)]
    Given a circle $\omega$ and four points $A$, $B$, $X$ and $Y$ on the circle. Choose points $U$ and $V$ on $\overline{AX}$ and $\overline{BY}$ such that $\overline{UV}$ $\parallel$ $\overline{XY}$. Then the points $A$, $B$, $U$ and $V$ are concyclic.
\end{theorem}

\begin{figure}[h]
    \centering
    \begin{asy}
        import geometry;
        size(7cm); defaultpen(fontsize(10pt));

        pair O1, O2;
        O1 = (0, 0); O2 = (0, 2);

        path C1 = circle(O1, 2.5);
        path C2 = circle(O2, 1);

        draw(C1); draw(C2, red+dashed);

        pair[] AA = intersectionpoints(C1, C2);
        dot("$A$", AA[1], dir(135)); dot("$B$", AA[0], dir(45));

        pair X, Y;
        X = dir(10) * 2.5; Y = dir(230) * 2.5;
        dot("$X$", X, dir(325)); dot("$Y$", Y, dir(225));

        pair[] UU = intersectionpoints(line(AA[1], X), C2);
        pair[] VV = intersectionpoints(line(AA[0], Y), C2);

        dot("$U$", UU[1], dir(15));
        dot("$V$", VV[1], dir(120));

        draw(AA[1]--X); draw(AA[0]--Y);
        draw(X--Y, blue); draw(UU[1]--VV[1], blue);
    \end{asy}
\end{figure}

To be precise, both of these results follow from a simple two-step angle chase. Nevertheless, because they appear so frequently in various configurations, it is important to have them at your fingertips to avoid overlooking any pair of parallel lines or cyclic quadrilaterals.

\subsection{Examples}
\begin{problem}[USA TSTST 2019]
Let $ABC$ be an acute triangle with circumcircle $\Omega$ and orthocenter $H$. Points $D$ and $E$ lie on segments $AB$ and $AC$ respectively, such that $AD = AE$. The lines through $B$ and $C$ parallel to $\overline{DE}$ intersect $\Omega$ again at $P$ and $Q$, respectively. Denote by $\omega$ the circumcircle of $\triangle ADE$.
\begin{enumerate}
    \item Show that lines $PE$ and $QD$ meet on $\omega$.
    \item Prove that if $\omega$ passes through $H$, then lines $PD$ and $QE$ meet on $\omega$ as well.
\end{enumerate}
\end{problem}
\begin{figure}[h]
    \centering
    \begin{asy}
        size(8cm);
        defaultpen(fontsize(10pt));

        pair O, A, B, C, H, L, P, Q, Y, D, EE, X, SS, T;
        O=(0,0);
        A=dir(110);
        B=dir(220);
        C=dir(320);
        H=A+B+C;
        L=dir(270);
        P=2*foot(O,B,foot(B,A,L))-B;
        Q=2*foot(O,C,foot(C,A,L))-C;
        Y=A+P+Q;
        D=extension(A,B,P,Y);
        EE=extension(A,C,Q,Y);
        X=extension(P,EE,Q,D);
        SS=foot(P,A,Q);
        T=foot(Q,A,P);

        draw(circle(O,1));
        draw(A--B--C--cycle);
        draw(B--P, heavygray);
        draw(C--Q, heavygray);
        draw(circumcircle(A,D,EE),gray(0.6));
        draw(D--EE, heavygray);
        draw(H--A--Y--cycle, heavygray);

        dot("$A$",A,N);
        dot("$B$",B,B);
        dot("$C$",C,C);
        dot("$H$",H,S);
        dot("$P$",P,P);
        dot("$Q$",Q,Q);
        dot("$Y$",Y,unit(Y-A));
        dot("$D$",D,dir(210));
        dot("$E$",EE,dir(18));
        dot("$X$",X,NW);

        draw(P--X, gray(0.2)+dashed);
        draw(Q--X, gray(0.2)+dashed);

        pair M, N;
        N = reflect(A, B) * H;
        M = reflect(A, C) * H;

        dot("$M$", M, dir(45));
        dot("$N$", N, dir(190));

        draw(B--M); draw(C--N);
        draw(P--N, gray(0.4)+dashed);
        draw(Q--M, gray(0.4)+dashed);

        draw(arc(circumcircle(N, M, H), 200, 360), gray(0.75)+dashed);
    \end{asy}
\end{figure}
\begin{proof}
For the first part, suppose $\omega$ $\cap$ $\Omega$ at $X$ $\neq$ $A$. Then applying converse of reim's theorem on $\overline{DE}$ $\parallel$ $\overline{BP}$, we get that $PE$ passes through $X$. Similarly, $QD$ passes through $X$ proving that $PE$ and $QD$ indeed meet on $\omega$ at $X$.

For the second part, suppose $Y$ lies on $\omega$ such that $\overline{AH}$ and$\overline{AY}$ are isogonal with respect to $\angle BAC$. Then, $\overline{HY}$ $\parallel$ $\overline{DE}$. Let $M$ and $N$ be the reflections of $H$ over $\overline{AC}$ and $\overline{AB}$. We can show that $D$ lies on $\overline{NY}$, because
\begin{align*}
    \angle NDH &= 2 \angle BDH \\ 
            &= 2 \left( 180^{\circ} - \angle ADH \right) \\ 
            &= 2 \angle AYH \\ 
            &= 2 \left( 90^{\circ} - \tfrac{1}{2} \angle HAY \right) \\ 
            &= 180^{\circ} - \angle HDY
\end{align*}
Similarly, we can show that $E$ lies on $\overline{MY}$. With a simple angle chase, we can show that $NHYM$ is a cyclic quadrilateral. By the converse of reim's theorem applied on circle $\odot(NHYM)$ and $\odot(ABC)$, and pairs of parallel lines $\overline{HY}$ $\parallel$ $\overline{BP}$ and $\overline{HY}$ $\parallel$ $\overline{QC}$ $\implies$ $Y$ lies on $\overline{PD}$ and $\overline{QE}$, as desired.
\end{proof}

\begin{problem}[Iran 2015]
    Let $ABC$ be a triangle with orthocenter $H$ and circumcenter $O$. Let $K$ be the midpoint of $AH$. point $P$ lies on $AC$ such that $\angle BKP=90^{\circ}$. Prove that $OP\parallel BC$.
\end{problem}
\begin{figure}[h]
    \centering
    \begin{asy}
        import geometry;
        size(6cm); defaultpen(fontsize(10pt));

        pair A, B, C, H, O, K, D, E, X, Y, P;
        O = origin;

        A = dir(110); B = dir(210); C = dir(330);

        dot("$A$", A, dir(120)); dot("$B$", B, dir(210)); dot("$C$", C, dir(330));
        dot("$O$", O, dir(45)); draw(A--B--C--cycle); draw(circumcircle(A, B, C));

        D = foot(A, B, C); E = foot(B, A, C); 
        H = extension(A, D, B, E); K = (A + H) / 2;

        X = reflect(B, C) * H; Y = reflect(A, C) * H;
        dot("$X$", X, dir(225)); dot("$Y$", Y, dir(45)); dot("$H$", H, dir(135));
        dot("$D$", D, dir(225)); dot("$E$", E, dir(75)); draw(A--X); draw(B--Y);
        dot("$K$", K, dir(135)); draw(circumcircle(B, X, E), dashed);

        path C1 = circumcircle(B, X, E);
        pair[] PP = intersectionpoints(line(A, C), C1);

        dot("$P$", PP[0], dir(40)); draw(O--PP[0]);
        draw(B--K); draw(K--PP[0]); draw(X--PP[0]);
    \end{asy}
\end{figure}
\begin{proof}
    Let $D$ and $E$ be the foot of perpendicular from $A$ and $B$ onto $\overline{BC}$ and $\overline{AC}$, and $X$ and $Y$ be the reflections of $H$ over $\overline{BC}$ and $\overline{AC}$. Since $K$ and $E$ are the midpoints of $\overline{AH}$ and $\overline{HY}$, therefore by the midpoint theorem $\overline{KE}$ $\parallel$ $\overline{AY}$. Applying the converse of reim's theorem, we get that $BKEX$ is a cyclic quadrilateral. Since $\angle BKP$ $=$ $90^{\circ}$ and $\angle BEP$ $=$ $90^{\circ}$ $\implies$ $P$ lies on the circle $\odot(BKEX)$. Since $K$ is the center of $\odot(AEH)$, we have
    \begin{align*}
        \angle XAP = \angle KAE = \angle KEA = 180^{\circ} - \angle KEP = \angle KXP = \angle AXP        
    \end{align*}
    Therefore, $\triangle PAX$ is isosceles and hence $P$ lies on the perpendicular bisector of $\overline{AX}$. Since, $O$ lies on the perpendicular bisector of $\overline{AX}$ too $\implies$ $\overline{OP}$ $\perp$ $\overline{AX}$. But $\overline{AX}$ $\perp$ $\overline{BC}$ $\implies$ $\overline{OP}$ $\parallel$ $\overline{BC}$.
\end{proof}

\subsection{Exercises}
\begin{exercise}[Iran IMO TST 2008]
    In the triangle $ ABC$, $ \angle B$ is greater than $ \angle C$. Suppose $T$ is the midpoint of the arc $ BAC$ from the circumcircle of $ ABC$ and $ I$ is the incenter of $ ABC$. Let $E$ be a point such that $\angle AEI = 90^\circ$ and $ AE\parallel BC$. Let $\overline{TE}$ intersect the $\odot(ABC)$ for the second time in $P$. If $\angle B = \angle IPB$, find the angle $ \angle A$.
\end{exercise}

\begin{exercise}[IMO 2019]
    In triangle $ABC$, point $A_1$ lies on side $BC$ and point $B_1$ lies on side $AC$. Let $P$ and $Q$ be points on segments $AA_1$ and $BB_1$, respectively, such that $PQ$ is parallel to $AB$. Let $P_1$ be a point on line $PB_1$, such that $B_1$ lies strictly between $P$ and $P_1$, and $\angle PP_1C=\angle BAC$. Similarly, let $Q_1$ be the point on line $QA_1$, such that $A_1$ lies strictly between $Q$ and $Q_1$, and $\angle CQ_1Q=\angle CBA$. Prove that points $P$, $Q$, $P_1$ and $Q_1$ are concyclic.
\end{exercise}

\begin{exercise}[APMO 2022]
    Let $ABC$ be a right triangle with $\angle B=90^{\circ}$. Point $D$ lies on the line $CB$ such that $B$ is between $D$ and $C$. Let $E$ be the midpoint of $AD$ and let $F$ be the second intersection point of the circumcircle of $\triangle ACD$ and the circumcircle of $\triangle BDE$. Prove that as $D$ varies, the line $EF$ passes through a fixed point.
\end{exercise}

\begin{exercise}[APMO 2024]
    Let $ABC$ be an acute triangle. Let $D$ be a point on side $AB$ and $E$ be a point on side $AC$ such that lines $BC$ and $DE$ are parallel. Let $X$ be an interior point of $BCED$. Suppose rays $DX$ and $EX$ meet side $BC$ at points $P$ and $Q$, respectively, such that both $P$ and $Q$ lie between $B$ and $C$. Suppose that the circumcircles of triangles $BQX$ and $CPX$ intersect at a point $Y \neq X$. Prove that the points $A, X$, and $Y$ are collinear.
\end{exercise}

\section{Forgotten Coaxiality Lemma}
Another powerful result to show that a circle is coaxial with other two circles is the following.
\begin{theorem}[Forgotten Coaxiality Lemma]
    Given two circles $\omega_1$ and $\omega_2$ that intersect in $A$ and $B$. Then the locus of points $P$ such that 
    \begin{align*}
        \frac{\operatorname{Pow}_{\omega_1} (P)}{\operatorname{Pow}_{\omega_2} (P)} = k
    \end{align*}
    for a real constant $k$, is a circle that is passes through $A$ and $B$.
\end{theorem}
An equivalent way to state the above result is to consider the following situation
\begin{example}
    Given two circles $\omega_1$ and $\omega_2$ that intersect in $A$ and $B$. For some points $C$ and $D$ we would like to show that $A$, $B$, $C$ and $D$ are concyclic.
\end{example}
Using the forgotten coaxiality lemma, we can essentially ignore points $A$ and $B$ and the concyclicity would be immediately implied if
\begin{align*}
\frac{\operatorname{Pow}_{\omega_1} \left( C \right)}{\operatorname{Pow}_{\omega_2} \left( C \right)} = \frac{\operatorname{Pow}_{\omega_1} \left( D \right)}{\operatorname{Pow}_{\omega_2} \left( D \right)}
\end{align*}
\begin{figure}[h]
    \centering
    \begin{asy}
        import geometry;
        size(8cm); defaultpen(fontsize(10pt));

        pair A, B, C, D, X, Y;

        B = (0, 0); A = (0, 3); X = (-5, 0); Y = (4, 0);
        draw(circumcircle(A, B, X));
        draw(circumcircle(A, B, Y));

        C = (2, 0); D = (2, 3);
        draw(circumcircle(A, B, C), heavygray+dashed);

        dot("$A$", A, dir(170));
        dot("$B$", B, dir(190));
        dot("$C$", C, dir(320));
        dot("$D$", D, dir(50));
    \end{asy}
\end{figure}
\begin{proof}
Suppose $AD$ intersects $\omega_1$ and $\omega_2$ at $D_1$ and $D_2$, and $BC$ intersects $\omega_1$ and $\omega_2$ at $C_1$ and $C_2$. Therefore,
\begin{align*}
    \frac{\operatorname{Pow}_{\omega_1} \left( C \right)}{\operatorname{Pow}_{\omega_2} \left( C \right)} = \frac{\overline{CB} \cdot \overline{CC_1}}{\overline{CB} \cdot \overline{CC_2}} = \frac{\overline{CC_1}}{\overline{CC_2}} 
\end{align*}
Similarly,
\begin{align*}
    \frac{\operatorname{Pow}_{\omega_1} \left( D \right)}{\operatorname{Pow}_{\omega_2} \left( D \right)} = \frac{\overline{DA} \cdot \overline{DD_1}}{\overline{DA} \cdot \overline{DD_2}} = \frac{\overline{DD_1}}{\overline{DD_2}} 
\end{align*}
However by reim's theorem, $\overline{C_1D_1} \parallel \overline{CD} \parallel \overline{C_2D_2}$. Hence,
\begin{align*}
\frac{\overline{CC_1}}{\overline{CC_2}} &= \frac{\overline{DD_1}}{\overline{DD_2}} \\
\end{align*}
which proves that,
\begin{align*}
    \frac{\operatorname{Pow}_{\omega_1} \left( C \right)}{\operatorname{Pow}_{\omega_2} \left( C \right)} &= \frac{\operatorname{Pow}_{\omega_1} \left( D \right)}{\operatorname{Pow}_{\omega_2} \left( D \right)} 
\end{align*}
\end{proof}

\subsection{Examples}
\begin{problem}[CGMO 2017]
    Let the $ABCD$ be a cyclic quadrilateral with circumcircle $\omega_1$. Lines $AC$ and $BD$ intersect at point $E$, and lines $AD$, $BC$ intersect at point $F$. Circle $\omega_2$ is tangent to segments $EB$, $EC$ at points $M$, $N$ respectively, and intersects with circle $\omega_1$ at points $Q$, $R$. Lines $BC$, $AD$ intersect line $MN$ at $S$, $T$ respectively. Show that $Q$, $R$, $S$, $T$ are concyclic.
\end{problem}
\begin{figure}[h]
    \centering
    \begin{asy}
        import geometry;
        size(11cm); defaultpen(fontsize(11pt));pen med=mediummagenta;pen light=pink;pen deep=deepmagenta;pen org=magenta;pen heavy=heavymagenta;

        pair O,A,B,C,D,E,F,G,M,N,S,T,Q,R;
        O=(0,0); D=dir(15); A=dir(110); B=dir(205); C=dir(330); 
        path w=circumcircle(A,B,C);
        E=extension(A,C,B,D);
        F=extension(A,D,B,C);
        G=2.2*incenter(B,E,C)-1.2*E;
        N=foot(G,E,C);
        M=foot(G,E,B);
        S=extension(B,C,N,M);
        T=extension(A,D,N,M);
        Q=intersectionpoints(circle(G,abs(G-M)),w)[0];
        R=intersectionpoints(circle(G,abs(G-M)),w)[1];

        draw(w);
        draw(A--B--C--D--cycle);
        draw(circle(G,abs(G-M)));
        draw(circumcircle(Q,R,S), heavygray+dashed);
        draw(D--T);draw(C--F);
        draw(M--T);
        draw(A--C);
        draw(B--D);

        clip((-1,1)--(3.5,1)--(3.5,-1.5)--(-1,-1.5)--cycle);

        dot("$B$",B,dir(B));
        dot("$C$",C,dir(300));
        dot("$A$",A,dir(A));
        dot("$D$",D,dir(55));
        dot("$E$",E,dir(80));
        dot("$F$",F,dir(70));
        dot("$M$",M,dir(140));
        dot("$N$",N,dir(60));
        dot("$S$",S,dir(270));
        dot("$T$",T,dir(60));
        dot("$Q$",Q,dir(Q));
        dot("$R$",R,dir(290));
    \end{asy}
\end{figure}
\begin{proof}
    We would like to show that $QRST$ is a cyclic quadrilateral. Since $\overline{QR}$ is the radical axis of $\odot(ABCD)$ and $\odot(MNQR)$ $\implies$ we need to show $QRST$ is coaxial with these two circles. Applying the forgotten coaxiality lemma, we want to show
    \begin{align*}
        \frac{\operatorname{Pow}_{\omega_1}(S)}{\operatorname{Pow}_{\omega_2}(S)} &= \frac{\operatorname{Pow}_{\omega_1}(T)}{\operatorname{Pow}_{\omega_2}(T)} \\ 
        \Longleftrightarrow \frac{\overline{SB} \cdot \overline{SC}}{\overline{SM} \cdot \overline{SN}} &= \frac{\overline{TA} \cdot \overline{TD}}{\overline{TN} \cdot \overline{TM}}
    \end{align*}    
    Since $\triangle EMN$ is isosceles $\implies$ $\angle CNS$ $=$ $\angle ENM$ $=$ $\angle DMT$. And, $\angle ACB$ $=$ $\angle ADB$ $\implies$ $\angle NCS$ $=$ $\angle MDT$. Therefore by AA similarity criterion, we have $\triangle NCS$ $\sim$ $\triangle MDT$. Similarly, $\triangle BMS$ $\sim$ $\triangle ANT$. These two similar triangles imply
    \begin{align*}
        \frac{\overline{SC}}{\overline{SN}} = \frac{\overline{TD}}{\overline{TM}} \,\,\,\, \& \,\,\,\, \frac{\overline{SB}}{\overline{SM}} = \frac{\overline{TA}}{\overline{TN}}
    \end{align*}
    Multiplying these two equalities we get the desired result 
    \begin{align*}
        \frac{\overline{SB} \cdot \overline{SC}}{\overline{SM} \cdot \overline{SN}} = \frac{\overline{TA} \cdot \overline{TD}}{\overline{TN} \cdot \overline{TM}} \implies \frac{\operatorname{Pow}_{\omega_1}(S)}{\operatorname{Pow}_{\omega_2}(S)} = \frac{\operatorname{Pow}_{\omega_1}(T)}{\operatorname{Pow}_{\omega_2}(T)}
    \end{align*}
    Therefore, $QRST$ is coaxial with $\omega_1$ and $\omega_2$ $\implies$ $QRST$ is cyclic.
\end{proof}

\begin{problem}[IMO Shortlist 2005]
    Let $\triangle ABC$ be an acute-angled triangle with $AB \not= AC$. Let $H$ be the orthocenter of triangle $ABC$, and let $M$ be the midpoint of the side $BC$. Let $D$ be a point on the side $AB$ and $E$ a point on the side $AC$ such that $AE=AD$ and the points $D$, $H$, $E$ are on the same line. Prove that the line $HM$ is perpendicular to the common chord of the circumscribed circles of triangle $\triangle ABC$ and triangle $\triangle ADE$.
\end{problem}
\begin{figure}[h]
\centering
\begin{asy}
    import geometry;
    size(7cm); defaultpen(fontsize(10pt));

    pair A, B, C, H, D, E, X, Y, M;
    A = dir(110);
    B = dir(200);
    C = dir(340);
    H = A+B+C;
    X = foot(B, A, C);
    Y = foot(C, A, B);
    D = extension(A, B, H, incenter(H,B,Y));
    E = extension(A, C, H, incenter(H,C,X));
    M = (B + C) / 2;
    
    draw(A--B--C--cycle^^D--E^^B--X^^C--Y);
    draw(circumcircle(A, D, E)^^circumcircle(A, B, C));
    draw(circumcircle(A, X, Y), dashed);

    pair[] QQ = intersectionpoints(line(H, M), circumcircle(A, B, C));
    
    dot("$A$", A, dir(90));
    dot("$B$", B, dir(225));
    dot("$C$", C, dir(315));
    dot("$D$", D, dir(200));
    dot("$E$", E, dir(30));
    dot("$H$", H, dir(270));
    dot("$X$", X, dir(0));
    dot("$Y$", Y, dir(210));
    dot("$M$", M, dir(310));
    dot("$Q$", QQ[1], dir(140));
    draw(M--QQ[1]); draw(A--QQ[1]); 
    markrightangle(M, QQ[1], A, 7); markrightangle(B, X, C, 7); markrightangle(B, Y, C, 7);
\end{asy}
\end{figure}
\begin{proof}
    It's well known that if the ray $MH$ intersects $\odot(ABC)$ at $Q$, then $\overline{AQ}$ $\perp$ $\overline{HM}$. This suggests us that $Q$ must lie on $\odot(ADE)$. To show that $ADEQ$ is cyclic, we only need to show 
    \begin{align*}
        \frac{\operatorname{Pow}_{\odot(ABC)}(D)}{\operatorname{Pow}_{\odot(AH)}(D)} = \frac{\operatorname{Pow}_{\odot(ABC)}(E)}{\operatorname{Pow}_{\odot(AH)}(E)}
    \end{align*}
    Suppose $X$ and $Y$ are the feet of perpendicular from $B$ and $C$ onto $\overline{AC}$ and $\overline{AB}$. Then,
    \begin{align*}
        \frac{\operatorname{Pow}_{\odot(ABC)}(D)}{\operatorname{Pow}_{\odot(AH)}(D)} &= \frac{\operatorname{Pow}_{\odot(ABC)}(E)}{\operatorname{Pow}_{\odot(AH)}(E)} \\ 
        \Longleftrightarrow \frac{\overline{DA} \cdot \overline{DB}}{\overline{DY} \cdot \overline{DA}} &= \frac{\overline{EA} \cdot \overline{EC}}{\overline{EX} \cdot \overline{EA}} \\ 
        \Longleftrightarrow \frac{\overline{DB}}{\overline{DY}} &= \frac{\overline{EC}}{\overline{EX}}
    \end{align*}
    Since, $\angle YHB$ $=$ $\angle BAC$ and $\angle YHD$ $=$ $90^{\circ} - \angle YDH$ $=$ $\tfrac{1}{2} \angle BAC$. Therefore, $\overline{DH}$ is the angle bisector of $\angle YHB$ and similarly $\overline{HE}$ is the angle bisector of $\angle CHX$. Using the fact that $BYXC$ is a cyclic quadrilateral, we can write
    \begin{align*}
        \overline{HY} \cdot \overline{HC} &= \overline{HX} \cdot \overline{HB} \\ 
        \implies \frac{\overline{HY}}{\overline{HB}} &= \frac{\overline{HX}}{\overline{HC}} \\ 
        \implies \frac{\overline{DY}}{\overline{DB}} &= \frac{\overline{EX}}{\overline{EC}}
    \end{align*}
    Due to the forgotten coaxiality lemma, we have $AQDE$ is a cyclic quadrilateral $\implies$ $\overline{HM}$ is perpendicular to the radical axis of $\odot(ABC)$ and $\odot(ADE)$.
\end{proof}

\subsection{Exercises}
\begin{exercise}
    In triangle $\triangle ABC$, let $E$ and $F$ be points on sides $AC$ and $AB$, respectively, such that $BFEC$ is cyclic. Let lines $BE$ and $CF$ intersect at point $P$, and $M$ and $N$ be the midpoints of $\overline{BF}$ and $\overline{CE}$, respectively. If $U$ is the foot of the perpendicular from $P$ to $BC$, and the circumcircles of triangles $\triangle BMU$ and $\triangle CNU$ intersect at second point $V$ different from $U$, prove that $A, P,$ and $V$ are collinear.
\end{exercise}

\begin{exercise}[European Mathematical Cup 2016]
    Let $C_{1}$, $C_{2}$ be circles intersecting in $X$, $Y$ . Let $A$, $D$ be points on $C_{1}$ and $B$, $C$ on $C_2$ such that $A$, $X$, $C$ are collinear and $D$, $X$, $B$ are collinear. The tangent to circle $C_{1}$ at $D$ intersects $BC$ and the tangent to $C_{2}$ at $B$ in $P$, $R$ respectively. The tangent to $C_2$ at $C$ intersects $AD$ and tangent to $C_1$ at $A$, in $Q$, $S$ respectively. Let $W$ be the intersection of $AD$ with the tangent to $C_{2}$ at $B$ and $Z$ the intersection of $BC$ with the tangent to $C_1$ at $A$. Prove that the circumcircles of triangles $YWZ$, $RSY$ and $PQY$ have two points in common, or are tangent in the same point.
\end{exercise}

\section{Practice Problems}
\begin{exercise}[IMO Shortlist 2019]
    Let $ABC$ be a triangle. Circle $\Gamma$ passes through $A$, meets segments $AB$ and $AC$ again at points $D$ and $E$ respectively, and intersects segment $BC$ at $F$ and $G$ such that $F$ lies between $B$ and $G$. The tangent to circle $BDF$ at $F$ and the tangent to circle $CEG$ at $G$ meet at point $T$. Suppose that points $A$ and $T$ are distinct. Prove that line $AT$ is parallel to $BC$.
\end{exercise}

\begin{exercise}[Iran 2021]
    Circle $\omega$ is inscribed in quadrilateral $ABCD$ and is tangent to segments $BC, AD$ at $E,F$ , respectively.$DE$ intersects $\omega$ for the second time at $X$. if the circumcircle of triangle $DFX$ is tangent to lines $AB$ and $CD$ , prove that quadrilateral $AFXC$ is cyclic.
\end{exercise}

\begin{exercise}[Romania TST 2017]
    Let $ABCD$ be a trapezium, $AD \parallel BC$, and let $E$ and $F$ be points on the sides $AB$ and $CD$, respectively. The circumcircle of $AEF$ meets $AD$ again at $A_1$, and the circumcircle of $CEF$ meets $BC$ again at $C_1$. Prove that $A_1C_1$, $BD$ and $EF$ are concurrent.
\end{exercise}

\begin{exercise}[IMO Shortlist 2017]
    Let $ABCC_1B_1A_1$ be a convex hexagon such that $AB=BC$, and suppose that the line segments $AA_1, BB_1$, and $CC_1$ have the same perpendicular bisector. Let the diagonals $AC_1$ and $A_1C$ meet at $D$, and denote by $\omega$ the circle $ABC$. Let $\omega$ intersect the circle $A_1BC_1$ again at $E \neq B$. Prove that the lines $BB_1$ and $DE$ intersect on $\omega$.
\end{exercise}

\begin{exercise}[IMO Shortlist 2012]
    Let $ABC$ be a triangle with circumcircle $\omega$ and $\ell$ a line without common points with $\omega$. Denote by $P$ the foot of the perpendicular from the center of $\omega$ to $\ell$. The side-lines $BC,CA,AB$ intersect $\ell$ at the points $X,Y,Z$ different from $P$. Prove that the circumcircles of the triangles $AXP$, $BYP$ and $CZP$ have a common point different from $P$ or are mutually tangent at $P$.
\end{exercise}

\begin{exercise}[IMO Shortlist 2017]
    A convex quadrilateral $ABCD$ has an inscribed circle with center $I$. Let $I_a, I_b, I_c$ and $I_d$ be the incenters of the triangles $DAB, ABC, BCD$ and $CDA$, respectively. Suppose that the common external tangents of the circles $AI_bI_d$ and $CI_bI_d$ meet at $X$, and the common external tangents of the circles $BI_aI_c$ and $DI_aI_c$ meet at $Y$. Prove that $\angle{XIY}=90^{\circ}$.
\end{exercise}

\end{document}