\documentclass[11pt]{scrartcl}
\let\captionof\undefined
\usepackage[sexy,von]{evan}
\usepackage{wrapfig}
% \renewcommand{\vonenvname}{example}
\lstset{basicstyle=\small\ttfamily,
  numbers=left,
  numbersep=5pt,
  numberstyle=\tiny,
  keywordstyle=\bfseries,
  showstringspaces=false,
  tabsize=4,
  frame=single,
  keywordstyle=\bfseries\color{blue},
  commentstyle=\color{green!70!black},
  identifierstyle=\color{green!20!black},
  stringstyle=\color{orange},
  breaklines=true,
  breakatwhitespace=true,
  frame=none
}

\usepackage{xcolor}
\setkomafont{captionlabel}{\bfseries\color{red}}
\renewcommand*{\figurename}{Fig}

\usepackage{answers}
\usepackage{cancel}
\usepackage{asymptote}

\begin{document}
\title{Lecture Notes (25th Jan, 2026)}
\date{\today}
\maketitle

\begin{abstract}
    \centering In this lecture, we study the properties of tangent circles.
\end{abstract}

\section{Internally Tangent Circles}
We would like to study about configurations that deal with tangent circles. When we say two circles are tangent, it means that these circles only have one point in common. A pair of circles could be either \vocab{internally} tangent or \vocab{externally} tangent.
\begin{definition}
  A pair of circles $\Gamma$ and $\omega$ are
  \begin{enumerate}
    \item \highlight{internally tangent}, if and only if $\Gamma$ and $\omega$ share a single point and the center of the smaller circle is contained inside the larger circle.
    \item \highlight{externally tangent}, if and only if $\Gamma$ and $\omega$ share a single point and the center of either of the circles lie outside the other circle. 
  \end{enumerate}
\end{definition}

We would like to focus on pair of circles that are internally tangent for now and study their properties. Let's start with the simplest configuration with two internaly tangent circles.

\begin{figure}[h]
  \centering
  \begin{asy}
    import geometry;
    size(5cm); defaultpen(fontsize(10pt));

    pair O1, O2;
    real r, R;
    R = 5; r = 1.5;
  
    O1 = origin; O2 = O1 + (R - r) * dir(60);
    draw(circle(O1, R));
    draw(circle(O2, r));

    dot("$O_1$", O1, dir(135));
    dot("$O_2$", O2, dir(135));
  \end{asy}
\end{figure}

Let $\Gamma$ and $\omega$ be two circles centered at $O_1$ and $O_2$, with radius $R$ and $r$, where $R > r$. Suppose that $\Gamma$ and $\omega$ are internally tangent to each other.
\subsection{Homothetic Mapping}
\begin{proposition}
  Suppose $T$ is the internal tangency point of $\Gamma$ and $\omega$, then the points $O_1$, $O_2$ and $T$ are collinear.
\end{proposition}
\begin{figure}[h]
  \centering
  \begin{asy}
    import geometry;
    size(5cm); defaultpen(fontsize(10pt));

    pair O1, O2, T;
    real r, R;
    R = 5; r = 1.5;
  
    O1 = origin; O2 = O1 + (R - r) * dir(60); T = O2 + r * dir(60);
    draw(circle(O1, R));
    draw(circle(O2, r));

    dot("$O_1$", O1, dir(135));
    dot("$O_2$", O2, dir(135));
    dot("$T$", T, dir(45));

    draw(O1--T, heavygray+dashed);
  \end{asy}
\end{figure}
\begin{proof}
  Consider a homothety at point $T$ that maps $\omega$ to $\Gamma$. Observe that under this homothetic transformation, we map the point $O_2$ to $O_1$. Hence, the points $O_1$, $O_2$ and $T$ must be collinear.
\end{proof}

As a consequence of the homothetic mapping, we have the following result

\begin{proposition}
    Choose a point $B$ on $\omega$. Suppose $TB$ meets $\Gamma$ again at $A$. Then $\overline{O_1A} \parallel \overline{O_2B}$.
\end{proposition}
\begin{figure}[h]
  \centering
  \begin{asy}
    import geometry;
    size(5cm); defaultpen(fontsize(10pt));

    pair O1, O2, T;
    real r, R;
    R = 5; r = 1.5;
  
    O1 = origin; O2 = O1 + (R - r) * dir(60); T = O2 + r * dir(60);
    draw(circle(O1, R));
    draw(circle(O2, r));

    dot("$O_1$", O1, dir(135));
    dot("$O_2$", O2, dir(135));
    dot("$T$", T, dir(45));

    draw(O1--T);

    pair A, B;
    B = O2 + r * dir(290);
    A = O1 + R * dir(290);

    dot("$A$", A, dir(320));
    dot("$B$", B, dir(320));
    draw(A--T); draw(A--O1); draw(B--O2);
  \end{asy}
\end{figure}
\begin{proof}
  Consider a homothety at point $T$ that maps $\omega$ to $\Gamma$. Under this homothety, the point $B$ is mapped to $A$. Hence, $\triangle TO_2B$ $\sim$ $\triangle TO_1A$ $\implies$ $\angle TBO_2$ $=$ $\angle TAO_1$ $\implies$ $\overline{O_1A}$ $\parallel$ $\overline{O_2B}$. 
\end{proof}

\begin{corollary}
  Let $T$ be the point of internal tangency of two circles $\Gamma$ and $\omega$ with radii $R$ and $r$, where $R$ $>$ $r$. Choose a point $B$ on $\omega$ and let $TB$ meet $\Gamma$ at $A$. Then
  \begin{align*}
    \frac{\overline{TB}}{\overline{TA}} = \frac{r}{R}
  \end{align*}
\end{corollary}

The proof for the above result immediately follows from the fact that $\triangle TO_2B$ $\sim$ $\triangle TO_1A$. Now, we move onto a very celebrated result by Archimedes which appears very frequently in geometry configurations.

\subsection{Archimedes' Lemma}

\begin{theorem}[Archimedes' Lemma]
  Let $\Gamma$ and $\omega$ be two circles centered at $O_1$ and $O_2$. Suppose that these circles are internally tangent at the point $T$. Let $\overline{XY}$ be the chord of $\Gamma$ such that $\overline{XY}$ is tangent to $\omega$ at point $B$. Let $A$ be the midpoint of the arc $XY$ that does not contain $T$. Then
  \begin{enumerate}[itemsep=0.01em]
    \item points $T$, $B$ and $A$ are collinear.
    \item $\overline{AB} \cdot \overline{AT} = \overline{AX}^2 = \overline{AY}^2$
  \end{enumerate}
\end{theorem}

\begin{figure}[h]
  \centering
  \begin{asy}
    import geometry;
    size(6cm); defaultpen(fontsize(10pt));

    pair O1, O2, T;
    real r, R;
    R = 5; r = 1.5;
  
    O1 = origin; O2 = O1 + (R - r) * dir(60); T = O2 + r * dir(60);
    draw(circle(O1, R));
    draw(circle(O2, r));

    dot("$O_1$", O1, dir(210));
    dot("$O_2$", O2, dir(135));
    dot("$T$", T, dir(45));

    draw(O1--T);

    pair A, B;
    B = O2 + r * dir(290);
    A = O1 + R * dir(290);

    dot("$A$", A, dir(320));
    dot("$B$", B, dir(320));
    draw(A--T, heavygray+dashed); draw(A--O1); draw(B--O2);

    pair P1 = B - rotate(90) * (O2 - B);
    pair P2 = B + rotate(90) * (O2 - B);
    pair[] XX = intersectionpoints(line(P1, P2), circle(O1, R));
    dot("$X$", XX[1], dir(210));
    dot("$Y$", XX[0], dir(40));

    draw(XX[0]--XX[1]);
    markrightangle(XX[0], B, O2, 7);

    draw(A--XX[0]); draw(A--XX[1]);
  \end{asy}
\end{figure}
\begin{proof}
  Consider a homothety at point $T$ that sends $\omega$ to $\Gamma$. Suppose that this homothety sends point $B$ to $A'$, where $A'$ lies on $\Gamma$. Since $\overline{XY}$ is tangent to $\omega$ at $B$, therefore this homothety maps $XY$ to a line $\ell$ passing through $A'$ that is tangent to $\Gamma$. Since $\ell$ is the image under a homothetic transformation of $XY$ $\implies$ $\overline{XY}$ $\parallel$ $\ell$. Therefore,
  \begin{align*}
  \angle YXA' = \angle \left( \overline{A'X}, \ell \right) = \angle XYA'
  \end{align*} 
  This implies that $\triangle XYA'$ is isosceles $\implies$ $A'$ is the midpoint of the arc $XY$ not containing $T$. Thus $A \equiv A'$, proving that $T$, $B$ and $A$ are collinear. 

  For the second part, we shall show that $\overline{AX}$ is tangent to $\odot(TBX)$ at $X$. This is easy to establish since, 
  \begin{align*}
    \angle XTB = \angle XTA = \angle XYA = \angle AXY = \angle AXB
  \end{align*}
  Using the power of a point theorem, we get that 
  \begin{align*}
    \overline{AX}^2 = \overline{AB} \cdot \overline{AT}
  \end{align*}
  Since $\overline{AX} = \overline{AY}$, which implies the relation.
\end{proof}

\begin{exercise}[Russia 2001]
  In the above configuration, show that the circumradius of $\triangle ABY$ is a constant that does not depend upon the position of point $B$.
\end{exercise}

Let's look at some nice results that revolve around the \vocab{Archimedes' Lemma}.
\subsection{Examples}
\begin{problem}
  Let $h$ be a semicircle with diameter $AB$. The two circles $k_1$ and $k_2$, $k_1 \ne k_2$, touch the segment $AB$ at the points $C$ and $D$, respectively, and the semicircle $h$ fom the inside at the points $E$ and $F$, respectively. Prove that the four points $C$, $D$, $E$ and $F$ lie on a circle.
\end{problem}
\begin{figure}[h]
  \centering
  \begin{asy}
    import geometry;
    size(8cm); defaultpen(fontsize(10pt));

    pair O1, O2, O;
    real R, r1, r2;
    R = 4; r1 = R / 3; r2 = R / 2.414;

    O = origin; O1 = O + (R - r1) * dir(150); O2 = O + (R - r2) * dir(45);
    dot("$O$", O, dir(225)); dot("$O_1$", O1, dir(60)); dot("$O_2$", O2, dir(130));
    
    draw(circle(O, R));
    draw(circle(O1, r1)); draw(circle(O2, r2));

    pair A, B; A = (-4, 0); B = (4, 0);
    dot("$A$", A, dir(180)); dot("$B$", B, dir(0)); draw(A--B);

    pair C, D; 
    C = O1 + r1 * dir(270); D = O2 + r2 * dir(270);

    pair E, F;
    E = O1 + r1 * dir(150); F = O2 + r2 * dir(45);

    dot("$C$", C, dir(225)); dot("$D$", D, dir(315));
    dot("$E$", E, dir(135)); dot("$F$", F, dir(45));

    draw(E--O, heavygray); draw(F--O, heavygray);

    pair X = O + R * dir(270);
    dot("$X$", X, dir(225));

    draw(E--X, gray); draw(F--X, gray);
    draw(circumcircle(E, C, D), heavygray+dashed); draw(O--X); markrightangle(X, O, B, 7);
    draw(O1--C); draw(O2--D); markrightangle(O, C, O1, 6); markrightangle(O2, D, O, 6);
  \end{asy} 
\end{figure}
\begin{proof}
Suppose $X$ is the midpoint of arc $AB$ not containing $E$. Then $X$ lies on lines $EC$ and $FD$ by archimedes' lemma. Since,
\begin{align*}
  \overline{XC} \cdot \overline{XE} = \overline{AX}^2 = \overline{XD} \cdot \overline{XF} 
\end{align*}
Therefore, by the converse of power of a point theorem $\implies$ points $C$, $D$, $E$ and $F$ lie on a circle. 
\end{proof}

\subsection{Exercises}
\begin{exercise}[INMO 2019]
  Let $AB$ be the diameter of a circle $\Gamma$ and let $C$ be a point on $\Gamma$ different from $A$ and $B$. Let $D$ be the foot of perpendicular from $C$ on to $AB$.Let $K$ be a point on the segment $CD$ such that $AC$ is equal to the semi perimeter of $ADK$.Show that the excircle of $ADK$ opposite $A$ is tangent to $\Gamma$.
\end{exercise}

\begin{exercise}[RMO 2019]
  Given a circle $\tau$, let $P$ be a point in its interior, and let $l$ be a line through $P$. Construct with proof using ruler and compass, all circles which pass through $P$, are tangent to $\tau$ and whose center lies on line $l$.
\end{exercise}

\begin{exercise}[RMO 2017]
  Let \(\Omega\) be a circle with a chord \(AB\) which is not a diameter. \(\Gamma_{1}\) be a circle on one side of \(AB\) such that it is tangent to \(AB\) at \(C\) and internally tangent to \(\Omega\) at \(D\). Likewise, let \(\Gamma_{2}\) be a circle on the other side of \(AB\) such that it is tangent to \(AB\) at \(E\) and internally tangent to \(\Omega\) at \(F\). Suppose the line \(DC\) intersects \(\Omega\) at \(X \neq D\) and the line \(FE\) intersects \(\Omega\) at \(Y \neq F\). Prove that \(XY\) is a diameter of \(\Omega\) .
\end{exercise}

\section{Curvilinear Incircles}

Let's move to something more complicated and miraculous. 
\begin{definition}
  Given $\triangle ABC$ and a point $D$ on $\overline{BC}$, a circle $\omega$ is called the \vocab{curvilinear incircle} of $\triangle ABC$ if $\omega$ is tangent to sides $\overline{AD}$ and $\overline{BC}$, and is internally tangent to $\odot(ABC)$. 
\end{definition}
Curvilinear incircles are a natural extension of the archimedes' lemma. Essentially, we are choosing another point on the outer circle and adding more tangents.

\begin{figure}[h]
  \centering
  \begin{asy}
    import geometry;
    size(6cm); defaultpen(fontsize(10pt));

    pair O, O1;
    real R, r;
    R = 4; r = 2.2;

    O = origin; O1 = O + (R - r) * dir(45);
    draw(circle(O, R)); draw(circle(O1, r));

    pair K = O1 + r * dir(270); dot("$K$", K, dir(305));
    pair P1 = K - rotate(90) * (O1 - K);
    pair P2 = K + rotate(90) * (O1 - K);

    pair[] AA = intersectionpoints(line(P1, P2), circle(O, R));
    dot("$B$", AA[0], dir(225)); dot("$C$", AA[1], dir(315));
    
    pair A = R * dir(98); dot("$A$", A, dir(100));
    pair T = O1 + r * dir(45); dot("$T$", T, dir(45));
    pair M = O + R * dir(270); 

    pair I = incenter(A, AA[0], AA[1]);

    pair[] LL = intersectionpoints(line(K, I), circle(O1, r));
    dot("$L$", LL[0], dir(150));

    pair D = extension(A, LL[0], AA[0], AA[1]);
    dot("$D$", D, dir(225)); 
    
    draw(A--AA[0]--AA[1]--cycle); draw(A--D);
  \end{asy}
\end{figure}

Let's look at some properties of the \highlight{curvilinear incircles}.

\subsection{More Circles!}
\begin{proposition}
  Given $\triangle ABC$ and a point $D$ on the $\overline{BC}$. Suppose $\omega$ is the curvilinear incircle of $\triangle ABC$ tangent to $\overline{AD}$ and $\overline{BC}$ at $L$ and $K$, and tangent to $\odot(ABC)$ at $T$. Let $M$ be the midpoint of arc $BC$ not containing $A$. Suppose $\overline{AM}$ intersects $\overline{KL}$ at $X$. Then the points $A$, $L$, $X$ and $T$ are concyclic. 
\end{proposition}

\begin{figure}[h]
  \centering
  \begin{asy}
    import geometry;
    size(6cm); defaultpen(fontsize(10pt));

    pair O, O1;
    real R, r;
    R = 4; r = 2.2;

    O = origin; O1 = O + (R - r) * dir(45);
    draw(circle(O, R)); draw(circle(O1, r));

    pair K = O1 + r * dir(270); dot("$K$", K, dir(305));
    pair P1 = K - rotate(90) * (O1 - K);
    pair P2 = K + rotate(90) * (O1 - K);

    pair[] AA = intersectionpoints(line(P1, P2), circle(O, R));
    dot("$B$", AA[0], dir(225)); dot("$C$", AA[1], dir(315));
    
    pair A = R * dir(98); dot("$A$", A, dir(100));
    pair T = O1 + r * dir(45); dot("$T$", T, dir(45));
    pair M = O + R * dir(270); dot("$M$", M, dir(225)); 

    pair I = incenter(A, AA[0], AA[1]);
    dot("$X$", I, dir(50));

    pair[] LL = intersectionpoints(line(K, I), circle(O1, r));
    dot("$L$", LL[0], dir(150));

    pair D = extension(A, LL[0], AA[0], AA[1]);
    dot("$D$", D, dir(225)); 
    
    draw(M--T);
    draw(A--AA[0]--AA[1]--cycle); 
    draw(A--M, heavygray);
    draw(A--D); 
    draw(K--LL[0], heavygray);

    draw(circumcircle(A, T, I), heavygray+dashed);
  \end{asy}
\end{figure}
\begin{proof}
  By archimedes' lemma, we know that $M$ lies on $\overline{KT}$. To show that the points $A$, $L$, $X$ and $T$ are concyclic, we just need to angle chase
  \begin{align*}
    \angle XLT = \angle KLT = \angle CKT = \angle MBT = \angle MAT = \angle XAT
  \end{align*}
  where, $\angle CKT = \angle MBT$ holds because, $\overline{TM}$ is the angle bisector of $\angle BTC$ $\implies$ $\triangle TBM$ $\sim$ $\triangle TKC$. Therefore, $\angle XLT$ $=$ $\angle XAT$, which implies that the four points are concyclic.
\end{proof}

\begin{proposition}
  Given $\triangle ABC$ and a point $D$ on the $\overline{BC}$. Suppose $\omega$ is the curvilinear incircle of $\triangle ABC$ tangent to $\overline{AD}$ and $\overline{BC}$ at $L$ and $K$, and tangent to $\odot(ABC)$ at $T$. Let $M$ be the midpoint of arc $BC$ not containing $A$. Suppose $\overline{AM}$ intersects $\overline{KL}$ at $X$. Then, $\overline{MX}$ is tangent to $\odot(XKT)$ at point $X$.
\end{proposition}

\begin{figure}[h]
  \centering
  \begin{asy}
    import geometry;
    size(6cm); defaultpen(fontsize(10pt));

    pair O, O1;
    real R, r;
    R = 4; r = 2.2;

    O = origin; O1 = O + (R - r) * dir(45);
    draw(circle(O, R)); draw(circle(O1, r));

    pair K = O1 + r * dir(270); dot("$K$", K, dir(305));
    pair P1 = K - rotate(90) * (O1 - K);
    pair P2 = K + rotate(90) * (O1 - K);

    pair[] AA = intersectionpoints(line(P1, P2), circle(O, R));
    dot("$B$", AA[0], dir(225)); dot("$C$", AA[1], dir(315));
    
    pair A = R * dir(98); dot("$A$", A, dir(100));
    pair T = O1 + r * dir(45); dot("$T$", T, dir(45));
    pair M = O + R * dir(270); dot("$M$", M, dir(225)); 

    pair I = incenter(A, AA[0], AA[1]);
    dot("$X$", I, dir(50));

    pair[] LL = intersectionpoints(line(K, I), circle(O1, r));
    dot("$L$", LL[0], dir(150));

    pair D = extension(A, LL[0], AA[0], AA[1]);
    dot("$D$", D, dir(225)); 
    
    draw(M--T);
    draw(A--AA[0]--AA[1]--cycle);
    draw(A--M);
    draw(A--D);
    draw(K--LL[0]);

    draw(circumcircle(A, T, I), gray);
    draw(circumcircle(K, I, T), heavygray+dashed);
  \end{asy}
\end{figure}
\begin{proof}
  Effectively, we just want to show that $\angle MXK$ $=$ $\angle MTX$. Fortunately, this is just straightforward angle chasing
  \begin{align*}
    \angle MTX = \angle MTL - \angle XTL = \angle DLK - \angle XAL = \angle AXL = \angle MKX
  \end{align*}
  which proves that $\overline{MX}$ is tangent to $\odot(XKT)$ at $X$.
\end{proof}

\subsection{Introducing the Incenter}
\begin{proposition}[Sawayama's Theorem]
  Show that the point $X$ is the \vocab{Incenter} of $\triangle ABC$.
\end{proposition}
\begin{proof}
Observe that $\overline{MB}$ $=$ $\overline{MC}$ $\implies$ $\overline{AM}$ is the angle bisector of $\angle BAC$. Since,
\begin{align*}
  \overline{MX}^2 = \overline{MK} \cdot \overline{MT} = \overline{MB}^2 = \overline{MC}^2
\end{align*}
Hence by the incenter/excenter lemma, we get that $X$ is the incenter of $\triangle ABC$.
\end{proof}

It's very surprising how the incenter appears in this configuration. Something even more interesting occurs when we add the other curvilinear incircle of the cevian $\overline{AD}$ to the diagram, which leads to a reknowned result by \emph{Victor Thébault}.

\subsection{Thébault's Theorem}
\begin{theorem}[Thébault's Theorem]
  Given $\triangle ABC$ and a point $D$ on $\overline{BC}$, let $\omega_1$ and $\omega_2$ be the two curvilinear incircles of $\triangle ABC$ tangent to the cevian $\overline{AD}$. Suppose $O_1$ and $O_2$ are the centers of the two curvilinear incircles and $I$ is the incenter of $\triangle ABC$, then points $O_1$, $I$ and $O_2$ are collinear.
\end{theorem}
\begin{figure}[h]
  \centering
  \begin{asy}
    import geometry;
    size(8cm); defaultpen(fontsize(10pt));

    pair O, O1;
    real R, r;
    R = 4; r = 2.2;

    O = origin; O1 = O + (R - r) * dir(45);
    draw(circle(O, R)); draw(circle(O1, r), blue);

    pair K = O1 + r * dir(270); dot("$K$", K, dir(305));
    pair P1 = K - rotate(90) * (O1 - K);
    pair P2 = K + rotate(90) * (O1 - K);

    pair[] AA = intersectionpoints(line(P1, P2), circle(O, R));
    dot("$B$", AA[0], dir(225)); dot("$C$", AA[1], dir(315));
    
    pair A = R * dir(98); dot("$A$", A, dir(100));
    pair T = O1 + r * dir(45); dot("$T$", T, dir(45));
    pair M = O + R * dir(270); dot("$M$", M, dir(225)); 

    pair I = incenter(A, AA[0], AA[1]);
    dot("$I$", I, dir(50));

    pair[] LL = intersectionpoints(line(K, I), circle(O1, r));
    dot("$L$", LL[0], dir(150));

    pair D = extension(A, LL[0], AA[0], AA[1]);
    dot("$D$", D, dir(225)); 
    
    draw(M--T, blue);
    draw(A--AA[0]--AA[1]--cycle);
    draw(A--M);
    draw(A--D);
    draw(K--LL[0]);

    pair I1 = I + rotate(90) * (K - I);
    pair P = intersectionpoint(line(A, D), line(I, I1));
    pair Q = extension(I, P, AA[0], AA[1]);

    dot("$Q$", P, dir(140)); dot("$P$", Q, dir(225));
    pair[] UU = intersectionpoints(line(Q, M), circumcircle(A, AA[0], AA[1]));
    dot("$U$", UU[1], dir(135)); draw(circumcircle(UU[1], P, Q), red);
    draw(M--UU[1], red); draw(Q--I);

    pair O2 = circumcenter(UU[1], P, Q);
    dot("$O_2$", O1, dir(100)); dot("$O_1$", O2, dir(120));
    draw(O2--O1, heavygray+dashed);
    
    draw(O2--Q); draw(O1--K);
    draw(O1--D); draw(O2--D);

    pair R, S;
    R = extension(O2, D, P, Q);
    S = extension(O1, D, K, LL[0]);

    dot("$R$", R, dir(180));
    dot("$S$", S, dir(0));
    draw(O2--P); draw(O1--LL[0]);
  \end{asy}
\end{figure}
\begin{proof}
  Suppose $\odot(O_1)$ touches $\overline{BC}$ at $P$ and $\overline{AD}$ at $Q$, and $\odot(O_2)$ touches $\overline{BC}$ at $K$ and $\overline{AD}$ at $L$. Let $\overline{PQ}$ intersect $\overline{O_1D}$ at $R$ and $\overline{KL}$ intersect $\overline{O_2D}$ at $S$.

  Since $\odot(O_1)$ and $\odot(O_2)$ are the $A$-curvilinear incircles of cevian $\overline{AD}$, hence $PQ$ intersects $LK$ at $I$. It's easy to that the quadrilaterals $PO_1QD$ and $LDO_2K$ are kites, which is because, 
  \begin{align*}
    \angle O_1PD = \angle O_1QD = 90^{\circ} \text{ \& } \angle O_2LD = \angle O_2KD = 90^{\circ}
  \end{align*}
  and $DP$ $=$ $QD$, $DK$ $=$ $DL$. Further, this implies that $DO_1$ and $DO_2$ are the angle bisectors of $\angle PDQ$ and $\angle KDL$. Therefore,
  \begin{align*}
    \angle O_1DO_2 = \angle O_1DQ + \angle LDO_2 = \frac{1}{2} \left( \angle PDQ + \angle KDL \right) = 90^{\circ}
  \end{align*}
  Since, $PQ \perp O_1D$ and $KL \perp O_2D$ $\implies$ $\overline{PQ}$ $\parallel$ $\overline{O_2D}$ and $\overline{KL}$ $\parallel$ $\overline{O_1D}$. Hence, $RDSI$ is a rectangle and $\angle PIK$ $=$ $90^{\circ}$. Suppose $\angle DO_2K = \alpha$, then
  \begin{align*}
    \angle O_1PQ = \angle O_1DP = \angle DKL = \angle DO_2K = \alpha    
  \end{align*}
  Therefore,
  \begin{align*}
    \frac{O_1R}{RI} &= \frac{O_1P\sin \alpha}{DS} = \frac{O_1D \sin^2 \alpha}{DS} \\
    &= \frac{O_1D \sin^2 \alpha}{DK \sin \alpha} = \frac{O_1D \sin^2 \alpha}{O_2D \sin^2 \alpha} = \frac{O_1D}{O_2D} 
  \end{align*}
  and as a result, we have $\triangle O_1RI$ $\sim$ $\triangle O_1DO_2$ $\implies$ $I$ lies on $\overline{O_1O_2}$.
\end{proof}

\subsection{Examples}
\begin{problem}
  Let $O$ be any point on the diameter $AB$ of a circle $\omega$. Let the perpendicular to $AB$ at $O$ meet $\omega$ at $P$. Suppose that the incircles of the curvilinear triangles $AOP$ and $BOP$ meet $AB$ at $R$ and $S$ respectively. Prove that $\angle{RPS}$ is independent of the position of $O$.
\end{problem}
\begin{figure}[h]
  \centering
  \begin{asy}
    import geometry;
    size(8cm); defaultpen(fontsize(10pt));

    pair  A, B, C, O;
    A = (-3, 0); B = (3, 0);
    O = (1.3, 0); C = (A + B) / 2;

    pair O1 = rotate(90, O) * A;
    pair[] PP = intersectionpoints(line(O, O1), circle(C, abs(A - C)));

    dot("$A$", A, dir(225));
    dot("$B$", B, dir(315));
    dot("$O$", O, dir(225));
    dot("$P$", PP[0], dir(45));

    draw(A--B);
    draw(circumcircle(A, B, PP[0]));

    pair I = incenter(A, B, PP[0]);
    pair M = dir(270) * 3; dot("$I$", I, dir(150));

    pair X = O + 3 * (unit(A - O) + unit(PP[0] - O));
    pair Y = O + 3 * (unit(B - O) + unit(PP[0] - O));

    pair I1 = I + (Y - O);
    pair I2 = I + (X - O);
    pair R = extension(I, I1, A, B);
    dot("$R$", R, dir(225));

    pair S = extension(I, I2, A, B);
    dot("$S$", S, dir(315));

    pair U = extension(I, I1, PP[0], O);
    pair V = extension(I, I2, PP[0], O);
    dot("$U$", U, dir(150));
    dot("$V$", V, dir(30));

    draw(R--U); draw(S--I); draw(PP[0]--O);
    draw(A--B--PP[0]--cycle);

    pair O1 = R + (U - O);
    pair O2 = S + (V - O);

    dot("$O_1$", O1, dir(220));
    dot("$O_2$", O2, dir(90));

    draw(circle(O1, abs(O1 - R)));
    draw(circle(O2, abs(O2 - S)));

    draw(R--PP[0]--S, red);

    draw(arc(circumcircle(PP[0], A, R), -130, 10), blue+dashed);
    draw(arc(circumcircle(PP[0], B, S), -240, -60), blue+dashed);
  \end{asy}
\end{figure}
\begin{proof}
  Since $\overline{IR}$ and $\overline{IS}$ are parallel to the angle bisectors of $\angle PAO = 90^{\circ}$ which means that $\triangle IRS$ is right-angled isosceles. Since $\angle API$ $=$ $\angle BPI$ $=$ $45^{\circ}$ $\implies$ $PARI$ and $BSIP$ are cyclic quadrilaterals. Hence,
  \begin{align*}
    \angle RPS &= 180^{\circ} - \angle PRS - \angle PSR \\ 
          &= 90^{\circ} - \angle PRI - \angle PSI \\ 
          &= 90^{\circ} - \tfrac{1}{2} \left( \angle PAB + \angle PBA \right) \\ 
          &= 45{^\circ}
  \end{align*}
  which is independent of the position of $O$.
\end{proof}

\begin{problem}[Romania TST 2006]
  Let $ABC$ be an acute triangle with $AB \neq AC$. Let $D$ be the foot of the altitude from $A$ and $\omega$ the circumcircle of the triangle. Let $\omega_1$ be the circle tangent to $AD$, $BD$ and $\omega$. Let $\omega_2$ be the circle tangent to $AD$, $CD$ and $\omega$. Let $\ell$ be the interior common tangent to both $\omega_1$ and $\omega_2$, different from $AD$. Prove that $\ell$ passes through the midpoint of $BC$ if and only if $2BC = AB + AC$.
\end{problem}
\begin{figure}[h]
  \centering
  \begin{asy}
    import geometry;
    size(8cm); defaultpen(fontsize(10pt));

    pair A, B, C, D;
    A = dir(105);
    B = dir(210);
    C = dir(330);
    D = foot(A, B, C);

    draw(A--B--C--cycle);
    dot("$A$", A, dir(130));
    dot("$B$", B, dir(225));
    dot("$C$", C, dir(315));
    dot("$D$", D, dir(225));

    pair I = incenter(A, B, C);
    dot("$I$", I, dir(60));

    pair X = D + 3 * (unit(B - D) + unit(A - D));
    pair Y = D + 3 * (unit(C - D) + unit(A - D));

    pair I1 = I + (Y - D);
    pair I2 = I + (X - D);
    pair R = extension(I, I1, B, C);
    dot("$T_1$", R, dir(225));

    pair S = extension(I, I2, B, C);
    dot("$T_2$", S, dir(315));

    pair U = extension(I, I1, A, D);
    pair V = extension(I, I2, A, D);
    dot("$S_1$", U, dir(165));
    dot("$S_2$", V, dir(30));

    draw(circumcircle(A, B, C));
    draw(R--I); draw(S--V);

    pair O1 = R + (U - D);
    pair O2 = S + (V - D);
    dot("$O_1$", O1, dir(120));
    dot("$O_2$", O2, dir(60));

    draw(circle(O1, abs(O1 - R)));
    draw(circle(O2, abs(O2 - S)));

    draw(A--D);

    pair M = (B + C) / 2;
    dot("$M$", M, dir(225));

    draw(O1--O2);
  \end{asy}
\end{figure}
\begin{proof}
  Suppose $I$ is the incenter of $\triangle ABC$, $O_1$, $O_2$ are the centers of $\omega_1$ and $\omega_2$ and they touch $\overline{BC}$ and $\overline{AD}$ at points $T_1$, $S_1$ and $T_2$, $S_2$ respectively. Let $M$ be the midpoint of $\overline{BC}$.

  By sawayama's theorem, we know that $T_1S_1$ and $T_2S_2$ pass through $I$ and they are parallel to angle bisectors of $\angle ADB$. Since $\angle ADB$ $=$ $90^{\circ}$ $\implies$ $\triangle IT_1T_2$ is right-angled isosceles. Let the internal common tangent of $\omega_1$ and $\omega_2$ meet $\overline{BC}$ at $P$. Since, $\angle O_1DO_2$ $=$ $90^{\circ}$ and $\angle O_1PO_2$ $=$ $90^{\circ}$, we get that $O_1DPO_2$ is cyclic. By thebault's theorem, we get that $I$ lies on $\overline{O_1O_2}$. Since $\triangle IT_1T_2$ is right-angled isosceles, hence if the incircle touches $\overline{BC}$ at $X$, then $X$ must be the midpoint of $\overline{T_1T_2}$ $\implies$ $I$ must be the midpoint of $\overline{O_1O_2}$. Since $\angle O_1DO_2$ $=$ $90^{\circ}$ $\implies$ $I$ is the center of $\odot(O_1DPO_2)$ $\implies$ $X$ is the midpoint of $\overline{DP}$ $\implies$ $\overline{DT_1}$ $=$ $\overline{PT_2}$. Using the fact that $\cos \left( \tfrac{B}{2} \right) = \sqrt{\tfrac{s(s-b)}{ca}}$, and
  \begin{align*}
    \overline{DX} = (s - b) - c \cos B \text{ \& } \overline{XP} = (s - c) - \tfrac{a}{2}
  \end{align*}
  we get $P$ is the midpoint of $\overline{BC}$ if and only if $2a = b + c$.
\end{proof}

\subsection{Exercises}
\begin{exercise}
  Two circles $\Omega$ and $\Gamma$ are internally tangent at the point $B$. The chord $AC$ of $\Gamma$ is tangent to $\Omega$ at the point $L$, and the segments $AB$ and $BC$ intersect $\Omega$ at the points $M$ and $N$. Let $M_1$ and $N_1$ be the reflections of $M$ and $N$ about the line $BL$; and let $M_2$ and $N_2$ be the reflections of $M$ and $N$ about the line $AC$. The lines $M_1M_2$ and $N_1N_2$ intersect at the point $K$. Prove that the lines $BK$ and $AC$ are perpendicular.
\end{exercise}

\begin{exercise}[Bulgaria 2005]
  Consider two circles $k_{1},k_{2}$ touching externally at point $T$. a line touches $k_{2}$ at point $X$ and intersects $k_{1}$ at points $A$ and $B$. Let $S$ be the second intersection point of $k_{1}$ with the line $XT$ . On the arc $\widehat{TS}$ not containing $A$ and $B$ is chosen a point $C$ . Let $\ CY$ be the tangent line to $k_{2}$ with $Y\in k_{2}$ , such that the segment $CY$ does not intersect the segment $ST$ . If $I=XY\cap SC$ . Prove that :
  \begin{enumerate}
    \item the points $C,T,Y,I$ are concyclic.
    \item $I$ is the excenter of triangle $ABC$ with respect to the side $BC$.
  \end{enumerate}
\end{exercise}

\section{Mixtilinear Incircles}
In the previous section, we looked at circles tangent to cevians called the curvilinear incircles. A special case arises when the point $D$ coincides with a vertex of the triangle.
\begin{definition}
    Given $\triangle ABC$, the circle internally tangent to $\odot(ABC)$ and sides $\overline{AB}$ and $\overline{AC}$ is known as the $A$-\vocab{mixtilinear incircle} of $\triangle ABC$.
\end{definition}
\begin{figure}[h]
  \centering
  \begin{asy}
    import geometry;
    size(6cm); defaultpen(fontsize(10pt));

    pair A, B, C, O, N, I;
    O = origin; A = dir(120); B = dir(210);
    C = dir(330); N = dir(90); I = incenter(A, B, C);

    pair A1 = rotate(90, I) * A;
    pair B1 = extension(I, A1, A, B);
    pair C1 = extension(I, A1, A, C);
   
    pair[] TT = intersectionpoints(line(N, I), circumcircle(A, B, C));
    pair T = TT[0];

    dot("$A$", A, dir(130));
    dot("$B$", B, dir(225));
    dot("$C$", C, dir(315));
    dot("$T$", T, dir(250));
    dot("$B_1$", B1, dir(150));
    dot("$C_1$", C1, dir(30));

    draw(A--B--C--cycle);
    draw(circumcircle(A, B, C));
    draw(circumcircle(T, B1, C1));
  \end{asy}
\end{figure}

In a triangle, there are three mixtilinear incircles. Each opposite to a vertex of the triangle. Since mixtilinear incircles are a special case of curvilinear incircles, hence the properties discussed in the previous sections apply to the mixtilinear incircles too!

\subsection{Immediate Properties}
\begin{proposition}
  Given $\triangle ABC$ and the $A$-mixtilinear incircle $\omega_A$ that touches the sides $\overline{AB}$ and $\overline{AC}$ at $B_1$ and $C_1$, and the circle $\odot(ABC)$ at $T$. Suppose $E$ and $F$ are the midpoints of the arc $AC$ not containing $B$ and arc $AB$ not containing $C$, then the points
  \begin{enumerate}
    \item $T$, $B_1$, $F$ are collinear.
    \item $T$, $C_1$, $E$ are collinear.
  \end{enumerate}
\end{proposition}
\begin{proof}
Immediate application of archimedes' lemma.
\end{proof}

\begin{proposition}[Verrier's Lemma]
  Given $\triangle ABC$ and the $A$-mixtilinear incircle $\omega_A$ that touches the sides $\overline{AB}$ and $\overline{AC}$ at $B_1$ and $C_1$, and the circle $\odot(ABC)$ at $T$. If $I$ is the incenter of $\triangle ABC$ then $I$ lies on the line $\overline{B_1C_1}$.
\end{proposition}
\begin{proof}
  Since $\omega_A$ is also the $C$-curvilinear incircle of the cevian $\overline{CA}$, hence the result follows from sawayama's theorem.
\end{proof}
\begin{figure}[h]
  \centering
  \begin{asy}
    import geometry;
    size(6cm); defaultpen(fontsize(10pt));

    pair A, B, C, O, N, I, E, F;
    O = origin; A = dir(120); B = dir(210);
    C = dir(330); N = dir(90); I = incenter(A, B, C);
    E = dir(45); F = dir(165);

    pair A1 = rotate(90, I) * A;
    pair B1 = extension(I, A1, A, B);
    pair C1 = extension(I, A1, A, C);
   
    pair[] TT = intersectionpoints(line(N, I), circumcircle(A, B, C));
    pair T = TT[0];

    dot("$A$", A, dir(130));
    dot("$B$", B, dir(225));
    dot("$C$", C, dir(315));
    dot("$T$", T, dir(250));
    dot("$B_1$", B1, dir(150));
    dot("$C_1$", C1, dir(5));
    dot("$I$", I, dir(150));
    dot("$E$", E, dir(45));
    dot("$F$", F, dir(135));

    draw(A--B--C--cycle);
    draw(circumcircle(A, B, C));
    draw(circumcircle(T, B1, C1));
    draw(T--E); draw(T--F); draw(B1--C1, gray); 
    draw(A--I); draw(E--F, gray); markrightangle(C1, I, A, 7);

    draw(B--E, gray+dashed); draw(C--F, gray+dashed);
  \end{asy}
\end{figure}
\newpage
\subsection{Some More Properties}
\begin{proposition}
  Given $\triangle ABC$ and the $A$-mixtilinear incircle $\omega_A$ that touches the sides $\overline{AB}$ and $\overline{AC}$ at $B_1$ and $C_1$, and the circle $\odot(ABC)$ at $T$. Let $I$ be the incenter of $\triangle ABC$. Then $IB_1BT$ and $IC_1CT$ are cyclic quadrilaterals.
\end{proposition}
\begin{figure}[h]
  \centering
  \begin{asy}
    import geometry;
    size(7cm); defaultpen(fontsize(10pt));

    pair A, B, C, O, N, I, E, F;
    O = origin; A = dir(120); B = dir(210);
    C = dir(330); N = dir(90); I = incenter(A, B, C);
    E = dir(45); F = dir(165);

    pair A1 = rotate(90, I) * A;
    pair B1 = extension(I, A1, A, B);
    pair C1 = extension(I, A1, A, C);
   
    pair[] TT = intersectionpoints(line(N, I), circumcircle(A, B, C));
    pair T = TT[0];

    dot("$A$", A, dir(130));
    dot("$B$", B, dir(225));
    dot("$C$", C, dir(315));
    dot("$T$", T, dir(250));
    dot("$B_1$", B1, dir(150));
    dot("$C_1$", C1, dir(5));
    dot("$I$", I, dir(270));
    dot("$E$", E, dir(45));
    dot("$F$", F, dir(135));

    draw(A--B--C--cycle);
    draw(circumcircle(A, B, C));
    draw(circumcircle(T, B1, C1));
    draw(T--E); draw(T--F); draw(B1--C1, gray); 
    draw(A--I); draw(E--F, gray); markrightangle(C1, I, A, 7);

    draw(B--E, gray); draw(C--F, gray);
    draw(circumcircle(T, B, I), heavygray+dashed);
    draw(circumcircle(T, C, I), heavygray+dashed);
  \end{asy}
\end{figure}
\begin{proof}
  Converse of reim's theorem applied on $\overline{B_1I}$ $\parallel$ $\overline{EF}$ and $\overline{C_1I}$ $\parallel$ $\overline{EF}$ implies that $IB_1BT$ and $IC_1CT$ are cyclic.
\end{proof}

\begin{proposition}
  Given $\triangle ABC$ and the $A$-mixtilinear incircle $\omega_A$ that touches the sides $\overline{AB}$ and $\overline{AC}$ at $B_1$ and $C_1$, and the circle $\odot(ABC)$ at $T$. Let $I$ be the incenter of $\triangle ABC$. Suppose $N$ is the midpoint of arc $BAC$, then $TI$ passes through $N$.
\end{proposition}
\begin{figure}[h]
  \centering
  \begin{asy}
    import geometry;
    size(7cm); defaultpen(fontsize(10pt));

    pair A, B, C, O, N, I, E, F;
    O = origin; A = dir(120); B = dir(210);
    C = dir(330); N = dir(90); I = incenter(A, B, C);
    E = dir(45); F = dir(165);

    pair A1 = rotate(90, I) * A;
    pair B1 = extension(I, A1, A, B);
    pair C1 = extension(I, A1, A, C);
   
    pair[] TT = intersectionpoints(line(N, I), circumcircle(A, B, C));
    pair T = TT[0];

    dot("$A$", A, dir(130));
    dot("$B$", B, dir(225));
    dot("$C$", C, dir(315));
    dot("$T$", T, dir(250));
    dot("$B_1$", B1, dir(150));
    dot("$C_1$", C1, dir(5));
    dot("$I$", I, dir(300));
    dot("$E$", E, dir(45));
    dot("$F$", F, dir(135));
    dot("$N$", N, dir(90));

    draw(A--B--C--cycle);
    draw(circumcircle(A, B, C));
    draw(circumcircle(T, B1, C1));
    draw(T--E); draw(T--F); draw(B1--C1, gray); 
    draw(A--I); draw(E--F, gray); markrightangle(C1, I, A, 7);

    draw(B--E, gray); draw(C--F, gray);
    draw(circumcircle(T, B, I), heavygray);
    draw(circumcircle(T, C, I), heavygray);

    draw(T--N, heavygray+dashed);
  \end{asy}
\end{figure}
\begin{proof}
  Since $AB_1$ and $AC_1$ are tangents drawn from $A$ to $\omega_A$ $\implies$ $\triangle AB_1C_1$ is isosceles. Hence,
  \begin{align*}
    \angle BTI = \angle AB_1I = \angle AC_1I = \angle ITC
  \end{align*}
  So $\overline{TI}$ is the angle bisector of $\angle BTC$ $\implies$ $TI$ passes through the midpoint of the arc $BAC$.
\end{proof}
\begin{remark}
  There are lot of angle bisectors in this configuration. For example,
  \begin{enumerate}
    \item $\overline{TB_1}$ is the $T$-angle bisector of $\triangle ATB$.
    \item $\overline{TC_1}$ is the $T$-angle bisector of $\triangle ATC$.
  \end{enumerate}
  Also $\overline{TA}$ is the $T$-symmedian in $\triangle TB_1C_1$. All of this makes the configuration a really nice treasure trove for angle bisector theorem and ratio lemma applications.
\end{remark}

\subsection{Isogonal Lines}
\begin{proposition}
  Given $\triangle ABC$ and the $A$-mixtilinear incircle $\omega_A$ that touches the sides $\overline{AB}$ and $\overline{AC}$ at $B_1$ and $C_1$, and the circle $\odot(ABC)$ at $T$. Let $I$ be the incenter of $\triangle ABC$. Then lines $TA$ and $TI$ are isogonal with respect to $\angle FTE$.
\end{proposition}
\begin{proof}
  Effectively, we want to show that $\angle ATF$ $=$ $\angle ITE$. However,
  \begin{align*}
    \angle ATF = \angle ACF = \angle ICC_1 = \angle ITC_1 = \angle ITE
  \end{align*}
  which implies the isogonal condition.
\end{proof}

\begin{proposition}
  Given $\triangle ABC$ and the $A$-mixtilinear incircle $\omega_A$ that touches the sides $\overline{AB}$ and $\overline{AC}$ at $B_1$ and $C_1$, and the circle $\odot(ABC)$ at $T$. Let $I$ be the incenter of $\triangle ABC$ and $X$ be the point where the $A$-excircle touches $\overline{BC}$. Then $\overline{AX}$ and $\overline{AT}$ are isogonal with respect to $\angle BAC$.
\end{proposition}
\begin{figure}[h]
  \centering
  \begin{asy}
    import geometry;
    size(7cm); defaultpen(fontsize(10pt));

    pair A, B, C, O, N, I, E, F, IA;
    O = origin; A = dir(120); B = dir(210);
    C = dir(330); N = dir(90); I = incenter(A, B, C);
    E = dir(45); F = dir(165); IA = excenter(C, B, A);

    pair A1 = rotate(90, I) * A;
    pair B1 = extension(I, A1, A, B);
    pair C1 = extension(I, A1, A, C);
   
    pair[] TT = intersectionpoints(line(N, I), circumcircle(A, B, C));
    pair T = TT[0];

    dot("$A$", A, dir(130));
    dot("$B$", B, dir(225));
    dot("$C$", C, dir(315));
    dot("$T$", T, dir(250));
    dot("$B_1$", B1, dir(150));
    dot("$C_1$", C1, dir(5));
    dot("$I$", I, dir(300));

    draw(A--B--C--cycle);
    draw(circumcircle(A, B, C));
    draw(circumcircle(T, B1, C1));
    draw(B1--C1, gray); draw(A--T); 
    draw(T--B); draw(T--B1, gray);
    draw(A--I); markrightangle(C1, I, A, 7);

    pair X = foot(IA, B, C);
    dot("$X$", X, dir(270));
    draw(A--X);

  \end{asy}
\end{figure}
\begin{proof}
  We will show that $\triangle ATB$ $\sim$ $\triangle ACX$. We already have that $\angle ATB$ $=$ $\angle ACX$. Hence it only remains to establish that,
  \begin{align*}
    \frac{\overline{AT}}{\overline{TB}} = \frac{\overline{AC}}{\overline{CX}}
  \end{align*}
  Using the relation $\cos \left( \tfrac{A}{2} \right) = \sqrt{\tfrac{s(s-a)}{bc}}$, we get
  \begin{align*}
    \frac{\overline{AT}}{\overline{TB}} &= \frac{\overline{BB_1}}{\overline{AB_1}} = \frac{\overline{AB} - \overline{AB_1}}{\overline{AB_1}} \\ 
                &= \frac{\overline{AB}}{\overline{AB_1}} - 1 = \frac{c \cos^2 \left( \tfrac{A}{2} \right)}{s - a} - 1 \\ 
                &= \frac{c \left( \frac{s(s-a)}{bc} \right)}{s(s-a)} - 1 = \frac{s - b}{b} = \frac{\overline{AC}}{\overline{CX}}
  \end{align*}
  and thus, $\triangle ATB$ $\sim$ $\triangle ACX$ $\implies$ $AT$ and $AX$ are isogonal with respect to $\angle BAC$.
\end{proof}

\begin{proposition}
  Given $\triangle ABC$ and the $A$-mixtilinear incircle $\omega_A$ that touches the sides $\overline{AB}$ and $\overline{AC}$ at $B_1$ and $C_1$, and the circle $\odot(ABC)$ at $T$. Let $I$ be the incenter of $\triangle ABC$ and $D$ be the point where the incircle of $\triangle ABC$ touches $\overline{BC}$. Then $TA$ and $TD$ are isogonal with respect to $\angle BTC$.
\end{proposition}
\begin{figure}[h]
  \centering
  \begin{asy}
    import geometry;
    size(7cm); defaultpen(fontsize(10pt));

    pair A, B, C, O, N, I, E, F;
    O = origin; A = dir(120); B = dir(210);
    C = dir(330); N = dir(90); I = incenter(A, B, C);
    E = dir(45); F = dir(165);

    pair A1 = rotate(90, I) * A;
    pair B1 = extension(I, A1, A, B);
    pair C1 = extension(I, A1, A, C);
   
    pair[] TT = intersectionpoints(line(N, I), circumcircle(A, B, C));
    pair T = TT[0];

    dot("$A$", A, dir(130));
    dot("$B$", B, dir(225));
    dot("$C$", C, dir(315));
    dot("$T$", T, dir(250));
    dot("$B_1$", B1, dir(150));
    dot("$C_1$", C1, dir(5));
    dot("$I$", I, dir(300));
    dot("$N$", N, dir(90));

    draw(A--B--C--cycle);
    draw(circumcircle(A, B, C));
    draw(circumcircle(T, B1, C1));
    draw(B1--C1, gray); 
    draw(A--I); markrightangle(C1, I, A, 7);

    pair D = foot(I, B, C); draw(I--D);
    dot("$D$", D, dir(315)); markrightangle(C, D, I, 7);

    draw(T--N, heavygray); draw(B--T); draw(C--T);
    draw(A--T); draw(D--T);
  \end{asy}
\end{figure}
\begin{proof}
Suppose $X$ is the point where the $A$-excircle touches $\overline{BC}$. Since $\angle BAX$ $=$ $\angle TAC$ $\implies$ $\triangle ATC$ $\sim$ $\triangle ABX$. So, we have 
\begin{align*}
  \frac{\overline{AT}}{\overline{TC}} = \frac{\overline{AB}}{\overline{BX}} = \frac{\overline{AB}}{\overline{CD}}
\end{align*}
This can be written as
\begin{align*}
  \frac{\overline{AT}}{\overline{AB}} = \frac{\overline{TC}}{\overline{CD}} 
\end{align*}
and using the fact that $\angle BAT$ $=$ $\angle DCT$, we can claim by the SAS similarity criterion that $\triangle ATB$ $\sim$ $\triangle CTD$. This implies that 
\begin{align*}
  \angle BTA = \angle DTC
\end{align*}
or in other words, lines $TA$ and $TD$ are isogonal with respect to $\angle BTC$.
\end{proof}

\begin{proposition}
  Given $\triangle ABC$ and the $A$-mixtilinear incircle $\omega_A$ that touches the sides $\overline{AB}$ and $\overline{AC}$ at $B_1$ and $C_1$, and the circle $\odot(ABC)$ at $T$. Let $I$ be the incenter of $\triangle ABC$ and $P$ be the intersection of $\overline{AT}$ and $\overline{B_1C_1}$. Then
  \begin{align*}
    \angle BPB_1 = \angle CPC_1
  \end{align*}
\end{proposition}
\begin{figure}[h]
  \centering
  \begin{asy}
    import geometry;
    size(7cm); defaultpen(fontsize(10pt));

    pair A, B, C, O, N, I, E, F;
    O = origin; A = dir(120); B = dir(210);
    C = dir(330); N = dir(90); I = incenter(A, B, C);
    E = dir(45); F = dir(165);

    pair A1 = rotate(90, I) * A;
    pair B1 = extension(I, A1, A, B);
    pair C1 = extension(I, A1, A, C);
   
    pair[] TT = intersectionpoints(line(N, I), circumcircle(A, B, C));
    pair T = TT[0];

    dot("$A$", A, dir(130));
    dot("$B$", B, dir(225));
    dot("$C$", C, dir(315));
    dot("$T$", T, dir(250));
    dot("$B_1$", B1, dir(150));
    dot("$C_1$", C1, dir(5));
    dot("$I$", I, dir(300));

    draw(A--B--C--cycle);
    draw(circumcircle(A, B, C));
    draw(circumcircle(T, B1, C1));
    draw(B1--C1, gray); 
    draw(A--I); markrightangle(C1, I, A, 7);
    draw(B--T); draw(C--T); draw(A--T);

    pair P = extension(A, T, B1, C1);

    dot("$P$", P, dir(130));
    draw(B--P); draw(C--P);
  \end{asy}
\end{figure}
\begin{proof}
  We will show that $\triangle BPB_1$ $\sim$ $\triangle CPC_1$. Since,
  \begin{align*}
    \frac{\overline{B_1P}}{\overline{C_1P}} &= \frac{\sin \angle BAT}{\sin \angle CAT}
    = \frac{\overline{BT}}{\overline{CT}} = \frac{\tfrac{\overline{BB_1}}{\overline{AB_1}}\cdot \overline{AT}}
    {\tfrac{\overline{CC_1}}{\overline{C_1A}} \cdot \overline{AT}}
    = \frac{\overline{BB_1}}{\overline{CC_1}}
  \end{align*}
  Since $\angle BB_1P = \angle CC_1P$ $\implies$ $\triangle BPB_1$ $\sim$ $\triangle CPC_1$ by SAS similarity criterion.
\end{proof}

\subsubsection{Exercises}
\begin{exercise}
  In \vocab{Proposition 3.10}, if $CP$ is extended to intersect $AB$ at $Q$ and $BP$ is extended to intersect $AC$ at $R$, then $BQRC$ is cyclic.
\end{exercise}

\begin{exercise}
  Given $\triangle ABC$, let $D$ be the point where the incircle of $\triangle ABC$ touches $\overline{BC}$ and $T$ be the point where the $A$-mixtilinear incircle touches $\odot(ABC)$. Suppose the tangent to $\odot(ABC)$ at $A$ intersects $\overline{BC}$ at $X$, then show that $AXTD$ is a cyclic quadrilateral.
\end{exercise}

\begin{exercise}
  Given $\triangle ABC$ and its $A$-mixtilinear incircle $\omega_A$. Suppose $\omega_A$ touches $\odot(ABC)$ at $T$ and sides $\overline{AB}$, $\overline{AC}$ at $B_1$, $C_1$. Let $M$ be the midpoint of arc $BC$ not containing $A$. Show that $B_1C_1$, $BC$ and $TM$ are concurrent.
\end{exercise}

\begin{exercise}
  Given $\triangle ABC$, its incenter $I$ and its $A$-mixtilinear incircle $\omega_A$. Let the incircle touch $\overline{BC}$ at $D$ and $\omega_A$ touch $\odot(ABC)$ at $T$ and the sides $\overline{AB}$ and $\overline{AC}$ at $B_1$ and $C_1$. Suppose the angle bisector of $\angle BAC$ intersects $\overline{BC}$ and $\odot(ABC)$ at $K$ and $M$. Then show that $KDTM$ is a cyclic quadrilateral. 
\end{exercise}

\begin{exercise}
  Given $\triangle ABC$, its incenter $I$ and its $A$-mixtilinear incircle $\omega_A$. Let the incircle touch $\overline{BC}$ at $D$ and $\omega_A$ touch $\odot(ABC)$ at $T$ and the sides $\overline{AB}$ and $\overline{AC}$ at $B_1$ and $C_1$. Suppose $B_1C_1$ intersects $BC$ at $X$, then show that $IXTD$ is a cyclic quadrilateral.
\end{exercise}

\begin{exercise}
  Given $\triangle ABC$, its incenter $I$ and its $A$-mixtilinear incircle $\omega_A$. Let the incircle touch $\overline{BC}$ at $D$ and $\omega_A$ touch $\odot(ABC)$ at $T$ and the sides $\overline{AB}$ and $\overline{AC}$ at $B_1$ and $C_1$. Suppose $B_1C_1$ intersects $BC$ at $X$, $AT$ intersects $\overline{B_1C_1}$ at $P$ and $\overline{TI}$ intersects $\overline{BC}$ at $Q$. Then show that $IPQD$ and $PXTQ$ are cyclic quadrilaterals.
\end{exercise}

\begin{exercise}
  Given $\triangle ABC$, its incenter $I$ and its $A$-mixtilinear incircle $\omega_A$. Let the incircle touch $\overline{BC}$ at $D$ and $\omega_A$ touch $\odot(ABC)$ at $T$ and the sides $\overline{AB}$ and $\overline{AC}$ at $B_1$ and $C_1$. Suppose the tangent to $\odot(ABC)$ at $T$ intersects $BC$ at $Y$, $AT$ intersects $\overline{B_1C_1}$ at $P$ and $\overline{TI}$ intersects $\overline{BC}$ at $Q$. Then show that $YPDT$ is cyclic and $Y$ is the center of $\odot(PQT)$.
\end{exercise}

\subsection{Multiple Mixtilinear Incircles}
Now let's add multiple mixtilinear incircles to the configuration. We use the notation $\omega_X$ to denote the $X$-mixtilinear incircle (the mixtilinear incircle opposite to vertex $X$).

\begin{proposition}
  Given $\triangle ABC$ and its incenter $I$, let the incircle touche $\overline{BC}$ at $D$, $M$ be the midpoint of arc $BC$ not containing $A$ and $N$ be midpoint of $\overline{ID}$. Then $MN$ is the radical axis of $\omega_B$ and $\omega_C$. 
\end{proposition}
\begin{figure}[h]
  \centering
  \begin{asy}
    import geometry;
    size(8cm); defaultpen(fontsize(10pt));
  
    pair A, B, C, I, D, E, F, G, M;
    A = dir(120); B = dir(210); C = dir(330);
    I = incenter(A, B, C); M = dir(270);

    pair B1 = rotate(90, I) * B;
    pair C1 = rotate(90, I) * C;
    D = extension(I, C1, B, C);
    E = extension(I, B1, B, C);
    F = extension(I, C1, A, C);
    G = extension(I, B1, A, B);

    pair[] TC = intersectionpoints(line(M, D), circumcircle(A, B, C));
    pair[] TB = intersectionpoints(line(M, E), circumcircle(A, B, C));

    dot("$A$", A, dir(130));
    dot("$B$", B, dir(225));
    dot("$C$", C, dir(315));
    dot("$P$", D, dir(225));
    dot("$R$", E, dir(315));
    dot("$Q$", F, dir(45));
    dot("$S$", G, dir(135));
    dot("$M$", M, dir(225));
    dot("$T_C$", TC[1], dir(150));
    dot("$T_B$", TB[1], dir(30));
    dot("$I$", I, dir(0));

    draw(A--B--C--cycle); draw(circumcircle(A, B, C));
    draw(circumcircle(TC[1], D, F), red);
    draw(circumcircle(TB[1], E, G), blue);

    draw(M--TC[1]); draw(M--TB[1]);
    draw(D--F, red); draw(E--G, blue);

    pair A1, B1, C1;
    A1 = foot(I, B, C); B1 = foot(I, C, A); C1 = foot(I, A, B); 
    dot("$D$", A1, dir(225)); dot("$E$", B1, dir(45)); dot("$F$", C1, dir(135));
    draw(circumcircle(A1, B1, C1)); draw(I--A1, heavygray); draw(I--B1, heavygray); draw(I--C1, heavygray);
    markrightangle(C, A1, I, 7);

    pair N = (I + A1) / 2;
    dot("$N$", N, dir(20));
  \end{asy}
\end{figure}
\begin{proof}
  Suppose the incircle touches the sides $\overline{BC}$, $\overline{CA}$ and $\overline{AB}$ at $D$, $E$ and $F$, and $\omega_B$ touches the sides $\overline{BC}$ and $\overline{AB}$ at $R$ and $S$, and $\omega_C$ touches the sides $\overline{BC}$ and $\overline{AC}$ at $P$ and $Q$. Due to archimedes' lemma, we have that $T_CP$ and $T_BR$ pass through $M$ and 
  \begin{align*}
    \overline{MP} \cdot \overline{MT_C} = \overline{MB}^2 =  \overline{MR} \cdot \overline{MT_B}
  \end{align*}
  Hence, $M$ has an equal power with respect to both the circles $\omega_B$ and $\omega_C$. Since the points $P$, $D$, $E$ and $Q$ are the tangency points of common tangents of the incircle and $\omega_C$ $\implies$ their radical axis is the midline of the isosceles trapezium $PQED$ and hence passes through $N$. Similarly, the radical axis of the incircle and $\omega_B$ passes through $N$ $\implies$ $N$ is the radical center of $\omega_B$, $\omega_C$ and incircle and thus, $N$ also has an equal power with respect to $\omega_B$ and $\omega_C$, implying that $MN$ is indeed the radical axis of $\omega_B$ and $\omega_C$. 
\end{proof}

\subsubsection{Exercises}
\begin{exercise}
  Suppose $\omega_A$ touches $\odot(ABC)$ at $T_A$ and $\omega_B$ and $\omega_C$ touch $\overline{BC}$ at $D$ and $E$ and $\odot(ABC)$ at $T_B$ and $T_C$. If $M$ is the midpoint of arc $BC$ not containing $A$, then 
  \begin{enumerate}
    \item $T_CDET_B$ is a cyclic quadrilateral.
    If $T_AM$ intersects $BC$ at $T$,
    \item $\overline{IT_A}$ $\perp$ $\overline{TM}$.
    \item $\overline{TI}$ $\perp$ $\overline{AM}$.
    \item $T_CTT_AD$ is a cyclic quadrilateral.
    \item $EDTM$ is a cyclic quadrilateral.
  \end{enumerate}
\end{exercise}

\subsection{Examples}
\begin{problem}[EGMO 2013]
  Let $\Omega$ be the circumcircle of the triangle $ABC$. The circle $\omega$ is tangent to the sides $AC$ and $BC$, and it is internally tangent to the circle $\Omega$ at the point $P$. A line parallel to $AB$ intersecting the interior of triangle $ABC$ is tangent to $\omega$ at $Q$. Prove that $\angle ACP = \angle QCB$.
\end{problem}
\begin{figure}[h]
  \centering
  \begin{asy}
    import geometry;
    size(8cm); defaultpen(fontsize(10pt));

    pair A, B, C, O, N, I, E, F, IA;
    O = origin; A = dir(120); B = dir(210);
    C = dir(330); N = dir(90); I = incenter(A, B, C);
    E = dir(45); F = dir(165); IA = excenter(C, B, A);

    pair A1 = rotate(90, I) * A;
    pair B1 = extension(I, A1, A, B);
    pair C1 = extension(I, A1, A, C);
   
    pair[] TT = intersectionpoints(line(N, I), circumcircle(A, B, C));
    pair T = TT[0];

    dot("$C$", A, dir(89));
    dot("$A$", B, dir(225));
    dot("$B$", C, dir(315));
    dot("$P$", T, dir(250));
    dot("$I$", I, dir(180));

    draw(A--B--C--cycle);
    draw(circumcircle(A, B, C));
    draw(circumcircle(T, B1, C1));

    pair X = foot(IA, B, C);
    dot("$X$", X, dir(270));
    draw(A--T);

    dot("$N$", N, dir(115));
    pair[] QQ = intersectionpoints(line(A, X), circumcircle(T, B1, C1));
    dot("$Q$", QQ[1], dir(35)); draw(T--N, heavygray+dashed); draw(A--X, heavygray+dashed);

    pair M = dir(270);
    dot("$M$", M, dir(225));
    draw(A--M);

    pair I1 = circumcenter(T, B1, C1);
    dot("$I_1$", I1, dir(20));
    draw(arc(circumcircle(T, A, I1), -90, 90), gray+dashed);
    draw(I1--QQ[1], gray); draw(M--N, gray);
  \end{asy}
\end{figure}
\begin{proof}
  Since $Q$ lies such that the tangent to $\omega$ at $Q$ is parallel to $\overline{AB}$, hence there exists a homothety at $P$ that maps $\omega$ to $\Omega$ under which $Q$ is mapped to the midpoint of arc $ACB$, let's say $N$. Also suppose $I$ is the incenter, $I_1$ is center of $\omega$ and $M$ is the midpoint of arc $AB$ not containing $C$.

  We know that $I_1$ lies on the line $\overline{CIM}$. Since $\overline{I_1Q}$ $\parallel$ $\overline{MN}$, therefore by converse of reim's theorem $\implies$ $PI_1QC$ is cyclic. Since, $\overline{PI_1}$ $=$ $\overline{QI_1}$ $\implies$ $\overline{CI_1}$ is the angle bisector of $\angle PCQ$, or in other words $\angle PCI$ $=$ $\angle QCI$ $\implies$ $\angle ACP$ $=$ $\angle QCB$.
\end{proof}

\begin{problem}[IMO 2019]
  Let $I$ be the incentre of acute triangle $ABC$ with $AB\neq AC$. The incircle $\omega$ of $ABC$ is tangent to sides $BC, CA$, and $AB$ at $D, E,$ and $F$, respectively. The line through $D$ perpendicular to $EF$ meets $\omega$ at $R$. Line $AR$ meets $\omega$ again at $P$. The circumcircles of triangle $PCE$ and $PBF$ meet again at $Q$. Prove that lines $DI$ and $PQ$ meet on the line through $A$ perpendicular to $AI$.
\end{problem}
\begin{figure}[h]
  \centering
  \begin{asy}
    import geometry;
    size(12cm); defaultpen(fontsize(12pt));

    pair A, B, C, I;

    A = dir(130);
    B = dir(200);
    C = dir(340);
    I = incenter(A, B, C);

    pair D, E, F;
    D = foot(I, B, C);
    E = foot(I, C, A);
    F = foot(I, A, B);

    dot("$A$", A, dir(130));
    dot("$B$", B, dir(225));
    dot("$C$", C, dir(315));
    dot("$D$", D, dir(225));
    dot("$E$", E, dir(45));
    dot("$F$", F, dir(135));
    dot("$I$", I, dir(0));

    draw(A--B--C--cycle);
    draw(circumcircle(A, B, C));
    draw(circumcircle(D, E, F));

    pair DD = 2 * I - D;
    dot("$D'$", DD, dir(45));

    pair M = dir(90);
    dot("$M$", M, dir(70));

    pair HH = orthocenter(D, E, F);
    pair R = reflect(E, F) * HH;
    dot("$R$", R, dir(130));

    pair[] PP = intersectionpoints(line(A, R), circumcircle(D, E, F));
    pair P = PP[0]; dot("$P$", P, dir(225));

    pair[] TT = intersectionpoints(line(A, R), circumcircle(A, B, C));
    pair T = TT[0]; dot("$T$", T, dir(225));

    pair K = 2 * T - I;
    dot("$K$", K, dir(225));

    draw(arc(circumcircle(B, I, C), 20, 220), heavygray+dashed);

    pair S = extension(D, I, A, M); dot("$S$", S, dir(120));
    pair[] QQ = intersectionpoints(line(K, P), circumcircle(B, F, P));
    pair Q = QQ[1]; dot("$Q'$", Q, dir(30));

    draw(circumcircle(B, F, P), heavygray);
    draw(circumcircle(C, E, P), heavygray);
    draw(K--S, blue); draw(I--D); draw(I--S, gray+dashed);
    draw(A--M); draw(K--M); draw(D--E--F--cycle, gray); draw(A--T, red); draw(D--R, gray);

    pair N = dir(270);
    dot("$N$", N, dir(315));
    draw(A--N);
    draw(N--M);

    draw(circumcircle(A, D, S), gray+dashed);
    draw(arc(circumcircle(A, I, D), -60, 40), gray+dashed);
    draw(arc(circumcircle(R, I, D), -70, 80), gray+dashed);

  \end{asy}
\end{figure}
\begin{proof}
  Let $T$ be the $A$-mixtilinear touch point with $\odot(ABC)$.
  \begin{claim}
    $T$ lies on the line $AR$.
  \end{claim}
  \begin{proof}
    Suppose $\overline{AT}$ cuts the incircle at $P'$ and $R'$ where $P'$ is closer to $T$ than $R'$. Since $\overline{TA}$ and $\overline{TD}$ are isogonal with respect to $\angle BTC$ and $\overline{BI}$ is the angle bisector of $\angle BTC$ $\implies$ $P'$ is the reflection of $D$ over $\overline{TI}$. Let $M$ be the midpoint of arc $BAC$ and $N$ be the midpoint of arc $BC$ not containing $A$. Then
    \begin{align*}
      \angle DR'P' = \tfrac{1}{2} \angle PID = \angle DIT = \angle NMT = \angle IAT
    \end{align*}
    Hence, $\overline{R'D}$ $\parallel$ $\overline{AI}$. But $\overline{AI}$ $\perp$ $\overline{EF}$ $\implies$ $\overline{R'D}$ $\perp$ $\overline{EF}$ $\implies$ $R' = R$ and $P' = P$.
  \end{proof}

  Define $S$ as the intersection of $DI$ and $AM$. We want to show that $PQ$ passes through $S$ to solve the problem. Suppose $D'$ is the reflection of $D$ over $I$. Since $\overline{DR}$ and $\overline{DD'}$ are isogonal with respect to $\angle EDF$ $\implies$ $\overline{RD'}$ $\parallel$ $\overline{EF}$ $\parallel$ $\overline{AM}$. Therefore, by converse of reim's theorem we get that $ASDP$ is a cyclic quadrilateral. We can further show that $RIDT$ is a cyclic quadrilateral too.
  \begin{align*}
    \angle RTI = \angle ATM = \angle ANM = \angle DIM = \angle RDI
  \end{align*}
  which implies that $RIDT$ is a cyclic quadrilateral. Now, define $K$ as the intersection of line $TI$ with $\odot(BIC)$.
  \begin{claim}
    $AIDK$ is a cyclic quadrilateral.
  \end{claim}
  \begin{proof}
    Since $\angle ITN$ $=$ $90^{\circ}$ and using the fact that $N$ is the center of $\odot(BIC)$, we get that $T$ is the midpoint of $\overline{IK}$. Suppose $I_A$ is the $A$-excenter of $\triangle ABC$ $\implies$ $\overline{II_A}$ is the diameter of $\odot(BIC)$ and therefore, $\angle IKI_A$ $=$ $90^{\circ}$ $=$ $\angle I_AAM$ $\implies$ $AMI_AK$ is cyclic too. Applying radical axis theorem on $\odot(AMI_AK)$, $\odot(ABC)$ and $\odot(BIC)$, we get that $MA$, $BC$ and $I_AK$ are concurrent. Hence, $AIDK$ is a cyclic quadrilateral.
  \end{proof}

  Using the fact that $\overline{TK}$ $=$ $\overline{TI}$ and previously established angle relation $\angle DIT$ $=$ $\angle IAT$, we get that
  \begin{align*}
    \overline{TP} \cdot \overline{TA} = \overline{TI}^2 = \overline{TK}^2
  \end{align*}
  Hence $\overline{TK}$ is tangent to $\odot(APK)$ at $K$. This implies that
  \begin{align*}
    \angle KPT = \angle AKT = \angle ADI = \angle ADS = \angle APS
  \end{align*}
  which implies that $K$ lies on $PS$. The final claim is showing that $Q$ lies on the line $\overline{KPS}$ too. Suppose $Q'$ is the intersection of $\overline{SK}$ and $\odot(BIC)$. Then,
  \begin{align*}
    \angle BFP = \angle FEP = \angle FED - \angle PED = \angle BID - \angle KID = \angle BIK = \angle BQ'P
  \end{align*}
  So $BFQ'P$ is cyclic and similarly, $CEQ'P$ is cyclic too. Therefore $Q'$ must be $Q$ and we get $Q$ lies on $\overline{SP}$
\end{proof}


\subsection{Exercises}
\begin{exercise}[USA TST 2016]
  Let $ABC$ be a scalene triangle with circumcircle $\Omega$, and suppose the incircle of $ABC$ touches $BC$ at $D$. The angle bisector of $\angle A$ meets $BC$ and $\Omega$ at $E$ and $F$. The circumcircle of $\triangle DEF$ intersects the $A$-excircle at $S_1$, $S_2$, and $\Omega$ at $T \neq F$. Prove that line $AT$ passes through either $S_1$ or $S_2$.
\end{exercise}

\begin{exercise}[IMO Shortlist 1999]
  Given a triangle $ABC$. The points $A$, $B$, $C$ divide the circumcircle $\Omega$ of the triangle $ABC$ into three arcs $BC$, $CA$, $AB$. Let $X$ be a variable point on the arc $AB$, and let $O_{1}$ and $O_{2}$ be the incenters of the triangles $CAX$ and $CBX$. Prove that the circumcircle of the triangle $XO_{1}O_{2}$ intersects the circle $\Omega$ in a fixed point.
\end{exercise}

\begin{exercise}[IMO Shortlist 2016]
  Let $ABC$ be a triangle with circumcircle $\Gamma$ and incenter $I$ and let $M$ be the midpoint of $\overline{BC}$. The points $D$, $E$, $F$ are selected on sides $\overline{BC}$, $\overline{CA}$, $\overline{AB}$ such that $\overline{ID} \perp \overline{BC}$, $\overline{IE}\perp \overline{AI}$, and $\overline{IF}\perp \overline{AI}$. Suppose that the circumcircle of $\triangle AEF$ intersects $\Gamma$ at a point $X$ other than $A$. Prove that lines $XD$ and $AM$ meet on $\Gamma$.
\end{exercise}

\begin{exercise}[IMO Shortlist 2017]
  In triangle $ABC$, let $\omega$ be the excircle opposite to $A$. Let $D, E$ and $F$ be the points where $\omega$ is tangent to $BC, CA$, and $AB$, respectively. The circle $AEF$ intersects line $BC$ at $P$ and $Q$. Let $M$ be the midpoint of $AD$. Prove that the circle $MPQ$ is tangent to $\omega$.
\end{exercise}

\section{Practice Problems}
\begin{exercise}[Japan 2009]
  Let $ \Gamma$ be a circumcircle. A circle with center $ O$ touches to line segment $ BC$ at $ P$ and touches the arc $ BC$ of $ \Gamma$ which doesn't have $ A$ at $ Q$. If $ \angle {BAO} = \angle {CAO}$, then prove that $ \angle {PAO} = \angle {QAO}$.
\end{exercise}

\begin{exercise}[ELMO Shortlist 2012]
  Circles $\Omega$ and $\omega$ are internally tangent at point $C$. Chord $AB$ of $\Omega$ is tangent to $\omega$ at $E$, where $E$ is the midpoint of $AB$. Another circle, $\omega_1$ is tangent to $\Omega, \omega,$ and $AB$ at $D,Z,$ and $F$ respectively. Rays $CD$ and $AB$ meet at $P$. If $M$ is the midpoint of major arc $AB$, show that $\tan \angle ZEP = \tfrac{PE}{CM}$.
\end{exercise}

\begin{exercise}[EGMO 2018]
  Let $\Gamma $ be the circumcircle of triangle $ABC$. A circle $\Omega$ is tangent to the line segment $AB$ and is tangent to $\Gamma$ at a point lying on the same side of the line $AB$ as $C$. The angle bisector of $\angle BCA$ intersects $\Omega$ at two different points $P$ and $Q$. Prove that $\angle ABP = \angle QBC$.
\end{exercise}

\begin{exercise}[IMO Shortlist 1992]
  Two circles touch externally at a point $ I$. The two circles lie inside a large circle and both touch it. The chord $ BC$ of the large circle touches both smaller circles (not at $ I$). The common tangent to the two smaller circles at the point $ I$ meets the large circle at a point $ A$, where the points $ A$ and $ I$ are on the same side of the chord $ BC$. Show that the point $ I$ is the incenter of triangle $ ABC$.
\end{exercise}

\begin{exercise}[Romania 1997]
  Let $ABC$ be a triangle, $D$ be a point on side $BC$, and let $\mathcal{O}$ be the circumcircle of triangle $ABC$. Show that the circles tangent to $\mathcal{O},AD,BD$ and to $\mathcal{O},AD,DC$ are tangent to each other if and only if $\angle BAD=\angle CAD$.
\end{exercise}

\begin{exercise}[USAMO 2017]
  Let $ABC$ be a scalene triangle with circumcircle $\Omega$ and incenter $I$. Ray $AI$ meets $\overline{BC}$ at $D$ and meets $\Omega$ again at $M$; the circle with diameter $\overline{DM}$ cuts $\Omega$ again at $K$. Lines $MK$ and $BC$ meet at $S$, and $N$ is the midpoint of $\overline{IS}$. The circumcircles of $\triangle KID$ and $\triangle MAN$ intersect at points $L_1$ and $L_2$. Prove that $\Omega$ passes through the midpoint of either $\overline{IL_1}$ or $\overline{IL_2}$.
\end{exercise}

\begin{exercise}[ELMO Shortlist 2017]
  Let $ABC$ be an acute triangle with incenter $I$ and circumcircle $\omega$. Suppose a circle $\omega_B$ is tangent to $BA,BC$, and internally tangent to $\omega$ at $B_1$, while a circle $\omega_C$ is tangent to $CA, CB$, and internally tangent to $\omega$ at $C_1$. If $B_2, C_2$ are the points opposite to $B,C$ on $\omega$, respectively, and $X$ denotes the intersection of $B_1C_2, B_2C_1$, prove that $XA=XI$.
\end{exercise}

\begin{exercise}[IMO Shortlist 1999]
  Two circles $\Omega_{1}$ and $\Omega_{2}$ touch internally the circle $\Omega$ in M and N and the center of $\Omega_{2}$ is on $\Omega_{1}$. The common chord of the circles $\Omega_{1}$ and $\Omega_{2}$ intersects $\Omega$ in $A$ and $B$. $MA$ and $MB$ intersects $\Omega_{1}$ in $C$ and $D$. Prove that $\Omega_{2}$ is tangent to $CD$.
\end{exercise}

\begin{exercise}[IMO Shortlist 2014]
  Let $ABC$ be a triangle with circumcircle $\Omega$ and incentre $I$. Let the line passing through $I$ and perpendicular to $CI$ intersect the segment $BC$ and the arc $BC$ (not containing $A$) of $\Omega$ at points $U$ and $V$ , respectively. Let the line passing through $U$ and parallel to $AI$ intersect $AV$ at $X$, and let the line passing through $V$ and parallel to $AI$ intersect $AB$ at $Y$ . Let $W$ and $Z$ be the midpoints of $AX$ and $BC$, respectively. Prove that if the points $I, X,$ and $Y$ are collinear, then the points $I, W ,$ and $Z$ are also collinear.
\end{exercise}

\begin{exercise}[Taiwan TST 2014]
  Let $M$ be any point on the circumcircle of triangle $ABC$. Suppose the tangents from $M$ to the incircle meet $BC$ at two points $X_1$ and $X_2$. Prove that the circumcircle of triangle $MX_1X_2$ intersects the circumcircle of $ABC$ again at the tangency point of the $A$-mixtilinear incircle.
\end{exercise}

\begin{exercise}[Taiwan TST 2015]
  Let $O$ be the circumcircle of the triangle $ABC$. Two circles $O_1,O_2$ are tangent to each of the circle $O$ and the rays $\overrightarrow{AB},\overrightarrow{AC}$, with $O_1$ interior to $O$, $O_2$ exterior to $O$. The common tangent of $O,O_1$ and the common tangent of $O,O_2$ intersect at the point $X$. Let $M$ be the midpoint of the arc $BC$ (not containing the point $A$) on the circle $O$, and the segment $\overline{AA'}$ be the diameter of $O$. Prove that $X,M$, and $A'$ are collinear.
\end{exercise}

\begin{exercise}[Sharygin 2021]
  Let $ABC$ be a scalene triangle, $AM$ be the median through $A$, and $\omega$ be the incircle. Let $\omega$ touch $BC$ at point $T$ and segment $AT$ meet $\omega$ for the second time at point $S$. Let $\delta$ be the triangle formed by lines $AM$ and $BC$ and the tangent to $\omega$ at $S$. Prove that the incircle of triangle $\delta$ is tangent to the circumcircle of triangle $ABC$.
\end{exercise}

\begin{exercise}
  Let $ABC$ be an acute scalene triangle with orthocentre $H$ and circumcircle $\Gamma$. Let $M$ be the midpoint of $BC$. Lines $BH$ and $CH$ meet sides $AC$ and $AB$ at points $B_1$ and $C_1$ respectively. The lines through $M$ perpendicular to $AB$ and $AC$ respectively meet line $B_1C_1$ at points $B_2$ and $C_2$ respectively. The perpendicular bisectors of $BC$ and $B_2C_2$ meet at point $T$. Prove that lines $TA$ and $MH$ meet at a point on $\Gamma$.
\end{exercise}




\end{document}